\documentclass{templateNote}
\usepackage{tcolorbox}

\definecolor{Violeta}{RGB}{124,0,254}
\definecolor{Verde2}{RGB}{130,254,0} % Complementario
\definecolor{Naranja2}{RGB}{254,124,0} % Triadico
\definecolor{Verde3}{RGB}{0,254,124} % Triadico

\definecolor{Amarillo}{RGB}{249,228,0}
\definecolor{Azul2}{RGB}{0,21,249} % Complementario
\definecolor{Celeste2}{RGB}{0,249,228} % Triadico
\definecolor{Rosa}{RGB}{228,0,249} % Triadico

\definecolor{Naranja}{RGB}{255,175,0}
\definecolor{Azul3}{RGB}{0,80,255} % Complementario
\definecolor{Cian}{RGB}{0,255,175} % Triadico
\definecolor{Violeta2}{RGB}{175,0,255} % Triadico

\definecolor{Rojo}{RGB}{245,0,79}
\definecolor{Verde4}{RGB}{0,245,166} % Complementario
\definecolor{Verde}{RGB}{79,245,0} % Triadico
\definecolor{Azul}{RGB}{0,79,245} % Triadico

\definecolor{Celeste}{RGB}{0,191,255}
\definecolor{Salmon}{RGB}{255,0,157}

\newcommand{\newparagraph}{\par\vspace{\baselineskip}\noindent}
\newcommand{\hlcolor}[2]{{\sethlcolor{#1}\hl{#2}}}
\newcommand{\comillas}[1]{``#1''}
\tcbuselibrary{skins}
\usetikzlibrary{shadings}
\tcbset{
  base/.style={
    empty,
    frame engine=path,
    colframe=white,
    sharp corners,
    %title={Comentario \thetcbcounter},
    attach boxed title to top left={yshift*=-\tcboxedtitleheight},
    boxed title style={size=minimal, top=4pt, left=4pt},
    coltitle=black,
    fonttitle=\large\it,
  }
}
\newtcolorbox{CuadroPersonalizado}[4]{%
  base,
  title={#2}, % Título personalizado #2 = nombre contador ; #3 = Título
  drop fuzzy shadow, % Sombra del cuadro de texto
  coltitle=#1,
  borderline west={3pt}{-3pt}{#3}, % Borde Izquierdo #4 = color del borde
  attach boxed title to top left={xshift=-3mm, yshift*=-\tcboxedtitleheight/2},
  boxed title style={right=3pt, bottom=3pt, overlay={
    \draw[draw=#4, fill=#4, line join=round]
      (frame.south west) -- (frame.north west) -- (frame.north east) --
      (frame.south east) -- ++(-2pt, 0) -- ++(-2pt, -4pt) --
      ++(-2pt, 4pt) -- cycle; % #5 = color del fondo
  }}, % Cuadro de titulo
  overlay unbroken={
    \scoped \shade[left color=#4!30!black, right color=#4]
    ([yshift=-0.2pt]title.south west) -- ([xshift=-3pt, yshift=-0.2pt]title.south-|frame.west) -- ++(0, -4pt) -- cycle;
  }, % Sombra de titulo #5 = color del titulo
}
\begin{document}

\imagenlogoU{img/LogoElNube.png}
\linklogoU{https://github.com/MarceloPazPezo}
\linkQRDoc{https://github.com/MarceloPazPezo/MyRepo/tree/main/Icinf}
\titulo{Certamen 1}
\asignatura{Formulación y Evaluación de Proyectos}
\autor{
Marcelo Paz
}
\vDoc{1.0.0}
\tipoDoc{Apunte}

% Metadatos del PDF
\title{[\asignatura]-\titulo}
\author{
    \autor
}
\portada
\margenes % Crear márgenes

\section{¿Qué es un proyecto?}
\textbf{Definición:} \hlcolor{Amarillo!50}{Es una concepción formal y fundamentada de una idea que pretende materializarse.} Es una suerte de \textit{''borrador de la realidad futura''} en la que se incorporan las previsiones acerca de todos los elementos del entorno, sobre los cuales influirá y actuará la idea que pretendemos materializar.
\newline
Se habla de:\newparagraph

\begin{minipage}[t]{0.45\textwidth}
  \begin{itemize}
    \item Proyecto de Ley.
    \item Proyecto de País.
    \item Proyecto Deportivo.
    \item Proyecto inmobiliario.
  \end{itemize}
\end{minipage}
\begin{minipage}[t]{0.45\textwidth}
  \begin{itemize}
    \item Proyecto de ampliación.
    \item Proyecto de empresa.
    \item Proyecto de modernización.
  \end{itemize}
\end{minipage}

\section{Tipos de proyecto}
\begin{itemize}
  \item \textbf{Proyecto de Negocio o Proyecto de empresa (Business Plan).}
  \item \textbf{Proyecto de Modernización.}
  \item \textbf{Proyecto de Amplificación.}
\end{itemize}

\section{Etapas de un proyecto}
\begin{itemize}
  \item \textbf{Preparación y Evaluación de Proyectos.}
  \begin{itemize}
    \item Idea.
    \item Perfil.
    \item Prefactibilidad.
    \item Factibilidad.
  \end{itemize}

  \item \textbf{Administración y Control de Proyectos.}
  \begin{itemize}
    \item Diseño.
    \item Ejecución.
    \item Operación.
  \end{itemize}
\end{itemize}

\newpage
\section{Estudios del Proyecto}
\begin{itemize}
  \item \hyperlink{EM}{\textbf{Estudio de Mercado}.}
  \item \hyperlink{ET}{\textbf{Estudio Técnico}.}
  \item \hyperlink{EO}{\textbf{Estudio Organizacional}.}
  \item \hyperlink{EL}{\textbf{Estudio Legal}.}
  \item \hyperlink{EEF}{\textbf{Estudio Económico Financiero}.}
\end{itemize}

\subsection{Estudio de mercado}\hypertarget{EM}{}
\textbf{DEF.} \hlcolor{Amarillo!50}{Es la recopilación, procesamiento y análisis de información respecto del diseño, producción, comercialización y consumo de bienes y servicios.}
\begin{itemize}
  \item \textbf{Mostrara:}
  \begin{itemize}
    \item Quiénes son los destinatarios del proyecto.
    \item Las necesidades que el proyecto abordará y que justifican su implementación, describiendo el diseño de los productos y servicios generados y cómo satisfacen las necesidades detectadas.
  \end{itemize}
  \item  \textbf{Cuantificará:}
  \begin{itemize}
    \item Los beneficios que el proyecto generará a los destinatarios del proyecto.
  \end{itemize}
  \item \textbf{Describe:}
  \begin{itemize}
    \item Las condiciones del entorno en que el proyecto se implementa, condiciones económicas generales, la competencia, los proveedores de insumos, y distribuidores.
  \end{itemize}
\end{itemize}
\begin{CuadroPersonalizado}{black}{Comentario del profesor}{Verde2!90!black}{Verde2!90!black}
  \begin{itemize}
    \item El cliente es un beneficiario de un bien o servicio.
    \item El estudio del mercado le da el contexto y la información para el nacimiento de un proyecto.
    \item El mercado \textbf{NO} es algo abstracto.
    \item Es importante estudiar una amplia gama de variables, como la competencia, la demanda, la oferta, los precios, los canales de distribución, entre otros.
    \item Un ejemplo de ventaja competitiva es con \textbf{LIDER} que tiene productos con un bajo precio y del \textbf{JUMBO} que tiene productos de calidad/variedad.
  \end{itemize}
\end{CuadroPersonalizado}

\begin{CuadroPersonalizado}{black}{Extra: 'Referencias a uno de los libros'}{Naranja!90!black}{Naranja!90!black}
  \begin{itemize}
    \item Análisis del Entorno del Mercado.
    \item Análisis del Cliente
    \begin{itemize}
      \item Análisis del Cliente y sus necesidades.
      \item Determinación del mercado objetivo.
      \item Definición de Estrategia de Posicionamiento.
      \item Determinación de la ventaja competitiva.
    \end{itemize}
    \item Análisis del Producto.
    \begin{itemize}
      \item Estudio de atributos de los productos.
    \end{itemize}
    \item Análisis del Precio.
    \item Estimación de la Demanda.
    \begin{itemize}
      \item Medición de mercados potenciales. Principales segmentos. Grado de penetración.
      \item Pronóstivo de Demanda Industrial.
      \item Estimación de la participación de mercado de la empresa.
      \item Análisis de las ventas de la empresa.
    \end{itemize}
  \end{itemize}
\end{CuadroPersonalizado}

\begin{CuadroPersonalizado}{black}{Ejercicio}{Celeste!90!black}{Celeste!90!black}
  \textbf{Caso:} Antes se realizaba el control de inventario utilizando lapiz y papel.
  \newline
  \textbf{Análisis:}
  \begin{itemize}
    \item El hacer el control manual requeria muchas HH.
    \item Esto genera un costo muy alto e innecesario pues hace perder ganancias (sobre todo en el caso de las pymes).
  \end{itemize}
  \textbf{Solución:} Se implementa un sistema de control de inventario digital/automatico, el cual reduce ese costo innecesario.
\end{CuadroPersonalizado}

\newpage
\subsubsection{Análisis del producto}
\textbf{DEF.} \hlcolor{Amarillo!50}{Es todo aquello que por su adquisición, uso o consumo puede satisfacer un deseo o una necesidad.} Un producto puede ser un objeto material, una persona, una idea, un lugar, una experiencia. \newline
Todo producto es un conjunto de atributos o características que tienen una utilidad o función base, y varias utilidades secundarias de distinto tipo que complementan la utilidad básica (estética, imagen, prestigio, conveniencia).
\begin{CuadroPersonalizado}{black}{Comentario del profesor}{Verde2!90!black}{Verde2!90!black}
  \begin{itemize}
    \item Cuando el caso de uso esta mal formulado, el producto falla o no cumple con las expectativas.
  \end{itemize}
\end{CuadroPersonalizado}
\textbf{Niveles:}
\begin{enumerate}
  \item \textbf{Nivel Básico:} Es la función principal del producto.
  \item \textbf{Nivel Esperado:} Son las características que el cliente espera del producto.
  \item \textbf{Nivel Aumentado:} Son las características que el cliente no espera pero que le aportan valor. 'Garantía'.
  \begin{itemize}
    \item Garantias.
    \item Transporte
    \item Instalacion
  \end{itemize}
\end{enumerate}
\begin{CuadroPersonalizado}{black}{Ejemplo}{Celeste!90!black}{Celeste!90!black}
  \textbf{Caso:} Una de las tendencias demográficas más marcadas hoy en día es el crecimiento de la población de la llamada \textit{\comillas{tercera edad}}. 

  \textbf{Se pide:} Utilizando el modelo de producto multiatributo, diseñar un producto/ servicio de centro de cuidado y atención al adulto mayor que se adecue a necesidades del segmento del mercado descrito.
  \newline
  \textbf{Solución:}
  \begin{itemize}
    \item \textbf{Producto Basico:} Lugar para residencia de adultos mayores.
    \item \textbf{Producto Real:}
    \begin{itemize}
      \item Características:
      \begin{itemize}
        \item Residencia permanente full-time; o part time.
        \item Habitaciones individuales; baño individual; T.V.; Wi-Fi; etc.
        \item Espacios abiertos; jardines; instalación deportiva; etc.
        \item Enfermería; kinesiologia; atenciones especiales; etc.
        \item Clases de gimnasia; juegos de mesas; etc.
      \end{itemize}
      \item Diseño:
      \begin{itemize}
        \item Limpieza; orden; Confort; Luminozidad; Seguridad.
      \end{itemize}
      \item Calidad:
      \begin{itemize}
        \item 90\% Encuesta de satisfacción.
        \item 5\% Retiro de clientes por año.
        \item $ < $ 3 incidencias por mes.
      \end{itemize}
      \item Nombre: \textit{'Residencia Santa Clara'}.
      \item Empaque:
      \begin{itemize}
        \item Contrato.
        \item Folleto informativo.
      \end{itemize}
    \end{itemize}
  \end{itemize}
\end{CuadroPersonalizado}
\newpage
\subsection{Estudio Técnico}\hypertarget{ET}{}
\textbf{DEF.}
\begin{itemize}
  \item \textbf{Mostrará:}
  \begin{itemize}
    \item Que el proyecto considera la alternativa tecnolpogica óptima de acuerdo a las posibilidades disponibles.
  \end{itemize}
  \item \textbf{Deberá:}
  \begin{itemize}
    \item Desplegar un completo \textit{'lay out'} los procesos productivos, operacionales y administrativos que se van a implementar.
  \end{itemize}
  \item \textbf{Proveerá:}
  \begin{itemize}
    \item La información necesaria para cuantificar el monto de las inversiones y los costos de operación (ejecución) del proyecto.
  \end{itemize}
\end{itemize}

\newpage
\subsection{Estudio Organizacional}\hypertarget{EO}{}
\textbf{DEF.}
\begin{itemize}
  \item \textbf{Se comprende:}
  \begin{itemize}
    \item La conformación del equipo de personas que desempeñarán las operaciones del proyecto y en definitiva, quienes tendrán la responsabilidad de materializarlo.
  \end{itemize}
  \item Los requirimientos de personal llevan a estimar los costos laborales, un importante ítem de los costos de operación.
  \item \textbf{Permite:}
  \begin{itemize}
    \item Conocer las posibiliades de contar con el personal idóneo para el éxito del proyecto. Dentro de este campo deben considerarse también lo relacionado con los procesos de gestión y la disponibilidad de los sistemas administrativos necesarios para operarlos.
  \end{itemize}
  \item Mención especial debe hacerse al equipo directivo que encabezará la realización del proyecto. Esto es particularmente crítico en los proyectos de nuevos negocios o de formación de nuevas empresas.
\end{itemize}

\newpage
\subsection{Estudio Legal}\hypertarget{EL}{}
\textbf{DEF.}
\begin{itemize}
  \item Este aspecto es crucial en determinados emprendimientos, y consiste en la delimitación de las normas jurídicas que afectarán el desarrollo de las operaciones.
  \item Los aspectos legales más comunes son:
  \begin{itemize}
    \item Normas de constitución y funcionamiento de sociedades y corporaciones.
    \item Legislación de comercio.
    \item Normas de protección al consumidor.
    \item Legislación tributaria, arancelaria y de derechos municipales.
    \item Normas sanitarias, de seguridad, ambientales, de urbanismo.
    \item Normas laborales.
    \item Normas de propiedad industrial y marcas comerciales.
  \end{itemize}
\end{itemize}

\newpage
\subsection{Estudio Económico Financiero}\hypertarget{EEF}{}
\textbf{DEF.}
\begin{itemize}
  \item En este an\'alisis predominan las consideraciones econ\'omicas de costos y beneficios del proyecto en t\'erminos monetarios.
  \item Debe resumirse sistemáticamente toda la información financiera asociada al proyecto la cual es derivada de los estudios mencionados anteriormente.
  \item \textbf{Se identifican:}
  \begin{itemize}
    \item Se identifican las inversiones, los ingresos (beneficios) y los costos operacionales y los valores residuales de las inversiones.
  \end{itemize}
  \item \textbf{Se determina:}
  \begin{itemize}
    \item La forma de financiamiento de las inversiones.
    \item Los flujos de caja que generará la realización del proyecto.
  \end{itemize}
  \item \textbf{Se calculan:}
  \begin{itemize}
    \item Diversos indicadores de evaluación de inversiones como el período de recuperación, el valor actual neto y la tasa interna de retorno.
  \end{itemize}
  \item \textbf{Se analiza:}
  \begin{itemize}
    \item El sensibilidad de los indicadores (principalmente el Valor Actual Neto) a la variación de determinados parámetros.
    \item El nivel de riesgo de la inversión a través de la volatilidad de los flujos de caja.
  \end{itemize}
\end{itemize}
% \begin{center}
%     \begin{overpic}[width=0.3\textwidth]{img/LogoElNube.png}
%         \put(5,50){\colorbox{Verde}{Verde}}
%         \put(5,40){\colorbox{Morado}{Morado}}
%         \put(5,30){\colorbox{Celeste}{Celeste}}
%         \put(5,20){\colorbox{Salmon}{Salmon}}
%         \put(5,10){\colorbox{RosaSuave}{RosaSuave}}
%         \put(60,50){\colorbox{Turquesa}{Turquesa}}
%         \put(60,40){\colorbox{Menta}{Menta}}
%         \put(60,30){\colorbox{Melocoton}{Melocoton}}
%         \put(60,20){\colorbox{AmarilloVainilla}{AmarilloVainilla}}
%         \put(60,10){\colorbox{Gris}{Gris}}
%     \end{overpic}
% \end{center}
\end{document}