\documentclass{templateNote}
\usepackage{tcolorbox}
\usepackage{amsmath}
\usepackage{amssymb}
\usepackage{pgfplots}
\usepackage{pgf-pie}
\usepackage{tabularx}
\usepackage[ruled,vlined]{algorithm2e}
\usepackage{soul}

\pgfplotsset{compat=1.18}
\newcolumntype{L}{>{\centering\arraybackslash}X}
\begin{document}

\imagenlogo{img/LogoElNube.png}
\universidad{Universidad del Bío-Bío}
\titulo{Apunte Certamen 2} % Titulo
\asignatura{Analisis y Diseño de Algoritmos} % Asignatura
\autor{
    Marcelo \textsc{Paz}
}
\portada
\margenes % Crear márgenes

\section{Backtracking}
\begin{itemize}
    \item Es una meta-heurística que trata de podar el árbol de búsqueda, el cuál se va generando de manera dinámica.
    
    \item Se descartan soluciones intermedias que se puede determinar no llegarán a una solución.
    
    \item La búsqueda se hace en profundidad.

    Para realizar backtracking, necesitamos:
    \begin{enumerate}
        \item Punto de partida del árbol.
        
        \item Función de rechazo.
        
        \item Función de aceptación.
        
        \item Funciones de hijo (primero y siguiente).
        
        \item Función de Output (completar).
    \end{enumerate}
\end{itemize}

\begin{algorithm*}[H]
    % \SetAlgoLined
    \SetKwFunction{FMain}{Backtracking}
    \SetKwProg{Fn}{Función}{:}{}
    \Fn{\FMain{c}}{
        \If{reject(P,c)}{
            \KwRet\;
        }

        \If{accept(P,c)}{
            output(P,c)\;
            exit(0)\;
        }
        s = first(P,c)\;

        \While{$s \neq \land$}{
            Backtracking(s)\;
            s = next(P,s)\;
        }
    }
    \caption{PseudoCódigo Backtracking}
\end{algorithm*}

% \newpage
\subsection*{Eficiencia de backtracking}

\begin{itemize}
    \item \textbf{Función de rechazo:} Mientras más cerca de la raíz, mejor.
    
    \item \textbf{Función de aceptación:} Mientras más cerca de la raíz, mejor.
    
    \item \textbf{Funciones de hijo (primero y siguiente):} Mientras más restrictiva, mejor.
    
    \item \textbf{Función de Output (completar):} Mientras más eficiente, mejor.
\end{itemize}

\subsection*{Cuando usar backtracking}
\begin{itemize}
    \item BackTracking está considerado para resolver Problemas de Satisfacción de Restricciones (CSP).
    Éste se define como los problemas consistentes de una tripleta $(X,D,C)$.
    \subitem $X$= Conjunto de variables.
    \subitem $D$ = Conjunto de dominios.
    \subitem $C$ = Conjunto de restricciones.
    
    \item Cada variable $x_i \in d_i$ y debe satisfacer $c_i$, el cual está formado por una relación entre elementos de $X$.
    
\end{itemize}


\subsection*{Ejemplos de problemas CSP}

\begin{tcolorbox}[colback=blue!4!white,colframe=blue!75!black,title=Aritmética Verbal]
    Dado un problema de aritmética verbal, encontrar la solución.
\end{tcolorbox}

\begin{tcolorbox}[colback=blue!4!white,colframe=blue!75!black,title=Coloración de mapas]
    Dado un mapa, colorear las regiones de tal forma que dos regiones adyacentes no tengan el mismo color.
\end{tcolorbox}

\begin{tcolorbox}[colback=blue!4!white,colframe=blue!75!black,title=Crucigramas]
    Dado un crucigrama, encontrar las palabras que lo completan.
\end{tcolorbox}

\begin{tcolorbox}[colback=blue!4!white,colframe=blue!75!black,title=Sudoku]
    Dado un tablero de $9 \times 9$, colocar los números del 1 al 9 de tal forma que no se repitan en la misma fila, columna o submatriz de $3 \times 3$.
\end{tcolorbox}

\begin{tcolorbox}[colback=blue!4!white,colframe=blue!75!black,title=Problema de las $n$ reinas]
    Dado un tablero de $n \times n$, colocar $n$ reinas de tal forma que no se ataquen entre ellas.
\end{tcolorbox}

\newpage

\section{Algoritmos Probabilísticos}
\begin{itemize}
    \item Los algoritmos probabilísticos son aquellos que introducen elementos al azar dentro de su lógica.
    \item Puede que el tiempo, la memoria o la respuesta sean afectados positivamente (o negativa con baja probabilidad) por el azar.
\end{itemize}

\subsection*{Tipos}
\begin{itemize}
    \item Algoritmos Numéricos.
    \begin{itemize}
        \item Estos algoritmos dan una respuesta aproximada al problema que se quiere resolver.

        \item Su precisión mejora conforme se realizan más ciclos de iteración.
    \end{itemize}

    \item \textbf{Algoritmos Monte Carlo} (Puede mentir).
    \begin{itemize}
        \item Se utilizan cuando no existen formas eficientes de resolver un problema de otra manera.

        \item \hl{Estos algoritmos dan la respuesta exacta, pero puede dar una solución errada, con probabilidad baja.}

        \item Mientras más larga la ejecución, mayor es la probabilidad de que la respuesta se la correcta.
        
        Se le dice a un algoritmo tipo Monte Carlo p-correcto si:
        \begin{itemize}
            \item La solución regresada es correcta con probabilidad p>0,5, no importando el dato de entrada.
            \item p puede depender del tamaño de la entrada, pero no de los datos de la entrada.
        \end{itemize}

    \end{itemize}
    
    \item \textbf{Algoritmos Las Vegas} (No miente, dice que no puede dar la respuesta correcta).
    \begin{itemize}
        \item El algoritmo tipo Las Vegas funciona similar a Monte Carlo, pero cuando no puede dar una respuesta correcta, lo admite.
        
        Se distinguen 2 sub-tipos en general:
        \begin{itemize}
            \item Siempre encuentra una solución correcta. Si el azar no beneficia a la ejecución, esta tomará más tiempo.
            \item \hl{A veces no es capaz de dar una solución, lo cual admite.}
        \end{itemize}
    \end{itemize}
\end{itemize}

\section{Algoritmos Genéticos}
\begin{itemize}
    \item Algoritmo genético es una metaheurística que se inspira en la evolución y selección natural para resolver problemas de optimización.
    
    \item En términos coloquiales, genera una población inicial y la hace evolucionar miles o millones de años, para finalmente elegir al individuo más fuerte.
    
    \newpage
    En términos generales, los tópicos ligados a un algoritmo genético son los siguientes:
    \begin{enumerate}
        \item \textbf{Población Inicial:} La Población Inicial consta de soluciones (ya sean parciales y/o totales).

        \item \textbf{Función de aptitud:}La Fitness Function o Función de Aptitud nos dice que tan apta es la solución para el problema.
        
        \item \textbf{Algoritmo de combinación (sexo):} Básicamente acá es que los individuos con mejor fitness tienen mayores posibilidades de procrear.

        \item \textbf{Algoritmo y tasa de Mutación:} Con alguna probabilidad (generalmente muy baja), cada bit de hijos puede cambiar.
        \begin{itemize}
            \item Mutación bit por string.
            
            \item Flip.
            
            \item Límite (para números).
            
            \item No uniforme.
            
            \item Uniforme.
            
            \item Gausiano.
            
            \item Shrink.
        \end{itemize}
        \item \textbf{Selección:} Hay varias versiones sobre supervivencia a la siguiente generación. Una común es que sobrevivan los mejores (siempre mismo tamaño).

    \end{enumerate}

\end{itemize}

\section{Búsqueda Informada (Heurística)}
La búsqueda no informada trata ciegamente de encontrar una solución.
\begin{itemize}
    \item Fuerte: Se usa heurística para tratar de resolver el problema lo mejor posible, pero no se asegura la solución.
    \item Débil: Heurística se conjuga con un método riguroso para llegar a la mejor solución. Muchas veces sigue siendo infactible de resolver en el peor caso.
\end{itemize}

\section{Complejidad Computacional}
\begin{itemize}
    \item La complejidad computacional trata sobre clasificación de problemas.
    \subitem Dificultad inherente.
    \subitem Clases de complejidad y sus relaciones.
    
    \item Mide Tiempo y Espacio utilizando modelos de cómputo.
\end{itemize}

\section{Teoría Algorítmica de la Información}

\section{Dureza, Completitud y Reducciones}

\section{Clases de Complejidad}
\end{document}