\documentclass{templateNote}
\usepackage{tcolorbox}


\begin{document}

\imagenlogo{img/LogoElNube.png}
\universidad{Universidad del Bío-Bío}
\titulo{Guia 2 SQL: Consultas} % Titulo
\asignatura{Base de Datos} % Asignatura
\autor{
    \indent
    Marcelo \textsc{Paz}
}   
\portada
\margenes % Crear márgenes

\section{Teoria}
\subsection{Subconsulta EXISTS}
\begin{tcolorbox}
    [colback=gray!5!white,colframe=gray!75!black,fonttitle=\bfseries,title=SQL]
    \begin{sqlcode}
    \end{sqlcode}
    \begin{verbatim}
SELECT * FROM table1
WHERE EXISTS (SELECT *
              FROM table2
              WHERE table1.column = table2.column);
    \end{verbatim}
\end{tcolorbox}
\indent
*OBS: Es necesario hacer un join.
\subsection{Subconsulta IN}
\begin{tcolorbox}
    [colback=gray!5!white,colframe=gray!75!black,fonttitle=\bfseries,title=SQL]
    \begin{sqlcode}
    \end{sqlcode}
    \begin{verbatim}
SELECT * FROM table1
WHERE table1.column IN (SELECT table2.column
                        FROM table2);
    \end{verbatim}
\end{tcolorbox}
\indent
*OBS: Es necesario que la subconsulta devuelva una sola columna a comparar.

\section{Ejercicios}
\subsection{Base de datos}

\begin{tcolorbox}
    [colback=gray!5!white,colframe=gray!75!black,fonttitle=\bfseries,title=SQL]
    \begin{sqlcode}
    \end{sqlcode}
    \begin{verbatim}
CREATE TABLE CLIENTE (
    NCLIENTE INTEGER PRIMARY KEY,
    NOMBRE VARCHAR(10),
    CIUDAD VARCHAR(10)
);
INSERT INTO CLIENTE VALUES (1,'MARIA', 'TALCAHUANO');
INSERT INTO CLIENTE VALUES (2,'JOSE', 'CHILLAN');
INSERT INTO CLIENTE VALUES (3,'FRANCISCA', 'TOME');
INSERT INTO CLIENTE VALUES (4,'GUSTAVO', 'TALCAHUANO');
CREATE TABLE PRODUCTO (
    COD INTEGER PRIMARY KEY,
    DESCRIPCION VARCHAR(15),
    TIPO VARCHAR(10),
    PRECIO INTEGER
);
INSERT INTO PRODUCTO VALUES (11,'YOGURT','LACTEO', 350);
INSERT INTO PRODUCTO VALUES (12,'MANTEQUILLA','LACTEO',
2000);
INSERT INTO PRODUCTO VALUES (21,'ESPIRALES','PASTA', 1200);
INSERT INTO PRODUCTO VALUES (22,'CORBATAS','PASTA', 1200);
INSERT INTO PRODUCTO VALUES (31,'NARANJA','FRUTA', 1800);
CREATE TABLE VENTA (
    NCLIENTE INTEGER,
    COD INTEGER,
    FECHA DATE,
    CANTIDAD INTEGER,
    PRIMARY KEY (NCLIENTE, COD, FECHA),
    FOREIGN KEY (NCLIENTE) REFERENCES CLIENTE,
    FOREIGN KEY (COD) REFERENCES PRODUCTO
);
INSERT INTO VENTA VALUES (1,11,'01/10/2023',4);
INSERT INTO VENTA VALUES (1,11,'15/10/2023',1);
INSERT INTO VENTA VALUES (2,21,'23/09/2023',4);
INSERT INTO VENTA VALUES (2,31,'12/10/2023',1);
INSERT INTO VENTA VALUES (1,22,'18/10/2023',6);
INSERT INTO VENTA VALUES (3,31,'20/09/2023',3);
INSERT INTO VENTA VALUES (3,12,'07/10/2023',8);
    \end{verbatim}
\end{tcolorbox}

\newpage
\subsection{Subconsulta 1}
Muestre el nombre de los productos que se han vendido más de 5 unidades por venta.
\begin{tcolorbox}
    [colback=gray!5!white,colframe=gray!75!black,fonttitle=\bfseries,title=SQL]
    \begin{sqlcode}
    \end{sqlcode}
    \begin{verbatim}
SELECT P.DESCRIPCION
FROM PRODUCTO P
WHERE EXISTS (SELECT *
                FROM VENTA V
                WHERE V.COD = P.COD AND V.CANTIDAD > 5);
    \end{verbatim}
\end{tcolorbox}
\begin{tcolorbox}
    [colback=gray!5!white,colframe=gray!75!black,fonttitle=\bfseries,title=SQL]
    \begin{sqlcode}
    \end{sqlcode}
    \begin{verbatim}
SELECT P.DESCRIPCION
FROM PRODUCTO P
WHERE P.COD IN (SELECT V.COD
                FROM VENTA V
                WHERE V.CANTIDAD > 5);
    \end{verbatim}
\end{tcolorbox}

\subsection{Subconsulta 2}
Listar los productos que han sido vendidos a cliente de la ciudad de Talcahuano.
\begin{tcolorbox}
    [colback=gray!5!white,colframe=gray!75!black,fonttitle=\bfseries,title=SQL]
    \begin{sqlcode}
    \end{sqlcode}
    \begin{verbatim}
SELECT P.COD, P.DESCRIPCION
FROM PRODUCTO P
WHERE EXISTS (SELECT *
                FROM VENTA V, CLIENTE C
                WHERE V.COD = P.COD AND C.NCLIENTE = V.NCLIENTE
                AND C.CIUDAD ILIKE 'TALCAHUANO');
                -- ILIKE es para ignorar mayusculas
    \end{verbatim}
\end{tcolorbox}

\begin{tcolorbox}
    [colback=gray!5!white,colframe=gray!75!black,fonttitle=\bfseries,title=SQL]
    \begin{sqlcode}
    \end{sqlcode}
    \begin{verbatim}
SELECT P.COD, P.DESCRIPCION
FROM PRODUCTO P
WHERE P.COD IN (SELECT V.COD
                FROM VENTA V, CLIENTE C
                WHERE C.NCLIENTE = V.NCLIENTE 
                AND C.CIUDAD ILIKE 'TALCAHUANO');
    \end{verbatim}
\end{tcolorbox}

\newpage
\subsection{Subconsulta 3}
Muestre el número de los clientes que han comprado productos de tipo lácteos.
\begin{tcolorbox}
    [colback=gray!5!white,colframe=gray!75!black,fonttitle=\bfseries,title=SQL]
    \begin{sqlcode}
    \end{sqlcode}
    \begin{verbatim}
SELECT C.NCLIENTE
FROM CLIENTE C
WHERE EXISTS (SELECT *
                FROM VENTA V, PRODUCTO P
                WHERE C.NCLIENTE = V.NCLIENTE AND V.COD = P.COD
                AND P.TIPO = 'LACTEO');
    \end{verbatim}
\end{tcolorbox}

\begin{tcolorbox}
    [colback=gray!5!white,colframe=gray!75!black,fonttitle=\bfseries,title=SQL]
    \begin{sqlcode}
    \end{sqlcode}
    \begin{verbatim}
SELECT C.NCLIENTE
FROM CLIENTE C
WHERE C.NCLIENTE IN (SELECT V.NCLIENTE
                    FROM VENTA V, PRODUCTO P
                    WHERE V.COD = P.COD AND P.TIPO = 'LACTEO');
    \end{verbatim}
\end{tcolorbox}

\subsection{Subconsulta 4}
Listar el nombre de los clientes que no han comprado productos.
\begin{tcolorbox}
    [colback=gray!5!white,colframe=gray!75!black,fonttitle=\bfseries,title=SQL]
    \begin{sqlcode}
    \end{sqlcode}
    \begin{verbatim}
SELECT C.NOMBRE
FROM CLIENTE C
WHERE NOT EXISTS (SELECT *
                    FROM VENTA V
                    WHERE V.NCLIENTE = C.NCLIENTE);
    \end{verbatim}
\end{tcolorbox}

\begin{tcolorbox}
    [colback=gray!5!white,colframe=gray!75!black,fonttitle=\bfseries,title=SQL]
    \begin{sqlcode}
    \end{sqlcode}
    \begin{verbatim}
SELECT C.NOMBRE
FROM CLIENTE C
WHERE C.NCLIENTE NOT IN (SELECT V.NCLIENTE
                        FROM VENTA V);
    \end{verbatim}
\end{tcolorbox}

\newpage

\end{document}