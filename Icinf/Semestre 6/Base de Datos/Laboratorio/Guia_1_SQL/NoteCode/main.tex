\documentclass{templateNote}
\usepackage{tcolorbox}


\begin{document}

\imagenlogo{img/logoUBB.png} % Logo Universidad
\universidad{Universidad del Bío-Bío pepito5} % Universidad
\titulo{SQL en PostgreSQL} % Titulo
\asignatura{Base de Datos 1} % Asignatura
\autor{
    \indent
    Marcelo \textsc{Paz}
} % Autor
\portada % Crear portada
\margenes % Crear márgenes


\section{Teoria}
\indent
Pequeño apunte hecho en clase, y pasado en limpio a LaTeX sobre algunos comandos para PostgreSQL.

\subsection{Seleccionar todos los atributos de una tabla}
\indent
Para seleccionar todos los atributos de una tabla se utiliza el caracter `*` en lugar de los atributos.
\begin{tcolorbox}
    [colback=gray!5!white,colframe=gray!75!black,fonttitle=\bfseries,title=SQL]
    \begin{sqlcode}
    \end{sqlcode}
    \begin{verbatim}
SELECT *
FROM PROPIEDAD;
    \end{verbatim}
\end{tcolorbox}


\subsection{Poner alias y renombrar atributos}
\indent
Para seleccionar un atributo y renombrarlo se utiliza la palabra reservada `AS` seguida del nombre que se le quiere dar al atributo.
\begin{tcolorbox}
    [colback=gray!5!white,colframe=gray!75!black,fonttitle=\bfseries,title=SQL]
    \begin{sqlcode}
    \end{sqlcode}
    \begin{verbatim}
SELECT COD_P AS CODIGO_PROPIEDAD, UBICACION
FROM PROPIEDAD;
    \end{verbatim}
\end{tcolorbox}

\subsection{Poner alias a tablas}
\indent
Para ponerle un alias a una tabla solo se le agrega el nombre al lado.
\begin{tcolorbox}
    [colback=gray!5!white,colframe=gray!75!black,fonttitle=\bfseries,title=SQL]
    \begin{sqlcode}
    \end{sqlcode}
    \begin{verbatim}
SELECT P.COD_P AS CODIGO_PROPIEDAD, P.UBICACION
FROM PROPIEDAD P;
    \end{verbatim}
\end{tcolorbox}

\newpage
\subsection{Para filtrar}
\indent
Para filtrar se utiliza la palabra reservada `WHERE` seguida de la condición que se quiere aplicar `(AND, OR)`.
\begin{tcolorbox}
    [colback=gray!5!white,colframe=gray!75!black,fonttitle=\bfseries,title=SQL]
    \begin{sqlcode}
    \end{sqlcode}
    \begin{verbatim}
SELECT P.COD_P AS CODIGO_PROPIEDAD, P.UBICACION
FROM PROPIEDAD P
WHERE P.TIPOPROPIEDAD='CASA' AND P.NUMDORM>2;
    \end{verbatim}
\end{tcolorbox}

\newpage
\section{Instancias de la Base de Datos}
\begin{samepage}
    \insertarfigura{img/TablaArrendatario.png}{2.5cm}{Tabla de instancia de Arrendatarios}{fig:Figura}
    \rule{\linewidth}{0.2 mm}\\[-0.4 cm]
    \insertarfigura{img/TablaArriendo.png}{3cm}{Tabla de instancia de Arriendos}{fig:Figura}
    \rule{\linewidth}{0.2 mm}\\[-0.4 cm]
    \insertarfigura{img/TablaDueño.png}{2cm}{Tabla de instancia de Dueños}{fig:Figura}
    \rule{\linewidth}{0.2 mm}\\[-0.4 cm]
    \insertarfigura{img/TablaPropiedad.png}{1.7cm}{Tabla de instancia de Propiedades}{fig:Figura}
    \rule{\linewidth}{0.2 mm}\\[-0.4 cm]
    \insertarfigura{img/TablaTelefono.png}{2.5cm}{Tabla de instancia de Telefonos}{fig:Figura}
\end{samepage}

\newpage
\section{Ejercicios}
\indent
Resuelva las siguientes consultas:

\subsection{a}
\indent
Muestre todos los datos de las propiedades que son de tipo casa.
\begin{tcolorbox}
    [colback=gray!5!white,colframe=gray!75!black,fonttitle=\bfseries,title=SQL]
    \begin{sqlcode}
    \end{sqlcode}
    \begin{verbatim}
SELECT * 
FROM PROPIEDAD P
WHERE P.TIPOPROPIEDAD='CASA';
    \end{verbatim}
\end{tcolorbox}

\subsection{b}
\indent
Muestre el código y ubicación de las propiedades que tienes más de 100 mts2 construido.
\begin{tcolorbox}
    [colback=gray!5!white,colframe=gray!75!black,fonttitle=\bfseries,title=SQL]
    \begin{sqlcode}
    \end{sqlcode}
    \begin{verbatim}
SELECT P.COD_P, P.UBICACION 
FROM PROPIEDAD P
WHERE P.MTSCONST>100;
    \end{verbatim}
\end{tcolorbox}

\subsection{c}
\indent
Muestre el código y ubicación de las propiedades que tengas sea de Oriente y tenga sala de estar.
\begin{tcolorbox}
    [colback=gray!5!white,colframe=gray!75!black,fonttitle=\bfseries,title=SQL]
    \begin{sqlcode}
    \end{sqlcode}
    \begin{verbatim}
SELECT P.COD_P, P.UBICACION
FROM PROPIEDAD P
WHERE P.UBICACION='ORIENTE' AND P.SALAESTAR='SI';
    \end{verbatim}
\end{tcolorbox}

\newpage

\subsection{d}
\indent
Muestre el nombre de los dueños y código de cada propiedad.
\begin{tcolorbox}
    [colback=gray!5!white,colframe=gray!75!black,fonttitle=\bfseries,title=SQL]
    \begin{sqlcode}
    \end{sqlcode}
    \begin{verbatim}
SELECT D.NOMBRE, P.COD_P
FROM DUENO D, PROPIEDAD P
WHERE D.RUTD=P.RUTD;
    \end{verbatim}
\end{tcolorbox}

\subsection{e}
\indent
Liste el código de las propiedades que estuvieron arrendadas en el año 2020.
\begin{tcolorbox}
    [colback=gray!5!white,colframe=gray!75!black,fonttitle=\bfseries,title=SQL]
    \begin{sqlcode}
    \end{sqlcode}
    \begin{verbatim}
SELECT A.COD_P
FROM ARRIENDO A
WHERE (A.FECHAINICIO>='01/01/2020' AND A.FECHAINICIO<='31/12/2020') OR 
(A.FECHATERM >= '01/01/2020' AND A.FECHATERM <= '31/12/2020') OR 
(A.FECHAINICIO >= '01/01/2020' AND A.FECHATERM IS NULL);

o

SELECT A.COD_P
FROM ARRIENDO A
WHERE (A.FECHAINICIO BETWEEN '01/01/2020' AND '31/12/2020') OR
(A.FECHATERM BETWEEN '01/01/2020' AND '31/12/2020') OR
(A.FECHAINICIO <= '01/01/2020' AND A.FECHATERM IS NULL);
    \end{verbatim}
\end{tcolorbox}

\newpage
\subsection{f}
\indent
Muestra el código de las propiedades cuyo arriendo fluctua entre \$150000 y 200000 ambos inclusive.
\begin{tcolorbox}
    [colback=gray!5!white,colframe=gray!75!black,fonttitle=\bfseries,title=SQL]
    \begin{sqlcode}
    \end{sqlcode}
    \begin{verbatim}
SELECT A.COD_P
FROM ARRIENDO A
WHERE A.MONTO >= 150000 AND A.MONTO <= 200000;

o

SELECT A.COD_P
FROM ARRIENDO A
WHERE A.MONTO BETWEEN 150000 AND 200000;
    \end{verbatim}
\end{tcolorbox}

\subsection{g}
\indent
Muestre el nombre de los arrendatorios y de los dueños, y el código de la propiedad que tienen vigente un arriendo.
\begin{tcolorbox}
    [colback=gray!5!white,colframe=gray!75!black,fonttitle=\bfseries,title=SQL]
    \begin{sqlcode}
    \end{sqlcode}
    \begin{verbatim}
SELECT A.NOMBRE, D.NOMBRE, AR.COD_P 
FROM DUENO D, PROPIEDAD P, ARRIENDO AR, ARRENDATARIO A
WHERE D.RUTD = P.RUTD AND P.COD_P = AR.COD_P AND AR.RUTA = A.RUTA AND 
AR.FECHATERM IS NULL;
    \end{verbatim}
\end{tcolorbox}

\subsection{h}
\indent
Muestre el nombre y dirección de los arrendatarios que son de Talca
\begin{tcolorbox}
    [colback=gray!5!white,colframe=gray!75!black,fonttitle=\bfseries,title=SQL]
    \begin{sqlcode}
    \end{sqlcode}
    \begin{verbatim}
SELECT A.NOMBRE, A.DIRECCION
FROM ARRENDATARIO A
WHERE A.DIRECCION LIKE ('%TALCA%');
    \end{verbatim}
\end{tcolorbox}

\newpage
\subsection{i}
\indent
Liste el código de las propiedades que ha arrendado Pedro, desde el 2020 en adelante.
\begin{tcolorbox}
    [colback=gray!5!white,colframe=gray!75!black,fonttitle=\bfseries,title=SQL]
    \begin{sqlcode}
    \end{sqlcode}
    \begin{verbatim}
SELECT AR.COD_P
FROM ARRIENDO AR, ARRENDATARIO A
WHERE AR.RUTA = A.RUTA AND
A.NOMBRE = 'PEDRO' AND AR.FECHAINICIO >= '01/01/2020';
    \end{verbatim}
\end{tcolorbox}

\subsection{j}
\indent
Muestre los teléfonos que tiene cada propiedad.
\begin{tcolorbox}
    [colback=gray!5!white,colframe=gray!75!black,fonttitle=\bfseries,title=SQL]
    \begin{sqlcode}
    \end{sqlcode}
    \begin{verbatim}
SELECT T.NUMERO, T.COD_P
FROM PROPIEDAD P, TELEFONO T
WHERE P.COD_P = T.COD_P
    \end{verbatim}
\end{tcolorbox}

\subsection{k}
\indent
Muestre el nombre de las propiedades que no se encuentra con arriendo vigente.
\begin{tcolorbox}
    [colback=gray!5!white,colframe=gray!75!black,fonttitle=\bfseries,title=SQL]
    \begin{sqlcode}
    \end{sqlcode}
    \begin{verbatim}
SELECT P.COD_P
FROM PROPIEDAD P
WHERE P.COD_P NOT IN (SELECT A.COD_P FROM ARRIENDO A
					  WHERE A.COD_P = P.COD_P AND A.FECHATERM IS NULL);
    \end{verbatim}
\end{tcolorbox}

\subsection{l}
\indent
Muestra el nombre de los arrendatarios que no están arrendando actualmente.
\begin{tcolorbox}
    [colback=gray!5!white,colframe=gray!75!black,fonttitle=\bfseries,title=SQL]
    \begin{sqlcode}
    \end{sqlcode}
    \begin{verbatim}
SELECT A.NOMBRE
FROM ARRENDATARIO A
WHERE A.RUTA NOT IN (SELECT AR.RUTA FROM ARRIENDO AR
                    WHERE AR.FECHATERM IS NULL);
    \end{verbatim}
\end{tcolorbox}
\end{document}