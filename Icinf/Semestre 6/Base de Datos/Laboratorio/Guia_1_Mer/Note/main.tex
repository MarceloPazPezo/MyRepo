\documentclass{templateNote}
\usepackage{tcolorbox}


\begin{document}

\imagenlogo{img/logoUBB.png}
\universidad{Universidad del Bío-Bío pepito5}
\titulo{Guia 1 MER} % Titulo
\asignatura{Base de Datos} % asignatura
\autor{
    \indent
    Marcelo \textsc{Paz}
}   
\portada 
\margenes % Crear márgenes

\section{Teoria}
\subsection{MER}
\indent
Modelo de datos entidad-relación está basado en un percepción del mundo real que consta de un conjunto de objetos
básicos llamados entidades y de relaciones entre estos objetos.

\subsection{Entidad}
\indent
Un objeto del mundo real que puede distinguirse de otros objetos, pueden tener existencia física o conceptual.
objeto singular (persona, lugar, cosa, concepto, suceso) dentro de una empresa que hay que representar
en una BD. Una entidad es descrita por un conjunto de atributos. Ademas una entidad posee una clave primaria (*) para 
identificarla univocamente.

\insertarfigura{img/Entidad.png}{2.5cm}{Entidad Persona con algunos atributos}{fig:Figura}

\subsection{Atributo}
\indent
Los atributos describen propiedades que posee cada entidad.

\subsection{Relacion}
\indent
Asociación entre dos o más entidades. Una relación puede tener atributos propios (descriptivos).

\insertarfigura{img/Relacion.png}{2.5cm}{Relacion entre Persona y una Empresa}{fig:Figura}

\subsection{Restriccion}
\subsubsection{Uno a Muchos}
\indent
Una relación de uno a muchos (1:M) se refiere a la situación en la que un registro
de una tabla se puede asociar a uno o varios registros de otra tabla.

\insertarfigura{img/unoAMuchos.png}{3cm}{Relacion 1:M}{fig:Figura}

\subsubsection{Muchos a Muchos}
\indent
Una relación de muchos a muchos (M:N) se refiere a la situación en la que varios 
registros de una tabla pueden estar asociados con varios registros de otra tabla. 

\insertarfigura{img/muchosAMuchos.png}{3cm}{Relacion N:M}{fig:Figura}

\subsubsection{Uno a Uno}
\indent
Una relación de uno a uno (1:1) se refiere a la situación en la que un registro
de una tabla está asociado con un único registro de otra tabla.

\insertarfigura{img/unoAUno.png}{3cm}{Relacion 1:1}{fig:Figura}

\newpage
\subsection{Herencia}
\indent
Se refiere a la capacidad de una entidad de heredar atributos y relaciones de otra entidad. 

\insertarfigura{img/Herencia.png}{4.5cm}{Herencia}{fig:Figura}

\subsection{Dependencia}
\indent
Existe una relación “de dependencia en existencia” cuando se vincula un a entidad
regular con otra que no puede existir sin la primera. Se puede observar que
siempre la entidad regular “fuerte” tiene cardinalidad (1,1).

\section{Ejercicios}
\indent
Para cada uno de los siguientes enunciados se pide generar el Modelo Entidad Relación, respectivo,
considerando las entidades, relaciones, atributos, clave principal y cardinalidades entre las entidades.

\subsection{Enunciado 1}
\indent
Los huéspedes de un hotel se alojan en una habitación, la cual tiene una valor único por tipo de
habitación; Single, doble o suite.

\insertarfigura{img/Ejercicio1.png}{4cm}{MER 1}{fig:Figura}

\newpage
\subsection{Enunciado 2}
\indent
Cada académico participa, durante cada semestre, en distintos proyectos: de investigación,
extensión o capacitación. Donde para cada uno es los proyectos es necesario conocer el Jefe de
proyecto que es uno de los académicos que participa en el proyecto.

\insertarfigura{img/Ejercicio2.png}{12cm}{MER 2}{fig:Figura}

\newpage
\subsection{Enunciado 3}
\indent
Los alumnos memoristas de cada carrera preparan un proyecto durante su proyecto de título.
Cada proyecto de título es guiado por uno o varios profesores.

\insertarfigura{img/Ejercicio3.png}{8cm}{MER 3}{fig:Figura}

\subsection{Enunciado 4}
\indent
Todos los docentes que dictan clases en la Universidad son contratados por una o varias
facultades, como jornada completa, jornada parcial, media jornada. Los que son evaluados
semestralmente.

\insertarfigura{img/Ejercicio4.png}{8cm}{MER 4}{fig:Figura}

\newpage
\subsection{Enunciado 5}
Una corredora de propiedades desea manejar las propiedades residenciales que tiene en arriendo.
La información a manejar es la siguiente:
\begin{enumerate}
    \item Para cada propiedad, su ubicación, tipo (casa o departamento), número de dormitorios, número
    de baños, sala de estar (si o no), metros cuadrados construidos y metros cuadrados de patio para
    las casas, si cuenta con teléfono o no (y si tiene cuál es), además del dueño de la misma.
    \item Para los dueños, su rut, nombre, dirección y teléfono.
    \item Para los arrendatarios, su rut, nombre, dirección trabajo, teléfono trabajo.
    \item Para cada propiedad, si ésta se encuentra arrendada, manejar quién es el arrendatario, desde
    cuando, y a cuanto asciende el arriendo.
    \item Interesa saber quiénes han arrendado la casa anteriormente (no sólo el arrendatario actual).
    \item Interesa conocer cuánto tiempo ha estado arrendada una propiedad, y cuánto tiempo ha estado
    sin arrendatario.
\end{enumerate}

\insertarfigura{img/Ejercicio5.png}{10cm}{MER 5}{fig:Figura}

% \subsection{Enunciado 6}
% La biblioteca de la Universidad del Bío-Bío desea diseñar una base de datos para manejar la
% información de los préstamos. Los datos a registrar y requerimientos de información son los
% siguientes:
% \begin{enumerate}
%     \item Para cada alumno: el rut, nombre completo, carrera que estudia y régimen de estudio (diurno o
%     vespertino), teléfono y estado de usuario (si está o no habilitado para pedir libros).
%     \item Para cada libro: código del ejemplar, título, estado(prestado, disponible, en mantención, de
%     baja), el(los) autor(es) (nombre(s) y nacionalidad(es)), editorial (nombre, representante legal,
%     dirección).
%     \item A la biblioteca le interesa hacer consultas por títulos de libro, por autor, por editorial y
%     año de edición.
%     \item Para cada préstamo: código del alumno que solicita el libro, código del libro, fecha del
%     préstamo, fecha de devolución, estado del préstamo (vigente, devuelto en fecha, devuelto con
%     atraso, no devuelto).
%     \item A la bibliotecaria no le interesa registrar las devoluciones, sólo le interesa que al momento
%     en que se devuelve un ejemplar se anule (no borrar) el préstamo y en que caso de que se devuelve
%     en fecha posterior a la devolución se sancione al alumno de acuerdo a las políticas de devolución
%     que dicen que el alumno quedará inhabilitado para pedir libros 5 ías por cada día de atraso.
%     \item A la biblioteca le interesa emitir un listado de alumnos morosos por carrera.
% \end{enumerate}
    
% \insertarfigura{img/Ejercicio6.png}{4cm}{MER 6}{fig:Figura}

% \subsection{Enunciado 7}

% \begin{enumerate}
%     \item La oficina de trabajo recibe ofertas de empleo y cada vez que esto ocurre se abre un llamado a
%     estudiantes interesados.
%     \item A cada llamado se le asigna un número, una descripción, la fecha de
% aparicíon, la fecha límite de presentación al mismo y la institución que solicita el llamado.
%     \item Los llamados pueden ser solicitados por una empresa o por una facultad. 
%     \item Si el llamado es para una empresa se sabe el nombre de la misma y si desea figurar o no en el aviso que saldrá publicado. Cuando la oferta de empleo proviene de una facultad. 
%     \item Para postular a un llamado, el estudiante se debe registrar, donde deben ingresar su cédula, nombre, fecha de nacimiento, dirección, email,
% currúculum, teléfonos y la carrera que estudia.
%     \item Se considera una sola carrera por estudiante.
%     \item De cada estudiante inscrito al llamado se registra la fecha de inscripción al mismo.
%     \item Los currúculum de los estudiantes presentados se envían a la empresa o facultad que ofrece el empleo, para que  ́esta
% realice la selección.
%     \item En caso que la empresa decida no contratar a nadie el llamado se declara
% como desierto y se registra el motivo de tal situación para tenerlo en cuenta en futuros llamados.
%     \item También puede suceder que ningún estudiante se inscriba para un llamado, en cuyo caso el llamado
% también será declarado como desierto. De lo contrario se registran los estudiantes contratados
% en el mismo.
% \end{enumerate}
% \insertarfigura{img/Ejercicio7.png}{4cm}{MER 7}{fig:Figura}


\end{document}