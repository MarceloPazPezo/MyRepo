\documentclass{templateNote}
\usepackage{tcolorbox}
\usepackage{pgfplots}
\usepackage{pgf-pie}
\usepackage{tabularx}
\usepackage{graphicx}
\usepackage{tikz}

\newcolumntype{L}{>{\centering\arraybackslash}X}
\pgfmathdeclarefunction{gauss}{2}{%
  \pgfmathparse{1/(#2*sqrt(2*pi))*exp(-((x-#1)^2)/(2*#2^2))}%
}
\begin{document}

\imagenlogo{img/logoUBB.png}
\universidad{Universidad del Bío-Bío}
\titulo{Practico 3} % Titulo
\asignatura{Estadistica y Probabilidades} % Asignatura
\autor{
    \indent
    Marcelo \textsc{Paz}
}   
\portada
\margenes % Crear márgenes

% \section{Definiciones generales}
% \begin{enumerate}
%     \item \textbf{Estadística:} Ciencia que estudia los métodos para recoger, organizar, resumir y analizar datos, así como para sacar conclusiones válidas y tomar decisiones razonables basadas en tal análisis.\\
%     \item \textbf{Estadística Descriptiva:} Parte de la estadística que se ocupa de la recolección, presentación, descripción, análisis y resumen de datos.\\
% \end{enumerate}
% \section{Tipos de variables}
% \begin{enumerate}
%     \item \textbf{Variable cualitativa:} Es aquella que no se puede medir numéricamente, sino que se clasifica en categorías.\\
%     \item \textbf{Variable cuantitativa:} Es aquella que se puede medir numéricamente.\\
%     \begin{enumerate}
%         \item \textbf{Variable discreta:} Es aquella que puede tomar valores aislados, es decir, no puede tomar todos los valores posibles entre dos valores cualesquiera.\\
%         \item \textbf{Variable continua:} Es aquella que puede tomar todos los valores posibles entre dos valores cualesquiera.\\
%     \end{enumerate}
% \end{enumerate}

% \newpage
% \section{Graficos}
% \subsection{Tabla de frecuencias:}
% \indent
% Es una tabla que representa los datos de forma ordenada en columnas.\\
% \begin{figure}[H]
%     \centering
%     \begin{tabularx}{\textwidth}{|L|L|L|L|}
%         \hline
%         \textbf{Signos Visibles} & \textbf{Frecuencia Absoluta $f_i$} & \textbf{Frecuencia Acumulada $F_i$} & \textbf{Frecuencia relativa porcentual $fr_i$} \\
%         \hline
%         Dieta Severa & 9 & 9 & 33\% \\
%         \hline
%         Uso de Ropa Holgada & 6 & 15 & 22\% \\
%         \hline
%         Miedo a Engordar & 3 & 18 & 11\% \\
%         \hline
%         Hiperactividad & 4 & 22 & 15\% \\
%         \hline
%         Uso de Laxantes & 5 & 27 & 19\% \\
%         \hline
%         Total & 27 & & 100\% \\
%         \hline
%     \end{tabularx}
%     \caption{Tabla de frecuencias de datos \textbf{cualitativos}}
% \end{figure}

% \begin{figure}[H]
%     \centering
%     \begin{tabularx}{\textwidth}{|L|L|L|L|}
%         \hline
%         \textbf{Número de asignaturas reprobadas $c_i$} & \textbf{Frecuencia Absoluta $f_i$} & \textbf{Frecuencia Acumulada $F_i$} & \textbf{Frecuencia Relativa porcentual $fr_i$} \\
%         \hline
%         0 & 26 & 26 & 28,8\% \\
%         1 & 17 & 43 & 18,8\% \\
%         2 & 21 & 64 & 23,3\% \\
%         3 & 11 & 75 & 12,2\% \\
%         4 & 7 & 82 & 7,7\% \\
%         5 & 4 & 86 & 4,4\% \\
%         6 & 3 & 89 & 3,3\% \\
%         7 & 1 & 90 & 1,1\% \\
%         \hline
%         Total & 90 & & 100\% \\
%         \hline
%     \end{tabularx}
%     \caption{Tabla de frecuencias de datos \textbf{cuantitativos discretos}}
% \end{figure}

% \begin{figure}[H]
%     \centering
%     \begin{tabularx}{\textwidth}{|L|L|L|L|L|}
%         \hline
%         \textbf{Fronteras} & \textbf{Frecuencia Absoluta $f_i$} & \textbf{Frecuencia Acumulada $F_i$} & \textbf{Frecuencia relativa porcentual $fr_i$} & \textbf{Marca de clase $m_i$} \\
%         \hline
%         1,5 - 2,5 & 5 & 5 & 10\% & 2 \\
%         2,5 - 3,5 & 14 & 19 & 28\% & 3 \\
%         3,5 - 4,5 & 6 & 25 & 12\% & 4 \\
%         4,5 - 5,5 & 25 & 50 & 50\% & 5 \\
%         \hline
%         Total & 50 & & 100\% &  \\
%         \hline
%     \end{tabularx}
%     \caption{Tabla de frecuencias de datos \textbf{cuantitativos continuos}}
% \end{figure}

% \newpage
% \subsection{Histograma:}
% \indent
% Es un gráfico de barras. Se construye ubicando en el eje horizontal a las fronteras y en el
% vertical a las frecuencias absolutas. Su particularidad es que las barras están pegadas,
% pues comparten un lado en común. Su utilidad se aprecia cuando se quiere estudiar la
% forma de la distribución. Esto es, cuando se quiere estudiar la simetría o sesgo de los datos

% \begin{figure}[H]
%     \centering
%     \begin{tikzpicture}
%         \begin{axis}[
%             xtick distance=1,
%             ytick distance=2,
%             width = 17cm,
%             ymin = 0,
%             height = 8cm,
%             xlabel={Número de asignaturas reprobadas $c_i$},
%             ylabel={$f_i$}
%         ]
%         \addplot+[ycomb] plot coordinates
%         {(0,26) (1,17) (2,21) (3,11) (4,7) (5,4) (6,3) (7,1)};
%         \end{axis}
%     \end{tikzpicture}
%     \caption{Histograma de datos \textbf{cuantitativos discretos}}
% \end{figure}

% \begin{figure}[H]
%     \centering
%     \begin{tikzpicture}
%         \begin{axis}[
%             ybar interval,
%             xlabel = {Variable},
%             ylabel = \(f_i\),
%             xlabel style = {yshift=-1em},
%             width = 17cm,
%             xmin = 0,
%             ymin = 0,
%             height = 8cm,
%             xticklabel=
%             {$[\pgfmathprintnumber\tick,%
%                 \pgfmathprintnumber\nexttick)$}
%         ]
%             \addplot coordinates {
%                 (0, 1) (10, 5)
%                 (20, 7) (30, 2)
%                 (40, 1) (50, 1)
%                 (60, 1) (70, 1)
%                 (80, 0) (90, 1)
%             };
%         \end{axis}
%     \end{tikzpicture}
%     \caption{Histograma de datos \textbf{cuantitativos continuos}}
% \end{figure}

% \newpage
% \subsection{Polígono de frecuencia:}
% \indent
% Es un gráfico que consiste en destacar las marcas de clase de cada intervalo frente a sus
% correspondientes frecuencias absolutas. Cada marca de clase se une con segmentos de
% rectas que generan la curva. Para cerrar el área se utilizan las marcas de clase ficticias: La
% primera se crea restando la amplitud a la marca de clase del primer intervalo y la segunda
% se crea sumando la amplitud a la marca de clase del último intervalo. Su utilidad se aprecia
% cuando se quiere comprar dos o más distribuciones de frecuencias. Generalmente se dibuja
% sobre el Histograma.
% \begin{figure}[H]
%     \centering
%     \begin{tikzpicture}
%         \begin{axis}[
%             xtick distance=1,
%             ytick distance=2,
%             ymajorgrids=true,
%             grid style=dashed,
%             width = 17cm,
%             xmin = 0,
%             ymin = 0,
%             height = 8cm,
%             xlabel={Número de asignaturas reprobadas $c_i$},
%             ylabel={$f_i$}
%         ]
%         \addplot+[ycomb] plot coordinates
%         {(0,26) (1,17) (2,21) (3,11) (4,7) (5,4) (6,3) (7,1)};
%         \addplot[color=red,mark=*] coordinates {(0,26) (1,17) (2,21) (3,11) (4,7) (5,4) (6,3) (7,1) (8,0)};
%         \end{axis}
%     \end{tikzpicture}
%     \caption{Histograma y Poligono de frecuencias de datos \textbf{cuantitativos discretos}}
% \end{figure}

% \begin{figure}[H]
%     \centering
%     \begin{tikzpicture}
%         \begin{axis}[
%             xlabel = {Variable},
%             ylabel = \(f_i\),
%             xlabel style = {yshift=-1em},
%             xmin = 0,
%             xtick distance=5,
%             ymin = 0,
%             ymajorgrids=true,
%             grid style=dashed,
%             width = 17cm,
%             height = 8cm,
%         ]
%             \draw [blue, fill= blue!10] (axis cs:0,0) rectangle (axis cs:10,1);
%             \draw [blue, fill= blue!10] (axis cs:10,0) rectangle (axis cs:20,5);
%             \draw [blue, fill= blue!10] (axis cs:20,0) rectangle (axis cs:30,7);
%             \draw [blue, fill= blue!10] (axis cs:30,0) rectangle (axis cs:40,2);
%             \draw [blue, fill= blue!10] (axis cs:40,0) rectangle (axis cs:50,1);
%             \draw [blue, fill= blue!10] (axis cs:50,0) rectangle (axis cs:60,1);
%             \draw [blue, fill= blue!10] (axis cs:60,0) rectangle (axis cs:70,1);
%             \draw [blue, fill= blue!10] (axis cs:70,0) rectangle (axis cs:80,1);
%             \draw [blue, fill= blue!10] (axis cs:80,0) rectangle (axis cs:90,0);
%             \addplot[color=red,mark=*, line width=1pt] coordinates {
%                 (0,0)
%                 (5,1)
%                 (15,5)
%                 (25,7)
%                 (35,2)
%                 (45,1)
%                 (55,1)
%                 (65,1)
%                 (75,1)
%                 (85,0)
%             };
%         \end{axis}
%     \end{tikzpicture}
%     \caption{Histograma y Poligono de frecuencia de datos \textbf{cuantitativos continuos}}
% \end{figure}

% \newpage
% \subsection{Ojiva:}
% \indent
% Es un gráfico de frecuencia acumulada. Se gráfica única y exclusivamente con las fronteras
% en el eje horizontal. Su utilidad se aprecia cuando se cuenta con medidas de posición tales
% como: Cuartiles, Deciles y Percentiles.
% \begin{figure}[H]
%     \centering
%     \begin{tikzpicture}
%         \begin{axis}[
%             xtick distance=1,
%             ytick distance=5,
%             width = 10cm,
%             xmin = 0,
%             ymin = 0,
%             height = 8cm,
%             xlabel={Número de asignaturas reprobadas $c_i$},
%             ylabel={$F_i$}
%         ]
%         \addplot[color=red,mark=*] coordinates { (0,26) (1,43) (2,64) (3,75) (4,82) (5,86) (6,89) (7,90)};
%         \end{axis}
%     \end{tikzpicture}
%     \caption{Ojiva de datos \textbf{cuantitativos discretos}}
% \end{figure}

% \begin{figure}[H]
%     \centering
%     \begin{tikzpicture}
%         \begin{axis}[
%             xlabel = {Variable},
%             ylabel = \(F_i\),
%             xlabel style = {yshift=-1em},
%             xmin = 0,
%             xtick distance=5,
%             ymin = 0,
%             width = 10cm,
%             height = 8cm,
%         ]
%             \addplot[color=red,mark=*, line width=1pt] coordinates {
%                 (0,0)
%                 (5,1)
%                 (15,6)
%                 (25,13)
%                 (35,15)
%                 (45,16)
%                 (55,17)
%                 (65,18)
%                 (75,19)
%             };
%         \end{axis}
%     \end{tikzpicture}
%     \caption{Ojiva de datos \textbf{cuantitativos continuos}}
% \end{figure}

% \newpage
% \section{Medidas de tendencia central}
% \subsection{Media aritmética:}
% Es la suma de todos los datos dividida por el total. Cambia un poco en su forma según como
% estén presentados los datos.
% \begin{enumerate}
%     \item \textbf{Datos no agrupados:}
%     \begin{equation*}
%         \overline{x} = \frac{\displaystyle \sum_{i=1}^{n} x_i}{n}
%     \end{equation*}
%     * $x_i$ es cada dato.
%     \item \textbf{Datos agrupados(discreta):}
%     \begin{equation*}
%         \overline{x} = \frac{\displaystyle \sum_{i=1}^{k} f_i \cdot c_i}{n}
%     \end{equation*}
%     * $c_i$ es cada dato.
%     \item \textbf{Datos agrupados(continua):}
%     \begin{equation*}
%         \overline{x} = \frac{\displaystyle \sum_{i=1}^{k} f_i \cdot m_i}{n}
%     \end{equation*}
%     * $m_i$ es la marca de clase de cada dato.
% \end{enumerate}
% * $f_i$ es la frecuencia absoluta de cada dato.\\
% * $n$ es el total de datos.\\
% * $k$ es el total de intervalos.

% \subsection{Mediana:}
% Es el valor que ocupa la posición central en un conjunto de datos ordenados. Cambia un
% poco en su forma según como estén presentados los datos.
% \begin{enumerate}
%     \item \textbf{Datos no agrupados:} es necesario ordenar de menor a mayor los datos.
%     \begin{equation*}
%         M_e = \displaystyle \left\{ \begin{array}{ll}
%             x_{\frac{n+1}{2}} & \textrm{si $n$ es impar}\\
%             \frac{\displaystyle x_{\frac{n}{2}} + x_{\frac{n}{2}+1}}{2} & \textrm{si $n$ es par}
%         \end{array} \right.
%     \end{equation*}
%     \item \textbf{Datos agrupados(discreta):}
%     $M_e$ Es el valor de la clase o categoria ($c_i$) donde se encuentra la mitad de los datos en la columna de la frecuencia acumulada.
%     \newpage
%     \item \textbf{Datos agrupados(continua):}
%     \begin{equation*}
%         M_e = FI_k + \left(\frac{\displaystyle \frac{n}{2} - F_{k-1}}{f_k}\right) \cdot A_k
%     \end{equation*}
% \end{enumerate}
% * $FI_k$ es la Frontera Inferior de la clase mediana.\\
% * $FI_k$ es la frecuencia absoluta acumulada de la clase anterior a la clase mediana.\\
% * $f_k$ es la frecuencia absoluta de la clase.\\
% * $A_k$ es la amplitud de la clase mediana.\\
% * $n$ es el total de datos.\\
% \subsection{Moda:}
% Es el valor que más se repite en un conjunto de datos. Cambia un poco en su forma según
% como estén presentados los datos.
% \begin{enumerate}
%     \item \textbf{Datos no agrupados:} es necesario ordenar de menor a mayor los datos.
%     \begin{equation*}
%         M_o = \textrm{Valor que más se repite}
%     \end{equation*}
%     \item \textbf{Datos agrupados(discreta):}
%     $M_o$ Es el valor de la clase o categoria ($c_i$) donde se encuentra la mayor frecuencia absoluta.
%     \item \textbf{Datos agrupados(continua):}
%     \begin{equation*}
%         M_o = FI_k + \left(\frac{a}{a + b}\right) \cdot A_k
%     \end{equation*}
% \end{enumerate}
% * $FI_k$ es la Frontera Inferior de la clase modal.\\
% * $a$ es la Diferencia entre la frecuencia absoluta de la clase modal y la frecuencia absoluta de la clase anterior a la clase modal.\\
% * $b$ es la Diferencia entre la frecuencia absoluta de la clase modal y la frecuencia absoluta de la clase posterior a la clase modal.\\
% * $A_k$ es la amplitud de la clase modal.\\

% \subsection{Resumen}
% \begin{figure}[H]
%     \centering
%     \begin{tabularx}{\textwidth}{|L|L|L|L|}
%         \hline
%         \textbf{Descripción} & \textbf{Media $\overline{x}$} & \textbf{Mediana $M_e$} & \textbf{Moda $M_o$} \\
%         \hline
%         \textbf{Datos NO agrupados} & 
%         \[
%             \overline{x} = \frac{\displaystyle \sum_{i=1}^{n} x_i}{n}
%         \] & Si el total de observaciones es impar, la mediana es el valor que se encuentra justo en la mitad del conjunto previamente ordenado de menor a mayor. Si el total de observaciones es par, la mediana es el promedio de las dos observaciones centrales del conjunto previamente ordenado. &
%         Valor que más se repite. \textbf{Unimodal:} una moda. \textbf{Bimodal:} dos modas. \textbf{Multimodal:} mas de 2 modas\\
%         \hline
%         \textbf{Datos agrupados (discreta)} & 
%         \[
%             \overline{x} = \frac{\displaystyle \sum_{i=1}^{n} f_i c_i}{n}
%         \] & Es el valor de la clase o categoria ($c_i$) donde se encuentra la mitad de los datos en la columna de la frecuencia acumulada. &
%         Es el valor de la clase o categoria ($c_i$) donde se encuentra la frecuencia absoluta mas alta.\\
%         \hline
%         \textbf{Datos agrupados (continua)} & 
%         \[
%             \overline{x} = \frac{\displaystyle \sum_{i=1}^{n} f_i m_i}{n}
%         \] &
%         \begin{equation*}
%             \scalebox{0.6}{$M_e = FI_k + \left(\frac{\displaystyle \frac{n}{2} - F_{k-1}}{f_k}\right) \cdot A_k$}   
%         \end{equation*}
%         &
%         \begin{equation*}
%             \scalebox{0.8}{$M_o = FI_k + \left(\frac{a}{a+b}\right) \cdot A_k$}
%         \end{equation*} \\
%         \hline
%     \end{tabularx}
%     \caption{Tabla Resumen \textbf{Medidas de Tendencia Central}}
% \end{figure}

% \newpage
% \section{Medidas de dispersión}
% \indent
% Nos indican que tanto se alejan los datos del centro. Las más utilizadas en Estadística
% Descriptiva son:
% \subsection{Varianza}
% \indent
% Es el promedio cuadrado de las distancias
% entre cada observación y el promedio de ellos.
% Se denota por $S^2$
% . Su gran desventaja es que
% crece conforme crecen los datos y también
% puede ser cero si estos, son muy parecidos
% entre si.

% Se utiliza la media $\overline{x}$
% \begin{enumerate}
%     \item \textbf{Datos no agrupados:}
%     \begin{equation*}
%         s^2 = \frac{\displaystyle \sum_{i=1}^{n} (x_i - \overline{x})^2}{n}
%     \end{equation*}
%     * $x_i$ es cada dato.\\
%     \item \textbf{Datos agrupados (discreta):}
%     \begin{equation*}
%         s^2 = \frac{\displaystyle \sum_{i=1}^{n} (c_i - \overline{x})^2 f_i}{n}
%     \end{equation*}
%     * $c_i$ es cada dato.\\
%     \item \textbf{Datos agrupados (continua):}
%     \begin{equation*}
%         s^2 = \frac{\displaystyle \sum_{i=1}^{n} (m_i - \overline{x})^2 f_i}{n}
%     \end{equation*}
%     * $m_i$ es la marca de clase de cada dato.
% \end{enumerate}
% * $f_i$ es la frecuencia absoluta de cada dato.\\
% * $n$ es el total de datos.\\

% \subsection{Desviación Estandar}
% \indent
% Es la raíz cuadrada de la varianza. Su gran
% ventaja por sobre ésta, es que entrega sus
% resultados en la misma unidad de medida que
% la variable.
% \begin{equation*}
%     s = \sqrt{s^2}
% \end{equation*}
% * $s^2$ es la varianza.\\

% \subsection{Coeficiente de Variación}
% \indent
% Se define como el cuociente entre la
% desviación estándar y el promedio de los datos.
% Generalmente se entrega en porcentaje. Su
% gran ventaja es que sus resultados carecen de
% unidad de medida, por lo que permite comparar
% datos aunque estén en distinta unidades de
% medida.
% \begin{equation*}
%     CV = \frac{s}{\overline{x}} \cdot 100
% \end{equation*}
% * $s$ es la desviación estándar.\\
% * $\overline{x}$ es la media.\\

% \subsection{Rango}
% \indent
% Es la diferencia entre el máximo y el mínimo de
% los datos. Su utilidad se aprecia cuando
% tenemos más información de la variable. 
% \begin{equation*}
%     R = x_{max} - x_{min}
% \end{equation*}
% * $x_{max}$ es el valor máximo de los datos.\\
% * $x_{min}$ es el valor mínimo de los datos.\\

% \section{Medidas de posición}
% \indent
% Son aquellas que permiten conocer con mayor detalle a una variable. Entre se encuentran
% los:
% \subsection{Cuartiles}
% Dividen al conjunto de datos en cuatro partes porcentualmente iguales.
% \[
%     Q_k \quad \text{, k = 1,2,3} \quad
%     \text{, donde: } Q_1 = 25\% \quad Q_2 = 50\% \quad Q_3 = 75\%
% \]
% \begin{enumerate}
%     \item Para datos cuantitativos discretos:
%     \begin{equation*}
%         Q_k = \frac{kn}{4}
%     \end{equation*}
%     \item Para datos cuantitativos continuos:
%     \begin{equation*}
%         Q_k = FI_k + \left(\frac{\displaystyle \frac{kn}{4} - F_{k-1}}{f_k}\right) \cdot A_k
%     \end{equation*}
%     * $FI_k$ es la Frontera Inferior de la clase del k-esimo cuartil.\\
%     * $F_{k-1}$ es la frecuencia acumulada hasta la clase anterior a la clase del k-esimo cuartil. \\
%     * $f_k$ es la frecuencia absoluta de la clase del k-esimo cuartil.\\
%     * $A_k$ es la amplitud de la clase del k-esimo cuartil.\\
% \end{enumerate}

% \newpage
% \subsection{Deciles}
% Dividen al conjunto de datos en diez partes porcentualmente iguales.
% \[
%     D_k \quad \text{, k = 1,2,...,8,9} \quad
%     \text{, donde: } D_1 = 10\% \quad ...\quad D_9 = 90\%
% \]
% \begin{enumerate}
%     \item Para datos cuantitativos discretos:
%     \begin{equation*}
%         D_k = \frac{kn}{10}
%     \end{equation*}
%     \item Para datos cuantitativos continuos:
%     \begin{equation*}
%         D_k = FI_k + \left(\frac{\displaystyle \frac{kn}{10} - F_{k-1}}{f_k}\right) \cdot A_k
%     \end{equation*}
%     * $FI_k$ es la Frontera Inferior de la clase del k-esimo decil.\\
%     * $F_{k-1}$ es la frecuencia acumulada hasta la clase anterior a la clase del k-esimo decil. \\
%     * $f_k$ es la frecuencia absoluta de la clase del k-esimo decil.\\
%     * $A_k$ es la amplitud de la clase del k-esimo decil.\\
% \end{enumerate}

% \subsection{Percentiles}
% Dividen al conjunto de datos en cien partes porcentualmente iguales.
% \[
%     P_k \quad \text{, k = 1,2,...,98,99} \quad
%     \text{, donde: } P_1 = 1\% \quad ...\quad P_{99} = 99\%
% \]
% \begin{enumerate}
%     \item Para datos cuantitativos discretos:
%     \begin{equation*}
%         P_k = \frac{kn}{100}
%     \end{equation*}
%     \item Para datos cuantitativos continuos:
%     \begin{equation*}
%         P_k = FI_k + \left(\frac{\displaystyle \frac{kn}{100} - F_{k-1}}{f_k}\right) \cdot A_k
%     \end{equation*}
%     * $FI_k$ es la Frontera Inferior de la clase del k-esimo percentil.\\
%     * $F_{k-1}$ es la frecuencia acumulada hasta la clase anterior a la clase del k-esimo percentil. \\
%     * $f_k$ es la frecuencia absoluta de la clase del k-esimo percentil.\\
%     * $A_k$ es la amplitud de la clase del k-esimo percentil.\\
% \end{enumerate}

% \newpage
% \section{Simetría}
% \begin{equation*}
%     f_1 = f_k \quad f_2 = f_{k-1} \quad f_3 = f_{k-2} \quad \text{... etc}
% \end{equation*}

% \subsection{Unimodal}
% \begin{equation*}
%     \overline{x} = M_e = M_o
% \end{equation*}

% \begin{figure}[H]
%     \centering
%     \begin{tikzpicture}
%         \begin{axis}[
%             ybar interval,
%             xlabel = {Variable},
%             ylabel = \(f_i\),
%             xlabel style = {yshift=-1em},
%             width = 17cm,
%             ymin = 0,
%             xmin = 0,
%             height = 8cm,
%             xticklabel=
%             {$[\pgfmathprintnumber\tick,%
%                 \pgfmathprintnumber\nexttick)$}
%         ]
%             \addplot coordinates {
%                 (0, 1) (10, 2)
%                 (20, 4) (30, 5)
%                 (40, 4) (50, 2)
%                 (60, 1) (70, 1)
%             };
%         \end{axis}
%     \end{tikzpicture}
%     \caption{Distribución \textbf{simétrica y unimodal}}
% \end{figure}

% \subsection{Bimodal}

% \begin{figure}[H]
%     \centering
%     \begin{tikzpicture}
%         \begin{axis}[
%             ybar interval,
%             xlabel = {Variable},
%             ylabel = \(f_i\),
%             xlabel style = {yshift=-1em},
%             width = 17cm,
%             ymin = 0,
%             xmin = 0,
%             height = 8cm,
%             xticklabel=
%             {$[\pgfmathprintnumber\tick,%
%                 \pgfmathprintnumber\nexttick)$}
%         ]
%             \addplot coordinates {
%                 (0, 1) (10, 2)
%                 (20, 4) (30, 4)
%                 (40, 2) (50, 1)
%                 (60, 1)
%             };
%         \end{axis}
%     \end{tikzpicture}
%     \caption{Distribución \textbf{simétrica y bimodal en el centro}}
% \end{figure}
% \begin{figure}[H]
%     \centering
%     \begin{tikzpicture}
%         \begin{axis}[
%             ybar interval,
%             xlabel = {Variable},
%             ylabel = \(f_i\),
%             xlabel style = {yshift=-1em},
%             width = 17cm,
%             ymin = 0,
%             xmin = 0,
%             height = 8cm,
%             xticklabel=
%             {$[\pgfmathprintnumber\tick,%
%                 \pgfmathprintnumber\nexttick)$}
%         ]
%             \addplot coordinates {
%                 (0, 5) (10, 3)
%                 (20, 2) (30, 1)
%                 (40, 2) (50, 3)
%                 (60, 5) (70, 5)
%             };
%         \end{axis}
%     \end{tikzpicture}
%     \caption{Distribución \textbf{simétrica y bimodal en los extremos}}
% \end{figure}

% \section{Sesgo (Asimétria)}
% \indent
% El sesgo es un comportamiento que se da en las medidas de tendencia central y es de la
% siguiente forma:
% \subsection{Positivo o a la derecha}
% \begin{equation*}
%     \overline{x} > M_e > M_o
% \end{equation*}
% \begin{figure}[H]
%     \centering
%     \begin{tikzpicture}
%         \begin{axis}[
%             ybar interval,
%             xlabel = {Variable},
%             ylabel = \(f_i\),
%             xlabel style = {yshift=-1em},
%             width = 17cm,
%             ymin = 0,
%             xmin = 0,
%             height = 8cm,
%             xticklabel=
%             {$[\pgfmathprintnumber\tick,%
%                 \pgfmathprintnumber\nexttick)$}
%         ]
%             \addplot coordinates {
%                 (0, 5) (10, 14)
%                 (20, 7) (30, 3)
%                 (40, 2) (50, 1)
%                 (60, 1)
%             };
%         \end{axis}
%     \end{tikzpicture}
%     \caption{Distribución \textbf{asimétrica positiva}}
% \end{figure}

% \subsection{Negativo o a la izquierda}
% \begin{equation*}
%     \overline{x} < M_e < M_o
% \end{equation*}
% \begin{figure}[H]
%     \centering
%     \begin{tikzpicture}
%         \begin{axis}[
%             ybar interval,
%             xlabel = {Variable},
%             ylabel = \(f_i\),
%             xlabel style = {yshift=-1em},
%             width = 17cm,
%             ymin = 0,
%             xmin = 0,
%             height = 8cm,
%             xticklabel=
%             {$[\pgfmathprintnumber\tick,%
%                 \pgfmathprintnumber\nexttick)$}
%         ]
%             \addplot coordinates {
%                 (0, 1) (10, 2)
%                 (20, 3) (30, 7)
%                 (40, 12) (50, 5)
%                 (60, 5)
%             };
%         \end{axis}
%     \end{tikzpicture}
%     \caption{Distribución \textbf{asimétrica positiva}}
% \end{figure}

% \subsection{Coeficientes de asimétria}
% \begin{equation*}
%     CA_1 = \frac{\overline{x} - M_o}{s} \\
%     CA_2 = \frac{3(\overline{x} - M_e)}{s} \\
%     CA_3 = \frac{Q_1 - 2Q_2 + Q_3}{Q_3 - Q_1} \\
% \end{equation*}
% * donde $Q_3 - Q_1$ se conoce como rango intercuartílico. \\
% * $CA_1 \ne CA_2 \ne CA_3$, pero coiciden en signo.\\
% \begin{enumerate}
%     \item Si el \textbf{coeficiente es positivo}: se dice que la asimetría es positiva y el sesgo va a la derecha.
%     \item Si el \textbf{coeficiente es negativo}: se dice que la asimetría es negativo y el sesgo va a la izquierda.
% \end{enumerate}

% \section{Regla Empírica}
% \indent
% Si una distribución es simétrica, unimodal de forma acampanada, se dice \textbf{Aproximadamente Normal}

% \begin{figure}[H]
%     \centering
%     \begin{tikzpicture}
%         \begin{axis}[
%             no markers, 
%             domain=-4:4, 
%             samples=100,
%             xtick={-3,-2,-1,0,1,2,3},
%             xticklabels={$\overline{x} - 3s$, $\overline{x} - 2s$, $\overline{x} - s$, $\overline{x}$, $\overline{x} + s$, $\overline{x} + 2s$, $\overline{x} + 3s$},
%             ymin=0,
%             ymax=0.5,
%             axis lines*=left, 
%             xlabel=$x$,
%             ylabel=$y$,
%             height=8cm, 
%             width=16cm,
%             enlargelimits=false, 
%             clip=false, 
%             axis on top,
%             grid = major,
%             axis lines = middle
%         ]
%         \draw [red, fill=red!10] (axis cs:-1,-0.08) rectangle (axis cs:1,-0.08);
%         \draw [red, fill=red!10] (axis cs:-1,-0.08) rectangle (axis cs:-1,-0.05);
%         \draw [red, fill=red!10] (axis cs:1,-0.08) rectangle (axis cs:1,-0.05);
%         \draw [green, fill=green!10] (axis cs:-2,-0.13) rectangle (axis cs:2,-0.13);
%         \draw [green, fill=green!10] (axis cs:-2,-0.13) rectangle (axis cs:-2,-0.05);
%         \draw [green, fill=green!10] (axis cs:2,-0.13) rectangle (axis cs:2,-0.05);
%         \draw [blue, fill=blue!10] (axis cs:-3,-0.18) rectangle (axis cs:3,-0.18);
%         \draw [blue, fill=blue!10] (axis cs:-3,-0.18) rectangle (axis cs:-3,-0.05);
%         \draw [blue, fill=blue!10] (axis cs:3,-0.18) rectangle (axis cs:3,-0.05);
%         \addplot [very thick,cyan!50!black] {gauss(0,1)};
%         \node[draw, fill=white] at (axis cs:0,-0.03) {$ \overline{x} $};
%         \node[draw, fill=white] at (axis cs:0,-0.08) {$ 68\% $};
%         \node[draw, fill=white] at (axis cs:0,-0.13) {$ 95\% $};
%         \node[draw, fill=white] at (axis cs:0,-0.18) {$ 99\% $};
%         \end{axis}
%     \end{tikzpicture}
%     \caption{Regla Empírica}
% \end{figure}

% \newpage
% \section{Teorema de Chebyshev}
% \indent
% Si la distribución es asimétrica o tiene algún tipo de sesgo:

% Dado un numero $k>=1$ y un conjunto de n mediciones $x_1, x_2, ..., x_n$, por lo menos $\left(1-\frac{1}{K^2}\right) \%$ de las mediciones estara en $\left(\overline{x} - Ks, \overline{x} + Ks\right)$
% \begin{equation*}
%     \begin{array}{ll}
%         \textrm{Si k=1} & \displaystyle \left( 1 - \frac{1}{K^2}\right)\% = 0\% \\
%         \textrm{Si k=2} & \displaystyle \left( 1 - \frac{1}{K^2}\right)\% = 75\% \\
%         \textrm{Si k=2,6} & \displaystyle \left( 1 - \frac{1}{K^2}\right)\% = 85,21\% \\
%         \textrm{Si k=3} & \displaystyle \left( 1 - \frac{1}{K^2}\right)\% = 88,9\% \\
%     \end{array}
% \end{equation*}

\section{Ejercicio}
\indent
A continuación, se muestran los tiempos invertidos en cierta actividad por hombres y mujeres de una empresa.

\begin{figure}[H]
    \centering
    \begin{minipage}{.45\textwidth}
        \centering
        \begin{tabular}{|c|c|c|c|}
            \hline
            5 & 2,8 & 2,4 & 6,8 \\
            6,2	& 5,7 & 5,9 & 3,7 \\
            3,8 & 3,3 & 2,2 & 2,7 \\
            5,1 & 5,5 & 4,3 & 4,3 \\
            3,7	& 3,2 & 3,0	& 7,0 \\
            2,7 & 3,2 & 3,4 & 4,2 \\
            4,3 & 4,8 & 4,9 & 5,3 \\
            5,7 & 4,0 & 4,0 & 4,1 \\
            4,5 & 5,0 & 4,6 & 5,3 \\
            5,3 & 5,6 & 6,0 & 5,3 \\
            4,5 & 4,1 & 4,8 & 4,9 \\
            \hline 
        \end{tabular}
        \caption{Datos hombres}
    \end{minipage}\hfill
    \begin{minipage}{.45\textwidth}
        \centering
        \begin{tabular}{|c|c|c|c|}
            \hline
            6,7&5,6&6,0&5,3\\
            5,1&5,8&5,3&2,5\\
            5,8&4,1&4,1&2,8\\
            4,5&6,6&4,3&2,6\\
            6,2&3,3&3,0&5,0\\
            2,1&4,8&5,4&3,5\\
            5,2&3,8&6,3&4,0\\
            6,3&4,4&2,4&4,5\\
            5,2&3,8&6,3&4,0\\
            4,6&6,4&4,7&5,0\\
            5,5&3,8&3,1&3,2\\
            \hline 
        \end{tabular}
        \caption{Datos mujeres}
    \end{minipage}
\end{figure}

\textbf{a)} Construya la tabla de distribución de frecuencias para cada conjunto de datos. 
* Para los hombres:\\
\begin{itemize}
    \item Rango: Xmax - Xmin $ = 7 - 2,2 = 4,8$
    
    \item Número de intervalos: $k = 1 + 3,3 \cdot log_{10}(44) = 1 + 3,3 \cdot 1,6435 = 6,42 \approx 7$
    
    \item Amplitud: $A = \frac{4,8}{7} = 0,6857 \approx 0,7$
\end{itemize}

\begin{figure}[H]
    \centering
    \begin{tabularx}{\textwidth}{|L|L|L|L|L|L|L|}
        \hline
        \textbf{Limite Aparente} & \textbf{Fronteras}& \textbf{$f_i$} & \textbf{$F_i$} & \textbf{$fr_i$} & \textbf{$Fr_i$} & \textbf{$m_i$} \\
        \hline
        $2,2;2,8$ & $2,15;2,85$ & 5 & 5 & 11,36\% & 11,36\% & 2,5 \\
        \hline
        $2,9;3,5$ & $2,85;3,55$ & 6 & 11 & 13,64\% & 25\% & 3,2 \\
        \hline
        $3,6;4,2$ & $3,55;4,25$ & 7 & 18 & 15,91\% & 40,91\% & 3,9 \\
        \hline
        $4,3;4,9$ & $4,25;4,95$ & 10 & 28 & 22,73\% & 63,64\% & 4,6 \\
        \hline
        $5,0;5,6$ & $4,95;5,65$ & 9 & 37 & 20,45\% & 84,09\% & 5,3 \\
        \hline
        $5,7;6,3$ & $5,65;6,35$ & 5 & 42 & 11,36\% & 95,45\% & 6 \\
        \hline
        $6,4;7,0$ & $6,35;7,05$ & 2 & 44 & 4,55\% & 100\% & 6,7 \\
        \hline
        \textbf{Total} & 44 &  & 100\% &  \\
        \hline
    \end{tabularx}
    \caption{Tabla de distribución de frecuencias para hombres}
\end{figure}

\newpage
* Para los hombres:\\
\begin{itemize}
    \item Rango: Xmax - Xmin $ = 6,7 - 2,1 = 4,6$
    
    \item Número de intervalos: $k = 1 + 3,3 \cdot log_{10}(44) = 1 + 3,3 \cdot 1,6435 = 6,42 \approx 7$
    
    \item Amplitud: $A = \frac{4,6}{7} = 0,6571 \approx 0,7$
\end{itemize}
\begin{figure}[H]
    \centering
    \begin{tabularx}{\textwidth}{|L|L|L|L|L|L|L|}
        \hline
        \textbf{Limite Aparente} & \textbf{Fronteras}& \textbf{$f_i$} & \textbf{$F_i$} & \textbf{$fr_i$} & \textbf{$Fr_i$} & \textbf{$m_i$} \\
        \hline
        $2,1;2,8$ & $2,05;2,85$ & 5 & 5 & 11,36\% & 11,36\% & 2,5 \\
        \hline
        $2,9;3,6$ & $2,85;3,55$ & 5 & 10 & 11,36\% & 22,73\% & 3,2 \\
        \hline
        $3,7;4,4$ & $3,65;4,45$ & 10 & 20 & 22,73\% & 45,45\% & 3,9 \\
        \hline
        $4,5;5,2$ & $4,45;5,25$ & 9 & 29 & 20,45\% & 65,91\% & 4,6 \\
        \hline
        $5,3;6,0$ & $5,25;6,05$ & 8 & 37 & 18,18\% & 84,09\% & 5,3 \\
        \hline
        $6,1;6,8$ & $6,05;6,85$ & 7 & 44 & 15,91\% & 100\% & 6 \\
        \hline
        \textbf{Total} & 44 &  & 100\% &  \\
        \hline
    \end{tabularx}
    \caption{Tabla de distribución de frecuencias para mujeres}
\end{figure}

\textbf{b)} Calcule las medidas de tendencia central para los datos no agrupados en cada grupo. Interprete cada medida descriptiva.

\textbf{c)} Repita “b” pero aplique fórmula de datos agrupados.

\begin{itemize}
    \item \textbf{Media:}
    
    \textbf{Datos agrupados(continua):}
    \begin{equation*}
        \overline{x} = \frac{\displaystyle \sum_{i=1}^{n} f_i m_i}{n} = \frac{5 \times 2,5 + 6 \times 3,2 + 7 \times 3,9 + 10 \times 4,6 + 9 \times 5,3 + 5 \times 6 + 2 \times 6,7}{44} = 4,5
    \end{equation*}

    \item \textbf{Mediana:}
    
    \textbf{Datos agrupados(continua):}
    \begin{equation*}
        M_e = FI_k + \left(\frac{\displaystyle \frac{n}{2} - F_{k-1}}{f_k}\right) \cdot A_k = 22 + \left(\frac{\displaystyle \frac{44}{2} - 20}{10}\right) \cdot 0,7 = 4,45
    \end{equation*}

    \item \textbf{Moda:}
    
    \textbf{Datos agrupados(continua):}
    \begin{equation*}
        M_o = FI_k + \left(\frac{a}{a + b}\right) \cdot A_k = 20 + \left(\frac{10}{10 + 9}\right) \cdot 0,7 = 4,45
    \end{equation*}
\end{itemize}
\end{document}