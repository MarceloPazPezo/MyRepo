\documentclass{templateNote}
\usepackage{tcolorbox}
\usepackage{amsmath}
\usepackage{pgfplots}
\usepackage{pgf-pie}
\usepackage{tabularx}


\pgfplotsset{compat=1.18}
\newcolumntype{L}{>{\centering\arraybackslash}X}
\begin{document}

\imagenlogo{img/logoubb.png}
\universidad{Universidad del Bío-Bío}
\titulo{TAREA-220149} % Titulo
\asignatura{Estadistica y Probabilidad} % Asignatura
\autor{
    Marcelo \textsc{Paz}
    \qquad
    José \textsc{Peña}
    % \qquad
    % Bastian \textsc{Rodriguez}
    \qquad
    Claudia \textsc{Sobino}
}
\portada
\margenes % Crear márgenes

\section{Problema 1}
\indent
Suponga que el número de autos X, que pasan a través de una máquina lavadora, entre las 16:00 y las 17:00
horas de un día viernes determinado, tiene la siguiente función de probabilidad:

\begin{figure}[H]
    \begin{center}
        \begin{tabularx}{\textwidth}{|L|L|L|L|L|L|L|}
            \hline
            \textbf{x} & 4 & 5 & 6 & 7 & 8 & 9 \\
            \hline
            & & & & & & \\
            \textbf{P(X=x)} & $\displaystyle\frac{1}{12}$ & $\displaystyle\frac{1}{12}$ & $\displaystyle\frac{1}{4}$ & $\displaystyle\frac{1}{4}$ & $\displaystyle\frac{1}{6}$ & $\displaystyle\frac{1}{6}$ \\
            & & & & & & \\
            \hline
        \end{tabularx}
        \caption{Tabla de probabilidad}
    \end{center}
\end{figure}

Sea $g(X) = 2X - 1$ que representa la cantidad de dinero en dólares que el gerente del negocio le paga al
encargado.

Definición de evento:
\begin{itemize}
    \item $X$: ''Número de autos que pasan por la lavadora''
\end{itemize}

\begin{figure}[H]
    \begin{center}
        \begin{tikzpicture}
            \begin{axis}[
                xtick distance=1,
                xlabel style={at={(rel axis cs:1.04,0)}, anchor=west},
                xlabel={$x$},
                ylabel style={rotate =-90 ,at={(rel axis cs:0.0,1.03)}, anchor=south},
                ymin = 0,
                ymax = 1,
                xmax = 10,
                ylabel={$f(x)$},
                grid=none,
                axis lines = left,
                tick style={draw=none} % Esto 
            ]
            \addplot+[ycomb] plot coordinates
            {(1,0) (2,0) (3,0) (4,0.08333) (5,0.08333) (6,0.25) (7,0.25) (8,0.16667) (9,0.16667)};
            \end{axis}
        \end{tikzpicture}
        \caption{Representación gráfica de la función de probabilidad}
    \end{center}
\end{figure}

\newpage
\subsection{a}
Encuentre las ganancias esperadas en este período de tiempo en particular.

\begin{align*}
    E(g(x)) &= \displaystyle \sum_{i=1}^{n}{g(x) \cdot P(X=x)} && \text{Remplazando} \\
    &= (7 \cdot \frac{1}{12}) + (9 \cdot \frac{1}{12}) + (11 \cdot \frac{1}{4}) + (13 \cdot \frac{1}{4}) + (15 \cdot \frac{1}{6}) + (17 \cdot \frac{1}{6}) \\
    &= \frac{7}{12} + \frac{9}{12} + \frac{33}{12} + \frac{39}{12} + \frac{30}{12} + \frac{34}{12} \\
    &= \frac{154}{12} \\
    &= 12,6667
\end{align*}

Por lo tanto, las ganancias esperadas en este período de tiempo en particular son de \$12,6667 aproximadamente \$13 dólares.  

\subsection{b}
¿Cuál es la probabilidad de que pasen 9 autos el día viernes entre las 16:00 y las 17:00 horas?

\begin{align*}
    P(X=9) &= \frac{1}{6}
\end{align*}

Por lo tanto, la probabilidad de que pasen 9 autos el día viernes entre las 16:00 y las 17:00 horas es de $\frac{1}{6}$.

\subsection{c}
Determine: $V(4+3X)$, $V(4)$, $E(6X)$, $E(5+5X)$, $V(9X)$.

Primero se calculara $V(X) = E(X^2) - (E(X)^2)$:
\begin{align*}
    E(X^2) &= \sum_{i=1}^{n}{x^2 \cdot P(X=x)} && \text{Remplazando} \\
    &= (16 \cdot \frac{1}{12}) + (25 \cdot \frac{1}{12}) + (36 \cdot \frac{1}{4}) + (49 \cdot \frac{1}{4}) + (64 \cdot \frac{1}{6}) + (81 \cdot \frac{1}{6}) \\
    &= \frac{16}{12} + \frac{25}{12} + 9 + \frac{49}{4} + \frac{64}{6} + \frac{81}{6} \\
    &= \frac{16}{12} + \frac{25}{12} + \frac{108}{12} + \frac{147}{12} + \frac{128}{12} + \frac{162}{12} \\
    &= \frac{586}{12} \\
    &= 48,8333
\end{align*}

\begin{align*}
    E(X) &= \sum_{i=1}^{n}{x \cdot P(X=x)} && \text{Remplazando} \\
    &= (4 \cdot \frac{1}{12}) + (5 \cdot \frac{1}{12}) + (6 \cdot \frac{1}{4}) + (7 \cdot \frac{1}{4}) + (8 \cdot \frac{1}{6}) + (9 \cdot \frac{1}{6}) \\
    &= \frac{4}{12} + \frac{5}{12} + \frac{6}{4} + \frac{7}{4} + \frac{8}{6} + \frac{9}{6} \\
    &= \frac{4}{12} + \frac{5}{12} + \frac{18}{12} + \frac{21}{12} + \frac{16}{12} + \frac{18}{12} \\
    &= \frac{82}{12} \\
    &= 6,8333
\end{align*}

Asi:
\begin{align*}
    V(X) &= E(X^2) - (E(X)^2) \\
    &= 48,8333 - (6,8333)^2 \\
    &= 48,8333 - 46,6944 \\
    &= 2,1389
\end{align*}

Para $V(4+3X)$:
\begin{align*}
    V(4+3X) &= V(4) + V(3X) \\
    &= 0 + 9 \cdot V(X) \\
    &= 9 \cdot 2,1389 \\
    &= 19,2501
\end{align*}

Para $V(4)$:
\begin{align*}
    V(4) &= 0
\end{align*}

Para $E(6X)$:
\begin{align*}
    E(6X) &= 6 \cdot E(X) \\
    &= 6 \cdot 6,8333 \\
    &= 40,9998
\end{align*}

Para $E(5+5X)$:
\begin{align*}
    E(5+5X) &= 5 + E(5X) \\
    &= 5 + 5 \cdot E(X) \\
    &= 5 + 5 \cdot 6,8333 \\
    &= 39,1665
\end{align*}

Para $V(9X)$:
\begin{align*}
    V(9X) &= 81 \cdot V(X) \\
    &= 81 \cdot 2,1389 \\
    &= 173,2509
\end{align*}

\subsection{d}
Obtenga la f.d.a. y grafíquela.

\begin{align*}
    F(X) =
    \displaystyle \begin{cases}
        0 \quad \text{ , } \quad \quad x < 4\\
        \frac{1}{12} \quad \text{, }4 \leq x < 5 \\
        \frac{2}{12} \quad \text{, }5 \leq x < 6 \\
        \frac{5}{12} \quad \text{, }6 \leq x < 7 \\
        \frac{8}{12} \quad \text{, }6 \leq x < 8 \\
        \frac{10}{12} \quad \text{, }8 \leq x < 9 \\
        \frac{12}{12} \quad \text{, }9 \leq x
    \end{cases}
\end{align*}


// Hacer gráfico que si coincida
\begin{figure}[H]
    \begin{center}
        \begin{tikzpicture}
            \begin{axis}[
                xtick distance=1,
                xlabel style={at={(rel axis cs:1.04,0)}, anchor=west},
                xlabel={$x$},
                ylabel style={rotate =-90 ,at={(rel axis cs:0.0,1.03)}, anchor=south},
                ymin = 0,
                ymax = 1.2,
                xmax = 10,
                ylabel={$F(x)$},
                grid=none,
                axis lines = left,
                % tick style={draw=none} % Esto 
            ]

            \addplot [
                domain=0:4, 
                samples=100, 
                color=blue,
            ]
            {0};

            
            \addplot [
                domain=4:5, 
                samples=100, 
                color=blue,
            ]
            {1/12};
            
            \addplot [
                domain=5:6, 
                samples=100, 
                color=blue,
            ]
            {2/12};
            
            \addplot [
                domain=6:7, 
                samples=100, 
                color=blue,
            ]
            {5/12};
            
            \addplot [
                domain=7:8, 
                samples=100, 
                color=blue,
            ]
            {8/12};
            
            \addplot [
                domain=8:9, 
                samples=100, 
                color=blue,
            ]
            {10/12};
            
            \addplot [
                domain=9:10, 
                samples=100, 
                color=blue,
            ]
            {12/12};
            
            \filldraw (0,0) circle (2pt);
             (4,0) circle (2pt);
            
            \filldraw (4,1/12) circle (2pt);
            \filldraw[fill=white, draw=black] (5,1/12) circle (2pt);

            \filldraw (5,2/12) circle (2pt);
            \filldraw[fill=white, draw=black] (6,2/12) circle (2pt);

            \filldraw (6,5/12) circle (2pt);
            \filldraw[fill=white, draw=black] (7,5/12) circle (2pt);

            \filldraw (7,8/12) circle (2pt);
            \filldraw[fill=white, draw=black] (8,8/12) circle (2pt);

            \filldraw (8,10/12) circle (2pt);
            \filldraw[fill=white, draw=black] (9,10/12) circle (2pt);

            \filldraw (9,12/12) circle (2pt);
            \end{axis}
        \end{tikzpicture}
        \caption{Representación gráfica de la función de probabilidad}
    \end{center}
\end{figure}

\subsection{e}
Calcule: $P(X > 6)$, $P(X \leq 8)$ y $P(5 \leq X \leq 8)$.

Para $P(X > 6)$:
\begin{align*}
    P(X > 6) &= 1 - F(6) \\
    &= 1 - \frac{5}{12} \\
    &= \frac{7}{12} \\
    &= 0,5833
\end{align*}

Para $P(X \leq 8)$:
\begin{align*}
    P(X \leq 8) = \displaystyle \sum_{i=1}^{n}{P(X=x)} && \text{Remplazando} \\
    &= \frac{1}{12} + \frac{1}{12} + \frac{1}{4} + \frac{1}{4} + \frac{1}{6} \\
    &= \frac{1}{12} + \frac{1}{12} + \frac{3}{12} + \frac{3}{12} + \frac{2}{12} \\
    &= \frac{10}{12} \\
    &= 0,8333
\end{align*}
\begin{align*}
    P(X \leq 8) &= F(8) && \text{Remplazando} \\
    &= \frac{10}{12} \\
    &= 0,8333
\end{align*}
\begin{align*}
    P(X \leq 8) &= 1 - P(X = 9) && \text{Remplazando} \\
    &= 1 - \frac{1}{6} \\
    &= \frac{5}{6} \\
    &= 0,8333
\end{align*}

Para $P(5 \leq X \leq 8)$:
\begin{align*}
    P(5 \leq X \leq 8) &= F(8) - F(4) && \text{Remplazando} \\
    &= \frac{10}{12} - \frac{1}{12} \\
    &= \frac{9}{12} \\
    &= 0,75
\end{align*}

\newpage
\section{Problema 2}
\indent
Considera la función $f(x)$ de la variable aleatoria X, que está dada por:
\begin{align*}
    f(x) = \begin{cases}
        kx \quad \quad \text{, } \quad 0 \leq x < 1 \\
        k - x \quad \text{, }\quad 1 \leq x < 2 \\
        0 \quad \quad \quad \text{, } \quad \quad \text{e.o.c}
    \end{cases}
\end{align*}

\subsection{a}
Determine el valor de k para que $f(x)$ sea una función de densidad de probabilidad.

Para que $f(x)$ sea una función de densidad de probabilidad, se debe cumplir que:
\begin{align*}
    \int_{R_x}{f(x)dx} = 1
\end{align*}

Por lo tanto es necesario calcular el valor de $k$ para que se cumpla la condición anterior:
\begin{align*}
    \int_{0}^{1}{(kx)dx} + \int_{1}^{2}{(k-x)dx} = 1 && \text{Aplicando la integral} \\
    \frac{kx^2}{2} \Big|_{0}^{1} + (kx - \frac{x^2}{2}) \Big|_{1}^{2} = 1 && \text{Remplazando} \\
    \frac{k}{2} + (2k - 2 - k + \frac{1}{2}) = 1 && \text{Despejando k}\\
    \frac{k}{2} + (k - \frac{3}{2}) = 1 \\
    \frac{k}{2} + k - \frac{3}{2} = 1 \\
    \frac{3k}{2} - \frac{3}{2} = 1 \\
    \frac{3k}{2} = \frac{5}{2} \\
    k = \frac{5}{3}
\end{align*}

Por lo tanto, para que $f(x)$ sea una función de densidad de probabilidad:

\begin{align*}
    f(x) = \begin{cases}
        \frac{5}{3}x \quad \quad \text{, } \quad 0 \leq x < 1 \\
        \frac{5}{3} - x \quad \text{, }\quad 1 \leq x < 2 \\
        0 \quad \quad \quad \text{, } \quad \quad \text{e.o.c}
    \end{cases}
\end{align*}

\newpage
\subsection{b}
Grafique $f(x)$

\begin{figure}[H]
    \begin{center}
        \begin{tikzpicture}
            \begin{axis}[
                xtick distance=1,
                xlabel style={at={(rel axis cs:1.04,0)}, anchor=west},
                xlabel={$x$},
                ylabel style={rotate =-90 ,at={(rel axis cs:0.0,2.0)}, anchor=south},
                ymin = -1,
                ymax = 2,
                xmax = 2,
                ylabel={$P(X=x)$},
                grid=none,
                axis lines = middle,
            ]
            \addplot [
                domain=0:1, 
                samples=100, 
                color=blue,
            ]
            {(5/3)*x};
            \filldraw[fill=white, draw=black] (1,5/3) circle (2pt);

            \filldraw (1,5/3 - 1) circle (2pt);
            \addplot [
                domain=1:2, 
                samples=100, 
                color=blue,
            ]
            {(5/3) - x};
            \filldraw[fill=white, draw=black] (2,5/3 - 2) circle (2pt);

            \addplot [
                domain=2:3, 
                samples=100, 
                color=blue,
            ]
            {0};

            \end{axis}
        \end{tikzpicture}
    \end{center}
\end{figure}

\subsection{c}
Calcule: $\displaystyle P\left(\frac{1}{2} \leq X \leq \frac{3}{2}\right)$  y $\displaystyle P\left(X \geq \frac{3}{8}\right)$

Para $\displaystyle P\left(\frac{1}{2} \leq X \leq \frac{3}{2}\right)$:
\begin{align*}
    \displaystyle P\left(\frac{1}{2} \leq X \leq \frac{3}{2}\right) &= \int_{\frac{1}{2}}^{1}{(\frac{5}{3}x)dx} + \int_{1}^{\frac{3}{2}}{{(\frac{5}{3} - x)dx}} && \text{Aplicando la integral} \\
    &= \frac{5}{6}x^2 \Big|_{\frac{1}{2}}^{1} + (\frac{5}{3}x - \frac{x^2}{2}) \Big|_{1}^{\frac{3}{2}} && \text{Remplazando} \\
    &= \frac{5}{6} - \frac{5}{24} + (\frac{5}{2} - \frac{9}{8} - \frac{5}{3} + \frac{1}{2}) \\
    &= \frac{20}{24} - \frac{5}{24} + (3 - \frac{67}{24}) \\
    &= \frac{15}{24} + 3 - \frac{67}{24} \\
    &= \frac{15}{24} + \frac{72}{24} - \frac{67}{24} \\
    &= \frac{20}{24} \\
    &= \frac{5}{6} \\
    &= 0,8333
\end{align*}

Para $\displaystyle P\left(X \geq \frac{3}{8}\right)$:
\begin{align*}
    \displaystyle P\left(X \geq \frac{3}{8}\right) &= 1 - F(\frac{3}{8})\\
    &= 1 - \int_{0}^{\frac{3}{8}}{(\frac{5}{3}x)dx} && \text{Aplicando la integral} \\
    &= 1 - \frac{5}{6}x^2 \Big|_{0}^{\frac{3}{8}} && \text{Remplazando} \\
    &= 1 - \frac{5}{6} \cdot \frac{9}{64} \\
    &= 1 - \frac{15}{128} \\
    &= \frac{113}{128} \\
    &= 0,8828
\end{align*}

\subsection{d}
Determine el valor de: $E(6-5X)$ y $V(6-5X)$

Para $E(6-5X)$:

Es necesario:
\begin{align*}
    E(X) &= \int_{-\infty}^{+\infty}{(x\cdot f(x))dx} \\
    &= \int_{0}^{1}{(x\cdot \frac{5}{3}x)dx} + \int_{1}^{2}{(x\cdot (\frac{5}{3} - x))dx} \\
    &= \frac{5x^3}{9}\Big|_{0}^{1} + \left(\frac{5x^2}{6} - \frac{x^3}{3}\right)\Big|_{1}^{2} \\
    &= \frac{5}{9} + \frac{20}{6} - \frac{8}{3} - \left(\frac{5}{6} - \frac{1}{3}\right) \\
    &= \frac{5}{9} + \frac{20}{6} - \frac{16}{6} - \frac{15-6}{18} \\
    &= \frac{5}{9} + \frac{4}{6} - \frac{1}{2} \\
    &= \frac{5}{9} + \frac{1}{6} \\
    &= \frac{13}{18}
\end{align*}
\begin{align*}
    E(6-5X) &= E(6) - E(5X) \\
    &= 6 - 5E(X) \\
    &= 6 - 5 \left(\frac{13}{18}\right) \\
    &= 6 - \frac{65}{18}\\
    &= \frac{43}{18} \\
    &= 2,3889
\end{align*}

Para $V(6-5X)$:

Es necesario:
\begin{align*}
    E(X^2) &= \int_{-\infty}^{+\infty}{(x^2\cdot f(x))dx} \\
    &= \int_{0}^{1}{(x^2\cdot \frac{5}{3}x)dx} + \int_{1}^{2}{(x^2\cdot (\frac{5}{3} - x))dx} \\
    &= \frac{5x^4}{12}\Big|_{0}^{1} + \left(\frac{5x^3}{9} - \frac{x^4}{4}\right)\Big|_{1}^{2} \\
    &= \frac{5}{12} + \frac{40}{9} - \frac{16}{4} - \left(\frac{5}{9} - \frac{1}{4}\right) \\
    &= \frac{5}{12} + \frac{40}{9} - \frac{16}{6} - \frac{15-6}{18} \\
    &= \frac{5}{12} + \frac{5}{36} \\
    &= \frac{5}{9}
\end{align*}
\begin{align*}
    V(6-5X) &= V(6) - 5^2V(X) \\
    &= 0 - 25 V(X) \\
    &= - 25 \left(E(x^2) - (E(x))^2\right) \\
    &= -25 \left(\frac{5}{9} - \left(\frac{13}{18}\right)^2\right) \\
    &= -25 \left(\frac{5}{9} - \frac{169}{324}\right) \\
    &= -25 \left(\frac{11}{324}\right) \\
    &= - \frac{275}{324} \\
    &= -0,8488
\end{align*}

\newpage
\section{Problema 3}
Si un banco de Concepción recibe en promedio 6 cheques sin fondo por día:
\subsection{a}
¿Cuál es la probabilidad de que se reciban 4 cheques sin fondo en un día determinado?

Sea $X$: número de cheques sin fondo por día.

$X \sim P(6)$
\begin{align*}
    P(X=4) &= \frac{\lambda^x \cdot e^{-\lambda}}{x!} \\
    &= \frac{6^4 \cdot e^{-6}}{4!} \\
    &= 0,1339
\end{align*}

\subsection{b}
¿Cuál es la probabilidad de que no se reciba ningún cheque sin fondo en 2 días?

Sea $X$: número de cheques sin fondo en 2 días.

$X \sim P(12)$
\begin{align*}
    P(X=0) &= \frac{\lambda^x \cdot e^{-\lambda}}{x!} \\
    &= \frac{12^0 \cdot e^{-12}}{0!} \\
    &= \frac{1 \cdot e^{-12}}{1} \\
    &= e^{-12} \\
    &= \frac{1}{e^{12}} \\
    &= 0,000006144
\end{align*}
\subsection{c}
¿Cuál es la probabilidad de que se reciban más de tres cheques sin fondo en 2 días?
\begin{align*}
    P(X>3) &= 1 - P(X \leq 3)\\
    &= 1 - \left(P(X=0) + P(X=1) + P(X=2) + P(X=3)\right) \\
    &= 1 - \left(\frac{12^0 \cdot e^{-12}}{0!} + \frac{12^1 \cdot e^{-12}}{1!} + \frac{12^2 \cdot e^{-12}}{2!} + \frac{12^3 \cdot e^{-12}}{3!}\right) \\
    &= 1 - 0,0023 \\
    &= 0,9977
\end{align*}

\newpage
\subsection{d}
¿Cuántos cheques sin fondo se reciben en promedio por año?

Sea $X$: número de cheques sin fondo en 365 día.

$X \sim P(2190)$
\begin{align*}
    E(X) &= \lambda \\
    &= 2190
\end{align*}
\subsection{e}
Si en un día determinado el banco recibe 10 cheques, ¿cuál es la probabilidad de que exactamente 4
no tengan fondos?

$X \sim P(6)$
\begin{align*}
    P(X=4) &= \frac{\lambda^x \cdot e^{-\lambda}}{x!} \\
    &= \frac{6^4 \cdot e^{-6}}{4!} \\
    &= 0,1339
\end{align*}

\section{Problema 4}
En un determinado aeropuerto del país, se sabe que 7 de 14 maletas contienen artículos de contrabando.
Determine la probabilidad de que exactamente 4 de 6 maletas seleccionadas en la inspección al azar de
pasajeros, contengan artículos de contrabando.

Sea $X$: Número de maletas que contienen artículos de contrabando.

$X \sim H(14, 6, 7)$
\begin{align*}
    P(X=x) &=  \frac{\displaystyle \binom{7}{x} \binom{7}{6-x}}{\displaystyle \binom{14}{6}}  && \text{, x=0, 1, 2, ..., min\{6,7\}}\\
    \\
    P(X=4) &= \frac{\displaystyle \binom{7}{4} \binom{14-7}{6-4}}{\displaystyle \binom{14}{6}} = \frac{\displaystyle \binom{7}{4} \binom{7}{2}}{\displaystyle \binom{14}{6}} \\
    \\
    &= \frac{\displaystyle \frac{7 \cdot 6 \cdot 5}{3!} \cdot \frac{7 \cdot 6}{2!}}{\displaystyle \frac{14 \cdot 13 \cdot 12 \cdot 11 \cdot 10 \cdot 9}{6!}} \\
    \\
    &= \frac{35}{143} \\
    &= 0,2448
\end{align*}
\end{document}