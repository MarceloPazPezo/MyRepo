\documentclass{templateNote}
\usepackage{tcolorbox}
\usepackage{pgfplots}
\usepackage{pgf-pie}
\usepackage{tabularx}
\usepackage{graphicx}
\usepackage{tikz}
\usepackage{amsmath}
\usepackage{amsfonts}

\newcolumntype{L}{>{\centering\arraybackslash}X}
\pgfmathdeclarefunction{gauss}{2}{%
  \pgfmathparse{1/(#2*sqrt(2*pi))*exp(-((x-#1)^2)/(2*#2^2))}%
}
\begin{document}

\imagenlogo{img/LogoElNube.png}
\universidad{Universidad del Bío-Bío}
\titulo{Practico distribución normal} % Titulo
\asignatura{Estadistica y Probabilidades} % Asignatura
\autor{
    \indent
    Marcelo \textsc{Paz}
}   
\portada
\margenes % Crear márgenes

\section*{Ejercicio 1}
Las longitudes de las sardinas recibidas por cierta enlatadora tienen una distribución normal con media  $\mu = 4.62$ pulgadas y desviación estándar $\sigma = 0.23$ pulgadas.

\textbf{X:} longitudes de las sardinas.

Notación: $X \sim N(4.62, 0.23^2)$
\begin{itemize}
  \item \textbf{a)} ¿Qué porcentaje de todas estas sardinas son mayores de 5 pulgadas?
  
  Notación normal estándar: $Z \sim N(0, 1)$
  \begin{align*}
    P(X > 5) &= 1 - P(X \leq 5) && \text{Estandarización}\\
    &= 1 - P\left(\frac{X - \mu}{\sigma} \leq \frac{5 - 4.62}{0.23}\right) \\
    &= 1 - P\left(Z \leq \frac{5 - 4.62}{0.23}\right) && \text{, } Z = \frac{X - \mu}{\sigma} \\
    &= 1 - P(Z \leq 1.65) \\
    &= 1 - 0.9505 \\
    &= 0.0495
  \end{align*}
  
  \textbf{Respuesta:} Por lo tanto, el $4.95\%$ de las sardinas son mayores a 5 pulgadas.

  \item \textbf{b)} ¿Qué porcentaje de las sardinas miden entre 4.35 y 4.85 pulgadas?
  \begin{align*}
    P(4.35 \leq X \leq 4.85) &= P\left(\frac{4.35 - 4.62}{0.23} \leq Z \leq \frac{4.85 - 4.62}{0.23}\right) \\
    &= P(-1.17 \leq Z \leq 1.00) \\
    &= P(Z \leq 1.00) - P(Z \leq -1.17) \\
    &= 0.8413 - 0.1210 \\
    &= 0.7203
  \end{align*}

  \textbf{Respuesta:} Por lo tanto, el $72.03\%$ de las sardinas miden entre 4.35 y 4.85 pulgadas.
  
  \item \textbf{c)} ¿Qué porcentaje de las sardinas miden como máximo 4.62 pulgadas?
  \begin{align*}
    P(X \leq 4.62) &= P\left(\frac{X - \mu}{\sigma} \leq \frac{4.62 - 4.62}{0.23}\right) \\
    &= P(Z \leq 0) \\
    &= 0.5000
  \end{align*}

  \textbf{Respuesta:} Por lo tanto, el $50\%$ de las sardinas miden como máximo 4.62 pulgadas.
\end{itemize}

\newpage
\section*{Ejercicio 2}
Dos estudiantes fueron informados de sus puntajes estandarizados en un examen de inglés, y corresponden a $0.8$ y $-0.4$, respectivamente. Si sus puntuaciones fueron $88$ y $64$, respectivamente, ¿cuál fue el promedio y la desviación estándar del examen de inglés?

\textbf{X:} puntajes en el examen de inglés.

Notación: $X \sim N(\mu, \sigma^2)$

Recordemos:
\begin{align*}
  Z = \frac{X - \mu}{\sigma}
\end{align*}

Para el primer estudiante:
\begin{align*}
  Z &= \frac{X - \mu}{\sigma} \\
  0.8 &= \frac{88 - \mu}{\sigma} && \text{Remplazando} \\
  0.8\sigma &= 88 - \mu \\
  \sigma &= \frac{88 - \mu}{0.8}
\end{align*}

Para el segundo estudiante:
\begin{align*}
  Z &= \frac{X - \mu}{\sigma} \\
  -0.4 &= \frac{64 - \mu}{\sigma} && \text{Remplazando} \\
  -0.4\sigma &= 64 - \mu \\
  \sigma &= \frac{64 + 0.4}{-0.4}
\end{align*}

Igualando ambas ecuaciones:
\begin{align*}
  \frac{88 - \mu}{0.8} &= \frac{64 - \mu}{-0.4} \\
  -0.4(88 - \mu) &= 0.8(64 - \mu) \\
  -35.2 + 0.4\mu &= 51.2 - 0.8\mu \\
  0.4\mu + 0.8\mu &= 51.2 + 35.2 \\
  1.2\mu &= 86.4 \\
  \mu &= \frac{86.4}{1.2} \\
  \mu &= 72
\end{align*}

Remplazando en la ecuación del primer estudiante:
\begin{align*}
  \sigma &= \frac{88 - 72}{0.8} \\
  \sigma &= \frac{16}{0.8} \\
  \sigma &= 20
\end{align*}

\textbf{Respuesta}: Asi, $\mu = 72$ y $\sigma = 20$

Notación: $X \sim N(72, 20^2)$

\newpage
\section*{Ejercicio 3}
En una industria alimenticia se envasa café instantáneo en frascos cuyos pesos netos tienen una distribución normal con desviación estándar de 5.5 gramos. Si el 5\% de los frascos pesa a lo menos 139 gramos, ¿cuál es el promedio de ellos?

\textbf{X:} pesos netos de los frascos.

Notación: $X \sim N(\mu, 5.5^2)$

\begin{align*}
  P(X \geq 139) &= 0.05 \\
  P\left(\frac{X - \mu}{5.5} \geq \frac{139 - \mu}{5.5}\right) &= 0.05 \\
  P\left(Z \geq \frac{139 - \mu}{5.5}\right) &= 0.05 && \text{, } Z = \frac{X - \mu}{5.5} \\
  1 - P\left(Z \leq \frac{139 - \mu}{5.5}\right) &= 0.05 \\
  P\left(Z \leq \frac{139 - \mu}{5.5}\right) &= 0.95 \\
  \frac{139 - \mu}{5.5} &= 1.65 \\
  139 - \mu &= 9.075 \\
  \mu &= 129.925
\end{align*}

\textbf{Respuesta:} Por lo tanto, el promedio de los frascos es de 129.925 gramos.

\newpage
\section*{Ejercicio 4}
Las alturas de los naranjos están distribuidos en forma normal. Se sabe que un 2.28\% miden más de 14pies y un 84.13\% menos de 12 pies. Determine la altura media de los naranjos y la desviación estándar de los naranjos.

\textbf{X:} alturas de los naranjos.

Notación: $X \sim N(\mu, \sigma^2)$

Sabemos que:
\begin{align*}
  P(X > 14) &= 0.0228 \\
  P(X < 12) &= 0.8413
\end{align*}

Resolviendo $P(X > 14)$:
\begin{align*}
  P(X > 14) &= 0.0228 \\
  P\left(\frac{X - \mu}{\sigma} > \frac{14 - \mu}{\sigma}\right) &= 0.0228 \\
  P\left(Z > \frac{14 - \mu}{\sigma}\right) &= 0.0228 && \text{, } Z = \frac{X - \mu}{\sigma} \\
  1 - P\left(Z \leq \frac{14 - \mu}{\sigma}\right) &= 0.0228 \\
  P\left(Z \leq \frac{14 - \mu}{\sigma}\right) &= 0.9772 \\
  \frac{14 - \mu}{\sigma} &= 2 \\
  14 - \mu &= 2\sigma && \text{Despejando } \mu \\
  \mu &= 14 - 2\sigma
\end{align*}

Resolviendo $P(X < 12)$:
\begin{align*}
  P(X < 12) &= 0.8413 \\
  P\left(\frac{X - \mu}{\sigma} < \frac{12 - \mu}{\sigma}\right) &= 0.8413 \\
  P\left(Z < \frac{12 - \mu}{\sigma}\right) &= 0.8413 && \text{, } Z = \frac{X - \mu}{\sigma} \\
  \frac{12 - \mu}{\sigma} &= 1 \\
  12 - \mu &= \sigma && \text{Despejando } \mu \\
  \mu &= 12 - \sigma
\end{align*}

\newpage
Igualando ambas ecuaciones:
\begin{align*}
  14 - 2\sigma &= 12 - \sigma \\
  2\sigma - \sigma &= 14 - 12 \\
  \sigma &= 2
\end{align*}

Remplazando en la ecuación de $P(X > 14)$:
\begin{align*}
  \mu &= 14 - 2\sigma \\
  \mu &= 14 - 2(2) \\
  \mu &= 14 - 4 \\
  \mu &= 10
\end{align*}

\textbf{Respuesta:} Por lo tanto, la altura media de los naranjos es de 10 pies y la desviación estándar de los naranjos es de 2 pies.

\section*{Ejercicio 5}
El tiempo de trabajo (en horas) que emplean los ejecutivos de ciertas empresas sigue una distribución normal con media  $\mu = 8$ y desviación estándar  $\sigma = 4$.

\begin{itemize}
  \item \textbf{a)} ¿Qué porcentaje de estos ejecutivos trabaja más de 7 horas?
  \begin{align*}
    P(X > 7) &= 1 - P(X \leq 7) && \text{Estandarización}\\
    &= 1 - P\left(\frac{X - \mu}{\sigma} \leq \frac{7 - 8}{4}\right) \\
    &= 1 - P\left(Z \leq \frac{7 - 8}{4}\right) && \text{, } Z = \frac{X - \mu}{\sigma} \\
    &= 1 - P(Z \leq -0.25) \\
    &= 1 - 0.4013 \\
    &= 0.5987
  \end{align*}

  \textbf{Respuesta:} Por lo tanto, el $59.87\%$ de los ejecutivos trabaja más de 7 horas.
  
  \item \textbf{b)}	¿Si el 20\% de los ejecutivos trabajan menos de $x_0$ horas. Encuentre $x_0$.
  \begin{align*}
    P(X < x_0) &= 0.20 \\
    P\left(\frac{X - \mu}{\sigma} < \frac{x_0 - \mu}{\sigma}\right) &= 0.20 \\
    P\left(Z < \frac{x_0 - \mu}{\sigma}\right) &= 0.20 && \text{, } Z = \frac{X - \mu}{\sigma} \\
    \frac{x_0 - \mu}{\sigma} &= -0.84 \\
    x_0 &= -0.84\sigma + \mu \\
    x_0 &= -0.84(4) + 8 \\
    x_0 &= 4.64
  \end{align*}

  \textbf{Respuesta:} Por lo tanto, $x_0 = 4.64$.
  
  \newpage
  \item \textbf{c)} ¿Cuál es la probabilidad de que un ejecutivo cualquiera trabaje entre 3 y 6 horas diarias?
  \begin{align*}
    P(3 \leq X \leq 6) &= P\left(\frac{3 - 8}{4} \leq Z \leq \frac{6 - 8}{4}\right) \\
    &= P(-1.25 \leq Z \leq -0.5) \\
    &= P(Z \leq -0.5) - P(Z \leq -1.25) \\
    &= 0.3085 - 0.1056 \\
    &= 0.2029
  \end{align*}
  
  \textbf{Respuesta:} Por lo tanto, la probabilidad de que un ejecutivo cualquiera trabaje entre 3 y 6 horas diarias es de $20.29\%$.

  \item \textbf{d)}	La cantidad de tiempo que dedica un ejecutivo a realizar tareas propias de sus subalternos, también sigue una distribución normal con media $\mu = 2.4$ horas. Determine la desviación estándar,$\sigma$, de este tiempo, si se sabe que el 10\% de los ejecutivos gasta más de 3.5 horas en tareas de este tipo.
  \begin{align*}
    P(X > 3.5) &= 0.10 \\
    P\left(\frac{X - \mu}{\sigma} > \frac{3.5 - \mu}{\sigma}\right) &= 0.10 \\
    P\left(Z > \frac{3.5 - \mu}{\sigma}\right) &= 0.10 && \text{, } Z = \frac{X - \mu}{\sigma} \\
    1 - P\left(Z \leq \frac{3.5 - \mu}{\sigma}\right) &= 0.10 \\
    P\left(Z \leq \frac{3.5 - \mu}{\sigma}\right) &= 0.90 \\
    \frac{3.5 - \mu}{\sigma} &= 1.28 \\
    3.5 - \mu &= 1.28\sigma \\
    \sigma &= \frac{3.5 - \mu}{1.28} && \text{Remplazando } \\
    \sigma &= \frac{3.5 - 2.4}{1.28} \\
    \sigma &= \frac{1.1}{1.28} \\
    \sigma &= 0.8594
  \end{align*}

  \textbf{Respuesta:} Por lo tanto, la desviación estándar es de 0.8594 horas.
\end{itemize}
\end{document}