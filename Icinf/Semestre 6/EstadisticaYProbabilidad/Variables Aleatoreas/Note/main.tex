\documentclass{templateNote}
\usepackage{tcolorbox}
\usepackage{pgfplots}
\usepackage{pgf-pie}
\usepackage{tabularx}
\usepackage{graphicx}
\usepackage{tikz}

\newcolumntype{L}{>{\centering\arraybackslash}X}
\pgfmathdeclarefunction{gauss}{2}{%
  \pgfmathparse{1/(#2*sqrt(2*pi))*exp(-((x-#1)^2)/(2*#2^2))}%
}
\begin{document}

\imagenlogo{img/LogoElNube.png}
\universidad{Universidad del Bío-Bío}
\titulo{Practico 3} % Titulo
\asignatura{Estadistica y Probabilidades} % Asignatura
\autor{
    \indent
    Marcelo \textsc{Paz}
}   
\portada
\margenes % Crear márgenes

\section{Teoría}
\begin{itemize}
    \item \textbf{Definición 1:} Una variable aleatoria (v.a.) es una función que transforma los resultados del
    espacio muestral asociado a un experimento en números.

    \item \textbf{Definición 2:} Las variables aleatorias pueden ser de dos tipos: Discretas y Continuas. Una
    variable aleatoria discreta es aquella que solo puede tomar valores enteros. Una variable
    aleatoria continua es aquella cuyos resultados pertenecen a uno o más conjuntos de los reales. 

    \item \textbf{Definición 3:} : Llamaremos función de probabilidad o función de cuantía de la v.a. discreta X, a
    $P(X = x)$ o bien a $p(x)$ si satisface las siguientes dos condiciones:
    \begin{align*}
        P(X=x) \geq 0, \forall x \in R_{X} \\
        \displaystyle\sum_{i=1}^{n}{P(X=x) = 1}
    \end{align*}

    \item \textbf{Definición 4:} Llamaremos función de densidad de la v.a. continua X, a $f(x)$, si satisface las
    siguientes dos condiciones:
    \begin{align*}
        f(x) \geq 0, \forall x \in R_{X} \\
        \int_{R_x}{f(x)dx} = 1
    \end{align*}

    \item \textbf{Definición 5:} Llamaremos función de distribución acumulada (f.d.a.) a la probabilidad de que
    X sea menor o igual que x y la denotaremos por $F(x)$. Formalmente,
    \begin{align*}
        F(x) := P(X \leq x) = \begin{cases}
            \displaystyle\sum_{i=1}^{n}{P(X=x)}, 
            & \text{si X es discreta} \\
            \\
            \displaystyle\int_{-\infty}^{x}{f(u)du},
            & \text{si U es continua}
        \end{cases}
    \end{align*}

    \textbf{Propiedades} de $F(x)$, cuando $X$ es discreta:
    \begin{enumerate}
        \item $\quad 0 \leq F(x) \leq 1$
        
        \item $\quad P(X>x) = 1 - F(X)$
        
        \item $\quad P(X=x) = F(x) - F(x-1)$
        
        \item $\quad P(x_i < X \leq x_j) = F(x_j) - F(x_{i-1})$
    \end{enumerate}  

    \newpage
    \textbf{Propiedades} de $F(x)$, cuando $X$ es continua:
    \begin{enumerate}
        \item $\quad 0 \leq F(x) \leq 1$
        
        \item $\quad P(X>x) = 1 - F(X)$
        
        \item $\quad P(X=x) = 0$
        
        \item $\quad P(x_i < X \leq x_j) = F(x_j) - F(x_{i-1}) = \int_{x_i}^{x_j}{f(x)dx}$
        
        \item $\quad \lim_{x \rightarrow-\infty}{F(x)} = 0 \quad y \quad \lim_{x \rightarrow+\infty}{F(x)} = 1$
    
        \item $ \frac{\partial}{\partial x}F(x) = f(x)$
    \end{enumerate} 

    \item \textbf{Definición 6:} Llamaremos Esperanza de una variable aleatoria X al promedio o media de ella y
    la denotaremos por $E(X)$. Formalmente, está definida por:
    \begin{align*}
        E(X) = \begin{cases}
            \displaystyle\sum_{i=1}^{n}{x_iP(X=x_i)}, 
            & \text{si X es discreta} \\
            \\
            \displaystyle\int_{-\infty}^{+\infty}{xf(x)dx},
            & \text{si U es continua}
        \end{cases}
    \end{align*}
    \textbf{Propiedades} de $E(X)$

    Consideremos c constante:

    \begin{enumerate}
        \item $\quad E(c) = c$
        
        \item $\quad E(X + c) = E(X) + c$
        
        \item $\quad E(cX) = cE(X)$
        
        \item $\quad E(g(X) + h(X)) = E(g(X)) + E(h(X))$
        
        Donde:
        \begin{align*}
            E(g(X)) = 
            \begin{cases}
                \displaystyle\sum_{i=1}^{n}{g(x)P(X=x)}, 
                & \text{si X es discreta} \\
                \\
                \displaystyle\int_{-\infty}^{+\infty}{g(x)f(x)dx},
                & \text{si U es continua}
            \end{cases}
        \end{align*}
    \end{enumerate}

    \item \textbf{Definición 7:} Llamaremos Varianza de la variable aleatoria $X$ a $V(X)$ o $Var(X)$ que está definida
    por:

    \begin{align*}
        V(X) = E(X - E(X))^2 = E(X^2) - E(X)^2
    \end{align*}

    \textbf{Propiedades} de $V(X)$

    Consideremos c constante:
    \begin{enumerate}
        \item $\quad V(c) = 0$
        \item $\quad V(X + c) = V(X)$
        \item $\quad V(cX) = c^2V(X)$
        \item $\quad V(X + Y) = V(X) + V(Y)$, si $X$ e $Y$ son independientes
    \end{enumerate}
    
\end{itemize}
\end{document}