\documentclass{templateNote}
\usepackage{tcolorbox}
\usepackage{tabularx}
\usepackage{amsmath}

\newcolumntype{L}{>{\centering\arraybackslash}X}

\begin{document}

\imagenlogo{img/LogoElNube.png}
\universidad{Universidad del Bío-Bío}
\titulo{Probabilidades} % Titulo
\asignatura{Estadistica y Probabilidad} % Asignatura
\autor{
    \indent
    Marcelo \textsc{Paz}
}   
\portada
\margenes % Crear márgenes


\section{Definiciones Generales}
\begin{enumerate}
    \item \textbf{Probabilidad:} Es la medida de la certeza de que un evento ocurra.
    \item \textbf{Experimento Determinista:} Es aquel que al repetirlo bajo las mismas condiciones, siempre da el \textbf{mismo resultado}.
    \item \textbf{Experimento Aleatorio:} Es aquel que al repetirlo bajo las mismas condiciones, puede dar \textbf{resultados distintos}.
    \item \textbf{Espacio Muestral:} Es el conjunto de todos los posibles resultados de un experimento aleatorio $\Omega$.
    \item \textbf{Evento:} Es un subconjunto del espacio muestral $A, B, C \subset \Omega$.
\end{enumerate}

\section{Teoria de Conjuntos}
\begin{enumerate}
    \item \textbf{Unión:} A o B o Ambos $A \cup B = \{x \in \Omega | x \in A \lor x \in B\}$
    \item \textbf{Intersección:} A y B o resultados comunes, $A \cap B = \{x \in \Omega | x \in A \land x \in B\}$
    \item \textbf{Vacio:} $\emptyset = \{\}$
    \item \textbf{Mutuamente excluyente:} $A \cap B = \emptyset$
    \item \textbf{Contenido:} $A \subseteq B \Leftrightarrow \forall x \in A \Rightarrow x \in B$
    \item \textbf{Complemento:} $A^c = \{x \in \Omega | x \notin A\}$
\end{enumerate}

\section{Probabilidad Clásica}
\indent
La probabilidad de A es la proporción de $n_A$ con respecto a $n$, esto es:
\begin{equation*}
    P(A) = \frac{n_A}{n_{\Omega}}
\end{equation*}

\newpage
\section{Funcion de Probabilidad}
\indent
Sea $\Omega$ un espacio muestral y sea $A$ un evento de $\Omega$. Se llamará función de
probabilidad sobre el espacio muestral a $P(A)$ si satisface los siguientes axiomas:
\begin{enumerate}
    \item $0 \leq P(A) \leq 1 $
    \item $P(\Omega) = 1 $
    \item Sea $A,B,C,...$ eventos mutuamente excluyentes, entonces: \newline
    $P(A \cup B \cup C \cup ...) = P(A) + P(B) + P(C) + ...$
\end{enumerate}

\section{Teoremas de Probabilidad}
\subsection{Teorema 1}
\indent
Si vacío es el evento imposible, entonces:
\begin{equation*}
    P(\emptyset) = 0
\end{equation*}

\subsection{Teorema 2}
\indent
Sean $A$ y $B$ dos eventos cualesquiera de $\Omega$, entonces:
\begin{equation*}
    P(A \cup B) = P(A) + P(B) - P(A \cap B)
\end{equation*}

\subsection{Teorema 3}
\indent
Sea $A$ un evento de $\Omega$, entonces:
\begin{equation*}
    P(A^c) = 1 - P(A)
\end{equation*}

\subsection{Teorema 4}
\indent
Sean $A$ y $B$ dos eventos cualesquiera de $\Omega$, entonces:
\begin{equation*}
    P(A \cap B^c) = P(A) - P(A \cap B)
\end{equation*}

\subsection{Teorema 5}
\indent
Sean $A$ y $B$ eventos de $\Omega$, tales que $A \subseteq B$, entonces:
\begin{equation*}
    P(A) \leq P(B)
\end{equation*}
\section{Propiedades de la Probabilidad}
\begin{enumerate}
    \item $P(A^c \cap B^c) = P((A \cup B)^c)$
    \item $P(A^c \cup B^c) = P((A \cap B)^c)$
\end{enumerate}

\newpage
\section{Tecnicas de Conteo}
\subsection{Principio de Multiplicación}
\indent
Si un evento $A$ puede ocurrir de $n_1$ formas distintas y si para cada una de estas formas, un evento $B$ puede ocurrir de $n_2$ formas distintas, entonces el evento $A$ seguido del evento $B$ puede ocurrir de $n_1 \cdot n_2$ formas distintas.
\subsection{Principio de Adición}
\indent
Si un evento $A$ puede ocurrir de $n_1$ formas distintas y si para cada una de estas formas, un evento $B$ puede ocurrir de $n_2$ formas distintas, entonces el evento $A$ o el evento $B$ puede ocurrir de $n_1 + n_2$ formas distintas.

\subsection{Permutaciones (Importa el orden)}
\indent
Una permutación de $n$ objetos distintos es un arreglo de los $n$ objetos en una secuencia ordenada. El número de permutaciones de $n$ objetos distintos es $n!$ :
\begin{equation*}
    P(n, r) = \frac{n!}{(n - r)!}
\end{equation*}

\subsection{Combinaciones (No importa el orden)}
\indent
Una combinación de $n$ objetos distintos tomados en grupos de $r$ es un subconjunto de $r$ objetos de un conjunto de $n$ objetos distintos. El número de combinaciones de $n$ objetos distintos tomados en grupos de $r$ es:
\begin{equation*}
    C(n, r) = \binom{n}{r} = \frac{n!}{r!(n - r)!} \quad \text{,} \quad r \leq n
\end{equation*}

\begin{figure}[H]
    \centering
    \begin{tabularx}{\textwidth}{|L|L|}
        \hline
        \textbf{Combinaciones} & \textbf{Permutaciones} \\
        \hline
        abc & abc acb bca bac cab cba \\
        bcd & bcd bdc cbd cdb dbc dcb \\
        abd & abd adb bad bda dab dba \\
        acd & acd adc cad cda dac dca \\
        \hline
    \end{tabularx}
    \caption{Tabla ejemplo: Sea a,b,c,d elegir 3 letras a la vez}
\end{figure}

\section{Espacio muestral equiprobable}
\indent
Un espacio muestral $\Omega$ es equiprobable si todos los eventos elementales tienen la misma probabilidad de ocurrir. En este caso, la probabilidad de un evento $A$ es:
\begin{equation*}
    P(A) = \frac{\text{número de maneras de como puede ocurrir el evento A}}{\text{número de maneras de como puede ocurrir el espacio muestral } \Omega}
\end{equation*}

\newpage
\section{Probabilidad Condicional}
La probabilidad condicional de que $A$ ocurra, dado que ocurrió $B$, está dada por:
\begin{equation*}
    P(A / B) = \left\{ \begin{array}{ll}\frac{P(A \cap B)}{P(B)} & \text{,} \quad P(B) > 0 \\ 0 & \text{,} \quad P(B) = 0\end{array}\right.
\end{equation*}
\textbf{Obs:}
\begin{enumerate}
    \item La probabilidad condicional cumple todas las propiedades vistas anteriormente.
    \begin{enumerate}
        \item $P(\Omega/A) = 1$
        \item $P(A \cup B / C) = P(A/C) + P(B/c)$, si $A \cap B = \emptyset$
        \item $P(A^c/B) = 1 - P(A/B) \leftarrow$ Complemento Condicional
    \end{enumerate}
    
    \item De la definición anterior se tiene que la probabilidad condicional de la
    intersección entre $A$ y $B$ puede ser escrita como $P(A \cap B) = P(B) \cdot P(A/B)$, llamada
    \textbf{Regla de la Multiplicación}.
    
    \item Por Simetría, se tiene que si $P(A) > 0$, entonces:
    \begin{equation*}
        P(B / A) = \left\{ \begin{array}{ll}\frac{P(B \cap A)}{P(A)} & \text{,} \quad P(A) > 0 \\ 0 & \text{,} \quad P(A) = 0\end{array}\right.
    \end{equation*}
    * \textbf{Obs 3.1:} La Regla de la Multiplicación se utiliza cuando se seleccionan personas
    de un grupo o artículos de un lote \textbf{sin sustitución}. \\
    * \textbf{Obs 3.2:} Cuando seleccionan personas de grupo o artículos de un lote \textbf{con
    sustitución}, se dice que los eventos son independientes y en ese caso, la probabilidad
    de la intersección de los eventos es el producto de las probabilidades.
    
    \item De las observaciones 2 y 3, se tiene que $ P(B) \cdot P(A/B) = P(A) \cdot P(B/A)$
    
    \item  La probabilidad condicional de $A$ dado $B_1,B_2, ... , B_n$ se escribe de la siguiente forma $P(A/B_1, B_2, ... , B_n)$ y se define $P(A / B_1 \cap B_2 \cap ... \cap B_n)$, cualesquiera sean los eventos $ A, B_1,B_2, ... , B_n$ tales que $P(B_1 \cap B_2 \cap ... \cap B_n ) > 0$. Un desarrollo algebraico simple
    conduce a la Regla de la Multiplicación Generalizada, que está dada por:
    $P(B_1 \cap B_2 \cap ... \cap B_n) = P(B_1) \cdot P(B_2/B_1) \cdot P(B_3 / B_1 \cap B_2) \cdot ... \cdot P(B_n / B_1 \cap B_2 \cap ... \cap B_{n-1})$
    
    \item La definición de probabilidad condicional puede extenderse para incluir
    cualquier número de eventos que se encuentren en el espacio muestral $\Omega$. Por
    ejemplo, puede demostrarse que para tres eventos $A$, $B$ y $C$.
    \begin{equation*}
        P(A / B \cap C) = \frac{P(A \cap B \cap C)}{P(B \cap C)} \quad \text{,} \quad P(B \cap C) > 0
    \end{equation*}
    \begin{equation*}
        P(A \cap B / C) = \frac{P(A \cap B \cap C)}{P(C)} \quad \text{,} \quad P(C) > 0
    \end{equation*}
    
\end{enumerate}

\newpage
\section{def 7 buscar nombre}
\indent
Diremos que los sucesos $B_1, B_2, ... , B_n$, representan una partición del
espacio muestral $\Omega$ si:
\begin{enumerate}
    \item $B_i \cap B_j = \emptyset, \forall i \neq j$
    \item $\displaystyle \bigcup_{i=1}^{n} B_i = \Omega$
    \item $P(B_i) > 0, \quad \forall i, \text{ o bien }, \displaystyle \sum_{i=1}^{n} P(B_i) = 1 $
\end{enumerate}

\section{Teorema de la probabilidad total}
\indent
Sean los eventos $B_1,B_2, ... , B_n$, representan una partición del espacio muestral $\Omega$ y $A$
un evento cualquiera de $\Omega$, entonces:
\begin{equation*}
    P(A) = \displaystyle \sum_{i=1}^{n} P(B_i) \cdot P(A / B_i)
\end{equation*}

\section{Teorema de Bayes}
\indent
Sean los eventos $B_1,B_2, ... , B_n$, representan una partición del espacio muestral $\Omega$ y $A$
un evento cualquiera de $\Omega$, entonces:
\begin{equation*}
    P(B_j/A) = \displaystyle \frac{P(B_j) \cdot P(A/B_j)}{\displaystyle \sum_{i=0}^{n} P(B_i) \cdot P(A/B_i)}
\end{equation*}

\section{Independencia de Eventos}
Dos eventos $A$ y $B$ se dicen estadísticamente independientes si se
verifica que:
\begin{equation*}
    P(A \cap B) = P(A) \cdot P(B)
\end{equation*}
\textbf{Obs:}
\begin{enumerate}
    \item $P(A/B) = P(A); \quad P(A/B) = \frac{P(A \cap B)}{P(B)} = \frac{P(A) \cdot P(B)}{P(B)} = P(A)$
    \item $P(B/A) = P(B)$
    \item $P(A^c/B) = P(A^c)$
    \item $P(B^c/A) = P(B^c)$
    \item $P(A^c \cap B^c) = P(A^c) \cdot P(B^c)$
    \item $P(A^c \cap B) = P(A^c) \cdot P(B)$
    \item $P(A \cap B^c) = P(A) \cdot P(B^c)$
\end{enumerate}

\end{document}