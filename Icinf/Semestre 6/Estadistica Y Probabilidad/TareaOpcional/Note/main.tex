\documentclass{templateNote}
\usepackage{tcolorbox}
\usepackage{amsmath}
\usepackage{amssymb}
\usepackage{pgfplots}
\usepackage{pgf-pie}
\usepackage{tabularx}
\usepackage{hyperref}

\pgfplotsset{compat=1.18}
\newcolumntype{L}{>{\centering\arraybackslash}X}
\begin{document}

\imagenlogoU{img/LogoElNube.png}
\linklogoU{https://github.com/MarceloPazPezo}
\linkDoc{https://github.com/MarceloPazPezo/MyRepo/blob/main/Icinf/Semestre\%206/Estadistica\%20Y\%20Probabilidad/TareaOpcional/TareaOpcional.pdf}
\universidad{Universidad del Bío-Bío}
\titulo{TAREA-OPCIONAL} % Titulo
\asignatura{Estadistica y Probabilidad} % Asignatura
\autor{
    Marcelo \textsc{Paz}
}
\portada
\margenes % Crear márgenes

\section{Enunciado}
\indent
Demostrar que $E(x) = \displaystyle \frac{1}{\lambda}$ y $V(x) = \displaystyle \frac{1}{\lambda^2}$:

Para demostrar que $E(x) = \displaystyle \frac{1}{\lambda}$:
\begin{align*}
    E(x) &= \int_{- \infty}^{\infty} x f(x) dx = \int_{0}^{\infty} x \lambda e^{- \lambda x} dx \\
    &= \lambda \int_{0}^{\infty} x e^{- \lambda x} dx
\end{align*}

Integral por partes, ILATE nos sugiere que $u = x$ y $dv = e^{- \lambda x}$ por lo tanto:
\begin{align*}
    du &= dx \\
    v &= - \frac{1}{\lambda} e^{- \lambda x} \\
    \int{u dv} &= uv - \int v du
\end{align*}

Sustituyendo:
\begin{align*}
    \int_{0}^{\infty} x e^{- \lambda x} dx &= - x \cdot \frac{1}{\lambda} e^{- \lambda x} \Big|_{0}^{\infty} - \int_{0}^{\infty} - \frac{1}{\lambda} e^{- \lambda x} dx \\
    &= - \frac{x}{\lambda e^{\lambda x}} \Big|_{0}^{\infty} + \frac{1}{\lambda} \left[- \frac{1}{\lambda} e^{- \lambda x} \Big|_{0}^{\infty}\right] \\
    &= - \frac{x}{\lambda e^{\lambda x}} \Big|_{0}^{\infty} - \frac{1}{\lambda^2 e^{\lambda x}} \Big|_{0}^{\infty} \\
    &= - \lim_{x \to \infty} \frac{x}{\lambda e^{\lambda x}} - \lim_{x \to \infty} \frac{1}{\lambda^2 e^{\lambda x}} - \left[- \frac{0}{\lambda e^{\lambda 0}} - \frac{1}{\lambda^2 e^{\lambda 0}}\right] \\
    &= - 0 - 0 - \left[- \frac{0}{\lambda} - \frac{1}{\lambda^2}\right] = \frac{1}{\lambda^2}
\end{align*}

\begin{align*}
    E(x) &= \lambda \left[\frac{1}{\lambda^2}\right] && \qquad \text{Remplazando}\\
    &= \displaystyle \frac{1}{\lambda}
\end{align*}

\begin{align*}
    \textbf{Q.E.D.} \qquad E(x) = \displaystyle \frac{1}{\lambda}
\end{align*}

\newpage
Para demostrar que $V(x) = \displaystyle \frac{1}{\lambda^2}$:
\begin{align*}
    V(x) &= E(x^2) - [E(x)]^2 \\
\end{align*}

Es necesario calcular $E(x^2)$:
\begin{align*}
    E(x^2) &= \int_{- \infty}^{\infty} x^2 f(x) dx = \int_{0}^{\infty} x^2 \lambda e^{- \lambda x} dx \\
    &= \lambda \int_{0}^{\infty} x^2 e^{- \lambda x} dx \\
    &= \lambda \left[x^2 \cdot \frac{-1}{\lambda e^{\lambda x}} \Big|_{0}^{\infty} - \int_{0}^{\infty}{\frac{-1}{\lambda e^{\lambda x}}\cdot 2x dx}\right] \\
    &= \lambda \left[\frac{-x^2}{\lambda e^{\lambda x}} \Big|_{0}^{\infty} + \frac{2}{\lambda} \int_{0}^{\infty}{\frac{x}{e^{\lambda x}}dx}\right] \\
    &= \lambda \left[- \lim_{x \to \infty}{\frac{x^2}{\lambda e^{\lambda x}}} + \frac{0^2}{\lambda e^{\lambda 0}} + \frac{2}{\lambda} \left(\frac{-x}{\lambda e^{\lambda x}} \Big|_{0}^{\infty} - \frac{1}{\lambda} \int_{0}^{\infty}{\frac{-1}{\lambda e^{\lambda x}}dx}\right)\right] \\
    &= \lambda \left[- 0 + 0 + \frac{2}{\lambda} \left(- \lim_{x \to \infty}{\frac{x}{\lambda e^{\lambda x}}} + \frac{0}{\lambda e^{\lambda 0}}+ \frac{1}{\lambda^2} \int_{0}^{\infty}{\frac{1}{e^{\lambda x}}dx}\right)\right] \\
    &= \lambda \left[\frac{2}{\lambda} \left(- 0 + 0 + \frac{1}{\lambda^2} \left(\frac{-1}{\lambda e^{\lambda x}} \Big|_{0}^{\infty}\right)\right)\right] \\
    &= \lambda \left[\frac{2}{\lambda} \left(\frac{1}{\lambda^2} \left(- \lim_{x \to \infty}{\frac{1}{\lambda e^{\lambda x}}} + \frac{1}{\lambda e^{\lambda 0}}\right)\right)\right] \\
    &= \lambda \left[\frac{2}{\lambda} \left(\frac{1}{\lambda^2} \left(0 + 1\right)\right)\right] \\
    &= \frac{2}{\lambda^2}
\end{align*}

Luego:
\begin{align*}
    V(x) &= \frac{2}{\lambda^2} - \left[\frac{1}{\lambda}\right]^2 && \qquad \text{Remplazando}\\
    &= \frac{2}{\lambda^2} - \frac{1}{\lambda^2} \\
    &= \frac{1}{\lambda^2}
\end{align*}

\begin{align*}
    \textbf{Q.E.D.} \qquad V(x) = \displaystyle \frac{1}{\lambda^2}
\end{align*}
\end{document}