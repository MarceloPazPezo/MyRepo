\documentclass{templateNote}
\usepackage{tcolorbox}
\usepackage{pgfplots}
\usepackage{amsmath}
\usepackage{amssymb}
\usepackage{systeme}
\usepackage{empheq}
\usepackage{array}
\pgfplotsset{compat=1.18}

\begin{document}

\imagenlogoU{img/LogoElNube.png}
\linklogoU{https://github.com/MarceloPazPezo}
\linkDoc{https://github.com/MarceloPazPezo/MyRepo/blob/main/Icinf/Semestre\%203/Calculo\%20Integral/Fracciones\%20Parciales/Fracciones\%20Parciales.pdf}
\universidad{Universidad del Bío-Bío}
\titulo{Fracciones Parciales} % Titulo
\asignatura{Calculo Integral} % Asignatura
\autor{
    \indent
    Marcelo \textsc{Paz}
}   
\portada
\margenes % Crear márgenes


\section{Teoria}
\indent

Las fracciones parciales permiten descomponer expresiones racionales complejas (en palabras mas simples fraciones irreducibles) en la suma de expresiones más simples. Para esto se deben seguir los siguientes pasos:
\subsection{Division Sintetica (Ruffini)}
\indent
La division sintetica es un metodo para dividir polinomios de la forma $P(x) = a_nx^n + a_{n-1}x^{n-1} + ... + a_1x + a_0$ por un polinomio de la forma $x - r$. Para esto se debe seguir los siguientes pasos:

\[
\begin{array}{ccccc|c}
  x^n & x^{n-1} & ... & x & x^0\\
  &&&&&\\
  a_n & a_{n-1} & ... & a_1 & a_0 \\ 
  & a_n r & ... & (a_{n-1} + a_n r + ...)r & (a_1 + (a_{n-1} + a_n r + ...)r)r & r\\ 
  \cline{1-6}
  a_n & a_{n-1} + a_n r & ... & a_1 + (a_{n-1} + a_n r + ...)r & a_0 + (a_1 + (a_{n-1} + a_n r + ...)r)r\\
  &&&&&\\
  x^{n-1} & x^{n-2} & ... & x^0 & /(x-r) \\ 
\end{array}
\]
Ejemplo:
\begin{align*}
    \frac{x^3 + 2x^2 - 5x + 6}{x-2} &= \quad \text{?} && \text{Aplicamos la division sintetica}
\end{align*}
\[
\begin{array}{cccc|c}
  x^3 & x^2 & x & x^0\\
  &&&&\\
  1 & 2 & -5 & 6 \\ 
    & 2 & 8 & 6 & 2\\ 
  \cline{1-5}
  1 & 4 & 3 & 12 \\
  &&&&\\
  x^2 & x & x^0 & /(x-r)
\end{array}
\]
\begin{align*}
    \frac{x^3 + 2x^2 - 5x + 6}{x-2} &= x^2 + 4x + 3 + \frac{12}{x-2}\\
\end{align*}

\newpage
\subsection{Descomposicion en fracciones simples}
\subsubsection{Pasos generales}
\begin{enumerate}
    \item Comprobar que el grado del numerador es menor que el grado del denominador.
\begin{align*}
    \frac{f(x)}{g(x)} \quad \text{, donde } f(x) \text{ y } g(x) \text{ son polinomios y } \text{grado}(f(x)) < \text{grado}(g(x))
\end{align*}

    \item Factorizar el denominador.
\begin{align*}
    \frac{f(x)}{g(x)} = \frac{f(x)}{(ax+b)(cx^2+d)} \quad \text{, donde } a, b, c, d \in \mathbb{R}
\end{align*}

    \item Escribir la función racional como una suma de fracciones con denominadores lineales y cuadráticos irreducibles.
\begin{align*}
    \frac{f(x)}{(ax+b)(cx^2+d)} = \frac{A}{ax+b} + \frac{Bx+C}{cx^2+d} \quad \text{, donde } A, B \in \mathbb{R}
\end{align*}

    \item Determinar las constantes desconocidas en las fracciones parciales.
\begin{align*}
    \frac{f(x)}{(ax+b)(cx^2+d)} = \frac{i}{ax+b} + \frac{jx + k}{cx^2+d} \quad \text{, donde } i, j, k \in \mathbb{R}
\end{align*}

    \item Escribir la función racional como una suma de fracciones parciales.
\begin{align*}
    \frac{f(x)}{g(x)} = \frac{f(x)}{(ax+b)(cx^2+d)} = \frac{i}{ax+b} + \frac{jx+k}{cx^2+d} \quad \text{, donde } i, j, k \in \mathbb{R}
\end{align*}

\end{enumerate}

Existen 4 casos dentro de las fracciones parciales a la hora de tener factorizados los denominadores, que son:

\newpage
\subsubsection{Caso 1: Factores lineales diferentes e irreducibles $(ax + b)$}
\indent
La fracción parcial toma la forma:
\[
    \frac{A}{ax +b} \quad \text{, donde A} \in \mathbb{R}
\]
Ejemplo:

\begin{align*}
    \frac{3}{(x)(x+1)} &= \frac{A}{x} + \frac{B}{x+1} && \text{Multiplicamos por el denominador} \\
    3 &= A(x+1) + B(x) \\
    \text{Para A tenemos que:} \\
    3 &= A(x+1) + B(x) && \text{Evaluamos en $x=0$} \\
    3 &= A(0+1) + B(0) \\
    3 &= A(1) + B(0) && \text{Resolvemos el sistema de ecuaciones} \\
    3 &= A \\
    A &= 3 \\
    \text{Para B tenemos que:} \\
    3 &= A(x+1) + B(x) && \text{Evaluamos en $x=-1$} \\
    3 &= A(-1+1) + B(-1) \\
    3 &= A(0) + B(-1) \\
    3 &= -B \\
    B &= -3 \\
    \text{Así:} \\
    \frac{3}{(x)(x+1)} &= \frac{3}{x} - \frac{3}{x+1} && \text{Remplazamos los valores de A y B} \\
\end{align*}

\newpage
\subsubsection{Caso 2: Factores lineales repetidos e irreducibles $(ax + b)^n$}
\indent
Cada término en la expansión toma la forma:
\[
    \frac{A_i}{(ax + b)^i} \quad \text{, donde i varía de 1 a n y cada } A_i \text{ es una constante.}
\]
Ejemplo:
\begin{align*}
    \frac{2x}{(x+1)^2} &= \frac{A}{x+1} + \frac{B}{(x+1)^2} && \text{Multiplicamos por el denominador} \\
    2x &= A(x+1) + B && \text{Agrupamos los terminos segun su grado}\\
    2x &= Ax + (A + B) \\
    \text{Por Coeficientes Equivalentes tenemos:}
\end{align*}

\begin{subequations}
    \begin{empheq}[left=\empheqlbrace]{align}
    A x &= 2x \\
    A + B &= 0
    \end{empheq}
\end{subequations}

\begin{align*}
    \text{Para la ecuacion (1a) tenemos que:} \\
    A x &= 2x && \text{Dividimos por $x$} \\
    A &= 2 \\
    \text{Para la ecuacion (1b) tenemos que:} \\
    A + B &= 0 && \text{Remplazamos el valor de A} \\
    2 + B &= 0 && \text{Despejamos B} \\
    B &= -2 \\
    \text{Así:} \\
    \frac{2x}{(x+1)^2} &= \frac{A}{x+1} + \frac{B}{(x+1)^2} && \text{Remplazamos los valores de A y B}\\
    \frac{2x}{(x+1)^2} &= \frac{2}{x+1} - \frac{2}{(x+1)^2}\\
\end{align*}
\newpage
\subsubsection{Caso 3: Denominador cuadrático diferentes e irreducible $(ax^2 + bx + c)$}
\indent
La fracción parcial toma la forma:
\[
    \frac{(Ax + B)}{(ax^2 + bx + c)} \quad \text{, donde A y B son constantes.}
\]

Ejemplo:
\begin{align*}
    \frac{5x + 1}{(x^2 + 1)(x^2 + 3)} &= \frac{Ax + B}{x^2 + 1} + \frac{Cx + D}{x^2 + 3} && \text{Multiplicamos por el denominador} \\
    5x + 1 &= (Ax + B)(x^2 + 3) + (Cx + D)(x^2+1) && \text{Agrupamos segun su grado}\\
    5x + 1 &= (A + C)x^3 + (B + D)x^2 \\ &+ (C + 3A)x + (3B +D) \\
    \text{Por C. E. tenemos:}
\end{align*}

\begin{subequations}
    \begin{empheq}[left=\empheqlbrace]{align}
    0x^3 &= (A + C) x^3 \\
    0x^2 &= (B + D)x^2 \\
    5x &= (C + 3A)x \\
    1 &= (3B + D)
    \end{empheq}
\end{subequations}

\begin{align*}
    \text{Para la ecuacion (2a) tenemos que:} \\
    0 &= A + C && \text{Restamos $C$} \\
    A &= -C \\
    \text{Para la ecuacion (2b) tenemos que:} \\
    0 &= B + D && \text{Restamos $D$} \\
    B &= -D \\
    \text{Para la ecuacion (2c) tenemos que:} \\
    5 &= C + 3A && \text{Remplazamos el valor de A} \\
    5 &= C - 3C \\
    5 &= -2C \\
    C &= \frac{-5}{2} \\
    \text{Así:} \\
    A &= -C \\
    A &= \frac{5}{2} \\
\end{align*}
\begin{align*}
    \text{Para la ecuacion (2d) tenemos que:} \\
    1 &= 3B + D && \text{Remplazamos el valor de B} \\
    1 &= 3(-D) + D \\
    1 &= -2D \\
    D &= \frac{-1}{2} \\
    \text{Así:} \\
    B &= -D \\
    B &= \frac{1}{2} \\
    \text{Luego:} \\
    \frac{5x+1}{(x^2+1)(x^2+3)} &= \frac{Ax+B}{x^2+1} + \frac{Cx + D}{x^2+3} && \text{Remplazamos los valores}\\
    \frac{5x+1}{(x^2+1)(x^2+3)} &= \frac{\frac{5}{2}x+\frac{1}{2}}{x^2+1} + \frac{\frac{-5}{2}x - \frac{1}{2}}{x^2+3}\\
    \frac{5x+1}{(x^2+1)(x^2+3)} &= \frac{1}{2}\left(\frac{5x+1}{x^2+1}\right) - \frac{1}{2}\left(\frac{5x + 1}{x^2+3}\right)\\
\end{align*}

\newpage
\subsubsection{Caso 4: Denominador cuadrático repetidos e irreducible $(ax^2 + bx + c)$}
\indent
La fracción parcial toma la forma:
\[
    \frac{(A_ix + B_i)}{(ax^2 + bx + c)^i} \quad \text{, donde i varía de 1 a n y cada } A_i, B_i \text{ es una constante.}
\]

Ejemplo:
\begin{align*}
    \frac{2x-4}{(x^2 + 1)^2} &= \frac{Ax + B}{x^2 + 1} + \frac{Cx + D}{(x^2 + 1)} && \text{Multiplicamos por el denominador} \\
    2x - 4 &= (Ax + B)(x^2 + 1) + (Cx + D) && \text{Agrupamos segun su grado}\\
    2x - 4 &= Ax^3 + Bx^2 + (C + A)x + (B + D)\\
    \text{Por C. E. tenemos:}
\end{align*}

\begin{subequations}
    \begin{empheq}[left=\empheqlbrace]{align}
    0x^3 &= Ax^3 \\
    0x^2 &= Bx^2 \\
    2x &= (C + A)x \\
    -4 &= (B + D)
    \end{empheq}
\end{subequations}

\begin{align*}
    \text{Para la ecuacion (3a) tenemos que:} \\
    0 &= A\\
    \text{Para la ecuacion (3b) tenemos que:} \\
    0 &= B \\
    \text{Para la ecuacion (3c) tenemos que:} \\
    2 &= C + A && \text{Remplazamos el valor de A} \\
    2 &= C \\
    \text{Para la ecuacion (3d) tenemos que:} \\
    -4 &= B + D && \text{Remplazamos el valor de B} \\
    -4 &= D \\
    \text{Luego:} \\
    \frac{2x-4}{(x^2+1)^2} &= \frac{Ax+B}{x^2+1} + \frac{Cx + D}{(x^2+1)^2} && \text{Remplazamos los valores}\\
    \frac{2x-4}{(x^2+1)^2} &= \frac{0x+0}{x^2+1} + \frac{2x + (-4)}{(x^2+1)^2} \\
    \frac{2x-4}{(x^2+1)^2} &= \frac{2x - 4}{(x^2+1)^2}
\end{align*}
\begin{center}
\[
    \therefore \text{No se puede descomponer}
\]
\end{center}

\end{document}
