\documentclass{templateNote}

\definecolor{Verde}{RGB}{170,239,31}
\definecolor{Morado}{RGB}{127,0,255}
\definecolor{Celeste}{RGB}{0,191,255}
\definecolor{Salmon}{RGB}{255,0,157}
\definecolor{RosaSuave}{RGB}{255,182,193}
\definecolor{Melocoton}{RGB}{255,218,185}
\definecolor{Gris}{RGB}{192,192,192}
\definecolor{Turquesa}{RGB}{64,224,208}
\definecolor{Menta}{RGB}{152,251,152}
\definecolor{AmarilloVainilla}{RGB}{255,255,153}

\newcommand{\newparagraph}{\par\vspace{\baselineskip}\noindent}
\newcommand{\hlcolor}[2]{{\sethlcolor{#1}\hl{#2}}}

\begin{document}

\imagenlogoU{img/LogoElNube.png}
\linklogoU{https://github.com/MarceloPazPezo}
\linkQRDoc{https://github.com/MarceloPazPezo/MyRepo/tree/main/Icinf}
\titulo{Apunte Rapido y re trucho}
\asignatura{GPF}
\autor{
Marcelo Paz
}
\vDoc{1.0.0}
\tipoDoc{Apunte}

% Metadatos del PDF
\title{[\asignatura]-\titulo}
\author{
    \autor
}
\portada
\margenes % Crear márgenes

\section{Ciclo de conversión de efectivo}
\begin{center}
    \includegraphics[width=\textwidth]{img/Screenshot 2024-07-22 225053.png}
\end{center}
\begin{equation*}
    CCC = EPI + PCP - PPP
\end{equation*}
Donde:
\begin{itemize}
    \item CCC: Ciclo de conversión de efectivo
    \item EPI: Edad promedio de inventario
    \item PCP: Periodo de cobranza promedio
    \item PPP: Periodo de pago promedio
\end{itemize}

\newpage
\section{Rotación de efectivo}
\begin{equation*}
    RE = \frac{360}{CCC}
\end{equation*}
Donde:
\begin{itemize}
    \item RE: Rotación de efectivo
    \item 360: Días del año
    \item CCC: Ciclo de conversión de efectivo
\end{itemize}

\section{Efectivo mínimo para operaciones}
\begin{equation*}
    EMO = \frac{DTA}{RE}
\end{equation*}
Donde:
\begin{itemize}
    \item EMO: Efectivo mínimo para operaciones
    \item DTA: Desembolsos totales anuales
    \item RE: Rotación de efectivo
\end{itemize}

*SI aunetna el EMO se traduce como una mayor exigencia minima de operacion.
\section{Estrategias de administración de efectivo}
\begin{enumerate}
    \item \textbf{Retraso de las cuentas por pagar ((+)PPP):} Aumentar el periodo de pago a proveedores, sin afectar la relación con ellos.
    \item \textbf{Administración eficiente de inventario-Producción((-)EPI):} Reducir el tiempo de producción y almacenamiento de inventario.
    \item \textbf{Aceleración de la cobranza de cuentas ((-)PCP):} Reducir el tiempo de cobranza de cuentas por cobrar.
\end{enumerate}

\newpage
\section{Análisis de la información de crédito}
\begin{equation*}
    Z = B_1 \cdot X_1 + B_2 \cdot X_2 + B_3 \cdot X_3 + B_4 \cdot X_4 + B_5 \cdot X_5 + B_6 \cdot X_6
\end{equation*}
Donde:
\begin{itemize}
    \item Z: Índice de predicción
    \item $B_1, B_2, B_3, B_4, B_5, B_6$: Coeficientes
    \item $X_1$: Tamaño de la familia
    \item $X_2$: Ingresos total anual
    \item $X_3$: Historia crediticia
    \item $X_4$: Antecedentes comerciales
    \item $X_5$: Situación de deurdas
    \item $X_6$: Vivienda propia
\end{itemize}

\subsection{Puntos de corte}
\begin{enumerate}
    \item \textbf{Z < 1.81:} Riesgo alto
    \item \textbf{1.81 < Z < 2.99:} Área gris
    \item \textbf{Z > 2.99:} Riesgo bajo
\end{enumerate}

\section{ROE}
\begin{equation*}
    ROE = \frac{Utilidad neta}{Capital contable}
\end{equation*}
\begin{itemize}
    \item Corresponde al porcentaje de utilidad o pérdida obtenido por cada peso que los dueños han invertido en la empresa, incluyendo las utilidades retenidas.
\end{itemize}

\section{ROA}
\begin{equation*}
    ROA = \frac{Utilidad neta}{Activo total}
\end{equation*}
\begin{itemize}
    \item Corresponde al porcentaje de utilidad o pérdida, obtenido por cada peso invertido en activos.
    \item Mientras más alto sea el ROA, más eficiente es la empresa al usar el capital para generar utilidades.
    \item Si el ROA es menor o igual que cero, significa que los inversionistas están perdiendo dinero.
\end{itemize}

\hl{* Si la tasa ROA es mayor que la tasa ROE el accionista está ganando una menor rentabilidad relativa en su inversión, lo que puede deberse a una insuficiente uso del endeudamentio.}

\section{Metodos para estimar la demanda}
\subsection{Cuantitativos}
\begin{itemize}
    \item Correlaciones y regresiones
    \item Alisamiento exponencial
    \begin{figure}[H]
        \centering
        \includegraphics[width=0.8\textwidth]{img/Screenshot 2024-07-22 231413.png}
    \end{figure}

    \item Series cronológicas
    \item Media móvil
    \begin{figure}[H]
        \centering
        \includegraphics[width=0.8\textwidth]{img/Screenshot 2024-07-22 231855.png}
    \end{figure}
    \item Ajuste de curvas
    \item Distribución de frecuencias
\end{itemize}

\subsection{Cualitativos}
\begin{itemize}
    \item Método Delphi
    \item Diseño de escenarios
    \item Tormenta de ideas
    \item Estudio de ciclo de vida
    \item Estudio de mercado
    \item Entrevista
    \item Encuesta
\end{itemize}

\subsection{Formula demanda total del mercado}
\begin{equation*}
    Q = n \cdot p \cdot q
\end{equation*}
Donde:
\begin{itemize}
    \item Q: Demanda total del mercado
    \item n: Cantidad de compradores en el mercado
    \item p: Precio de una unidad promedio
    \item q: Cantidad compradas por el usuario promedio al año
\end{itemize}

\subsection{Participación de mercado}
\begin{center}
    \includegraphics[width=\textwidth]{img/Screenshot 2024-07-22 232214.png}
\end{center}
\section{Decisiones Financieras de corto plazo}
\begin{center}
    \includegraphics[width=\textwidth]{img/Screenshot 2024-07-22 231227.png}
\end{center}
\end{document}