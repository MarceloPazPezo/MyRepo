\documentclass{templateNote}
\begin{document}

\imagenlogoU{img/LogoElNube.png}
\linklogoU{https://github.com/MarceloPazPezo}
\imagenlogoD{img/LogoNGM.png}
\linklogoD{https://github.com/NicoGomezM}
\linkQRDoc{https://github.com/MarceloPazPezo/MyRepo/blob/main/Icinf/Semestre\%207/Gesti\%C3\%B3n\%20Estrat\%C3\%A9gica/Formativa\%201/Formativa-1.pdf}
\titulo{Formativa 1}
\asignatura{Gestión Estratégica}
\autor{
Nicolás Gómez\\
Marcelo Paz
}
\vDoc{1.0.1}

% Metadatos del PDF
\title{[\asignatura]-\titulo}
\author{
    \autor
}
\portada
\margenes % Crear márgenes

\section{Certamen 1 - 2022} 
\subsection*{ITEM I (3 Ptos. c/u)}
Responda las siguientes aseveraciones, marcando con un “V” si la pregunta es verdadera o con una “F” si es 
falsa. Justifique las falsas.

\begin{enumerate}
    \item (\textcolor{green}{V}) El concepto de Estrategia implica un conjunto de decisiones y acciones que se llevan a cabo para lograr un determinado objetivo.
    
    \item (\textcolor{red}{F}) La Estrategia proyectada en una empresa es lo mismo que la Estrategia realizada.\newline
    \textcolor{blue}{
        La estrategia proyectada es el ideal, mientras que la estrategia realizada son las acciones que se llevan a cabo.
    }
    
    \item (\textcolor{green}{V}) Las características de la Administración Estratégica implican un proceso dinámico y participativo, compromiso de los jefes y transversal, entre otras.
    
    \item (\textcolor{green}{V}) El proceso de Administración Estratégica incluye la selección de la misión y objetivos principales, análisis del ambiente competitivo externo y del ambiente operativo interno.\newline
    \textcolor{blue}{
        Además de la formulación, implantación y control de estrategias.
    }
    \item (\textcolor{green}{V}) El triángulo de la estratégica contiene: el análisis estratégico, la elección y la implementación de la estrategia.
    
    \item (\textcolor{green}{V}) Las consecuencias de rivalidad entre los competidores existentes son disminución de precios, aumento de costos y reducción de rentabilidad.
    
    \item (\textcolor{green}{V}/\textcolor{red}{F}?) Los factores críticos de éxito son elementos que ayudan a la empresa a lograr los objetivos. 
    \textcolor{yellow}{
        NO lo pasamos.
    }
    \item (\textcolor{green}{V}) Los requisitos para el desarrollo de una estrategia son: Conocimientos, Capacidad para integración, Imaginación y lógica para elegir entre alternativas, entre otros.
    
    \item (\textcolor{red}{F}) Al hacer la exploración ambiental y los 8 ambientes críticos de una empresa forestal se pueden concluir los mismos análisis que para la UBB.\newline
    \textcolor{blue}{
        A pesar de que comparten los mismos ambientes críticos, las conclusiones optenidas de estos seran diferentes.
    }
    \item (\textcolor{red}{F}) Una Unidad Estratégica de Negocios pertenece a una empresa productiva y tiene sus propios competidores.\newline
    \textcolor{blue}{
        De forma directa no tiene competidores, pero si al pertener a una empresa.
    }
\end{enumerate}

\subsection*{ITEM II (5 Ptos. c/u)}
Explique con un ejemplo, cada uno de los siguientes conceptos: 

\begin{enumerate}
    \item Unidad Estratégica de Negocios\newline
    \textcolor{red}{
        NO ME SUENA
    }
    
    \item Matriz de crecimiento-participación\newline
    \textcolor{blue}{
        En la Universidad del Bío-Bío.
        \begin{itemize}
            \item Estrella: Ingeniería Civil Informática.
            \item Interrogante: Medicina.
            \item Vaca lechera: Ingeniería Civil Industrial.
            \item Perro: Ingeniería en Maderas.
        \end{itemize}
    }
    \item Ciclo de vida del producto\newline
    \textcolor{blue}{
        En la etapa de:
        \begin{itemize}
            \item Introducción tenemos a Temu.
            \item Aceptación tenemos a Edenred.
            \item Madurez tenemos a Discord.
            \item Saturación tenemos a Zoom.
            \item Obsolescencia tenemos a Skype.
        \end{itemize}
    }
    \item Ventaja competitiva\newline
    \textcolor{blue}{
        Ofrecer un servicio de delivery de comida a domicilio, con un tiempo de entrega menor a 30 minutos.
    }
    \item Fuerzas competitivas de Porter
    \textcolor{blue}{
        \begin{itemize}
            \item Poder de negociación de los compradores: Cliente elige comprar en carrito de verduras informal, en vez de supermercado.
            \item Rivalidad entre competidores: WhatsApp añade una nueva función de videollamada, para competir con Telegram.
            \item Amenaza de productos sustitutos: Miel y Stevia como sustitutos del azúcar.
            \item Poder de negociación de los proveedores: Panaderia que vende pan a supermercados.
            \item Amenaza de nuevos competidores: Llegada de Uber a un pueblo con monopolio de taxis.
        \end{itemize}
    }
\end{enumerate}

\newpage
\section{Certamen 1 - 2023} 

\subsection*{ITEM I (3 Ptos. c/u)}
Responda las siguientes aseveraciones, marcando con un “V” si la pregunta es verdadera o con una “F” si es falsa. Justifique las falsas.   
\begin{enumerate}
    \item (\textcolor{red}{F}) El producto Estrella, le indica a una empresa que el mercado lo acepta y por lo tanto, éste le reporta una alta participación de mercado.\newline
    \textcolor{blue}{
        Y además tiene un alto crecimiento en el mercado.
    }
    \item (\textcolor{red}{F}) Cuando un producto está en la etapa de introducción en el mercado, la empresa prácticamente no tiene que invertir en publicidad.\newline
    \textcolor{blue}{
        Falso, ya que es necesario invertir en publicidad para dar a conocer el producto.
    }
    \item (\textcolor{green}{V}/\textcolor{red}{F}) El análisis FODA le sirve a las empresas para determinar el punto máximo de ganancias.\newline
    \textcolor{blue}{
        Falso, dado que el FODA es un análisis a nivel interno y externo que nos permite maximizar los puntos positivos y minimizar los negativos que afectan a la empresa.
    }
    \item (\textcolor{green}{V}) Las áreas funcionales de una empresa, son Finanzas \& Contabilidad, RRHH \& Administración, Producción y Comercialización \& Marketing.
    \item (\textcolor{green}{V}) El proceso de la administración estratégica contiene: selección de misión y objetivos, formulación de estrategias e implementación de la estrategia.\newline
    \textcolor{blue}{
        Además le falta el control de la estrategia, el análisis del entorno externo e interno de la empresa.
    }
    \item (\textcolor{green}{V}) El Modelo 'de' resumen de los elementos de la Dirección Estratégica son: Análisis Estratégico, Elección Estratégica e Implantación de la Estrategia.
    \item (\textcolor{green}{V}) El modelo de negocio de una empresa involucra sólo al entorno interno.
    \item (\textcolor{green}{V}) Un proceso de formulación de estrategias para una empresa, implica considerar el corto plazo.
    \item (\textcolor{red}{F}) El Análisis PESTA le sirve a una empresa para determinar el punto máximo de ganancias.
    \textcolor{blue}{
        Falso, ya que el análisis PESTA es un análisis a nivel externo que nos permite conocer el entorno en el que se encuentra la empresa.
    }
    \item (\textcolor{red}{F}) Una estrategia competitiva implica que la empresa tiene las máximas ganancias económicas del mercado.\newline
    \textcolor{blue}{
        Falso, ya que una estrategia competitiva implica que la empresa tiene una ventaja competitiva sobre sus competidores.
    }
\end{enumerate}

\newpage
\subsection*{ITEM II (5 Ptos. c/u)}
Explique con un ejemplo, cada uno de los siguientes conceptos:
\begin{enumerate}
    \item Estrategia Competitiva
    \textcolor{blue}{
        \begin{itemize}
            \item Lider: Walmart, con sus precios bajos y gran variedad de productos a nivel mundial.
            \item Retador: Unimarc, con sus precios bajos y gran variedad de productos a nivel nacional.
            \item Seguidor: Ganga, con sus precios bajos y con no tanta variedad de productos a nivel regional.
            \item Especialista: Confiteria Turquesa, precio accesible y productos especializados en importación turca.
        \end{itemize}
    }
    \item Segmentación del mercado.
    \item Fuerzas competitivas de Porter.\newline
    \textcolor{blue}{
        Lo mismo que en el certamen anterior.
    }
    \item Entorno de una empresa
    \textcolor{red}{
        Jodimos.
    }
    \item Análisis Estratégico\newline
    \textcolor{blue}{
        Suponiendo que se quiere abrir una empresa de comida rápida en un sector de la ciudad, se debe analizar:
        \begin{itemize}
            \item El entorno: Otros locales de comida rápida, proveedores de insumos, clientes potenciales, zona de ubicación.
            \item Expectativas y propósitos: Vender lo suficiente para tener mas beneficios que costos.
            \item Recursos: Insumos para la comida, personal, local, utensilios.
            \item Competencias: Personal con las capacidades aptas y tenga atención al cliente.
            \item Capacidades: Ofrecer productos y atención de calidad.
        \end{itemize}
    }
\end{enumerate}
\end{document}