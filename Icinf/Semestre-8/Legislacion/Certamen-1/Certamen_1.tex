\documentclass{templateApunte}
\usepackage{tcolorbox}

\definecolor{Violeta}{RGB}{124,0,254}
\definecolor{Verde2}{RGB}{130,254,0} % Complementario
\definecolor{Naranja2}{RGB}{254,124,0} % Triadico
\definecolor{Verde3}{RGB}{0,254,124} % Triadico

\definecolor{Amarillo}{RGB}{249,228,0}
\definecolor{Azul2}{RGB}{0,21,249} % Complementario
\definecolor{Celeste2}{RGB}{0,249,228} % Triadico
\definecolor{Rosa}{RGB}{228,0,249} % Triadico

\definecolor{Naranja}{RGB}{255,175,0}
\definecolor{Azul3}{RGB}{0,80,255} % Complementario
\definecolor{Cian}{RGB}{0,255,175} % Triadico
\definecolor{Violeta2}{RGB}{175,0,255} % Triadico

\definecolor{Rojo}{RGB}{245,0,79}
\definecolor{Verde4}{RGB}{0,245,166} % Complementario
\definecolor{Verde}{RGB}{79,245,0} % Triadico
\definecolor{Azul}{RGB}{0,79,245} % Triadico

\definecolor{Celeste}{RGB}{0,191,255}
\definecolor{Salmon}{RGB}{255,0,157}

\newcommand{\newparagraph}{\par\vspace{\baselineskip}\noindent}
\newcommand{\hlcolor}[2]{{\sethlcolor{#1}\hl{#2}}}

\tcbuselibrary{skins}
\usetikzlibrary{shadings}
\newcounter{counter_comentario}
\newcounter{counter_observacion}
\tcbset{
  base/.style={
    empty,
    frame engine=path,
    colframe=white,
    sharp corners,
    %title={Comentario \thetcbcounter},
    attach boxed title to top left={yshift*=-\tcboxedtitleheight},
    boxed title style={size=minimal, top=4pt, left=4pt},
    coltitle=black,
    fonttitle=\large\it,
  }
}
\newtcolorbox{cP}[5]{%
  base,
  title={#3 \csname the#2\endcsname}, % Título personalizado #2 = nombre contador ; #3 = Título
  drop fuzzy shadow, % Sombra del cuadro de texto
  coltitle=#1,
  borderline west={3pt}{-3pt}{#4}, % Borde Izquierdo #4 = color del borde
  attach boxed title to top left={xshift=-3mm, yshift*=-\tcboxedtitleheight/2},
  boxed title style={right=3pt, bottom=3pt, overlay={
    \draw[draw=#5, fill=#5, line join=round]
      (frame.south west) -- (frame.north west) -- (frame.north east) --
      (frame.south east) -- ++(-2pt, 0) -- ++(-2pt, -4pt) --
      ++(-2pt, 4pt) -- cycle; % #5 = color del fondo
  }}, % Cuadro de titulo
  overlay unbroken={
    \scoped \shade[left color=#5!30!black, right color=#5]
    ([yshift=-0.2pt]title.south west) -- ([xshift=-3pt, yshift=-0.2pt]title.south-|frame.west) -- ++(0, -4pt) -- cycle;
  }, % Sombra de titulo #5 = color del titulo
  before upper={\stepcounter{#2}}
}
\newtcolorbox{cPB}[2]{%
  base,
  coltitle=#1,
  % drop fuzzy shadow, % Sombra del cuadro de texto
  borderline west={3pt}{-3pt}{#2}, % Borde Izquierdo #4 = color del borde
}
\begin{document}
% Seteo de contadores a 1
\setcounter{counter_comentario}{1}
\setcounter{counter_observacion}{1}

\imagenlogoU{img/LogoElNube.png}
\linklogoU{https://github.com/MarceloPazPezo}
\linkQRDoc{https://github.com/MarceloPazPezo/MyRepo/blob/main/Icinf/Semestre-8/Legislacion/Certamen-1/Certamen_1.pdf}
\titulo{Certamen 1}
\asignatura{Legislación}
\autor{
Marcelo Paz
}
\vDoc{1.1.0}
\tipoDoc{Apunte del apunte}

% Metadatos del PDF
\title{[\asignatura]-\titulo}
\author{
    \autor
}
\portada
\margenes % Crear márgenes

\begin{center}
  \begin{cPB}{black}{Rojo}
    \large
    \textbf{Artículo 1, Código Civil:}
    \newline
    La ley es una declaración de la voluntad soberana que, manifestada en la forma prescrita por la Constitución, manda, prohíbe o permite.
  \end{cPB}
\end{center}

\section{Hombre, naturaleza y sociedad}
\hlcolor{Amarillo!80}{La naturaleza y la sociedad constituyen el ambiente inevitable del hombre.}\footnote{Se utiliza hombre para referirse a la humanidad en general.}
Esto quiere decir que el hombre tiene un medio natural y otro medio social.

\subsection{Naturaleza}
En la naturaleza, el hombre nace, crece, se desarrolla y muere, en un clima de cierto orden y disposición de las cosas y fenómenos que componen el universo, y en cuyo origen no ha cabido intervención alguna del hombre.

\subsection{Sociedad}
Por otra parte, se entiende por \textit{sociedad} a una agrupación de individuos que establecen vínculos y relaciones reciprocas e interacciones estables.
\newline
Respecto de los orígenes de la sociedad, \textbf{existen 2 corrientes:}
\begin{itemize}
  \item Planteada por \textbf{Aristóteles}. 
  \begin{itemize} 
    \item Quien sostenía que por defecto el hombre es un animal político.
    \item Decía: \textit{El hombre aislado es un bruto o un Dios.}
    \item \hlcolor{Amarillo!80}{Sostenía que la sociedad es una institución\footnote{Estructura o mecanismo de orden social que regula el comportamiento de los individuos dentro de una comunidad.} natural}, por el hombre individualmente considerado no puede ser autosuficiente, sino que necesita del resto para poder satisfacer sus exigencias físicas, espirituales y desarrollarse como persona.
    \item La familia era la primera comunidad, luego se genera las aldeas, las que luego constituirían la polis o ciudad.
  \end{itemize}

  \newpage
  \item Planteada por \textbf{Rousseau}.
  \begin{itemize}
    \item \hlcolor{Amarillo!80}{Sostenía} que la sociedad no deriva de una necesidad natural del hombre, sino \hlcolor{Amarillo!80}{que ven a la sociedad como una institución convencional.}
    \item La sociedad la forman los hombres a partir de cierto instante, producto de un pacto o contrato social que pone término a un estado previo.
  \end{itemize}
\end{itemize}
Sea que consideremos a la Sociedad como una institución natural o convencional, sin mucha dificultad, podemos advertir en primer lugar que la naturaleza está regida por leyes, las que llamaremos \textit{Leyes de la Naturaleza}; en tanto la sociedad, como medio de interacción entre distintos individuos, no excluye sino que, por el contrario, supone la presencia de conflicto entre los hombres, lo que hizo necesaria la creación de normas de conducta.

\section{Las normas de conducta}
Nuestra interacción en la vida social está regida por una serie de normas de conducta.
En un ejemplo dado por el profesor se hace referencia a:
\begin{itemize}
  \item \textbf{Norma de trato social}, donde se saluda al vecino.
  \item \textbf{Reglas técnicas}, donde al subirse al auto, nos ponemos el cinturón.
  \item \textbf{Contrato de prestación de servicios}, donde al entrar al estacionamiento se saca un ticket.
  \item \textbf{Deber moral}, donde se le da dinero a una persona, la cual pide para comer.
\end{itemize}
Si bien, las distintas normas de conductas nombradas, pueden tener marcadas diferencias entre ellas, decimos igualmente que son normas de conducta, ello, porque existe un núcleo común entre ellas.
\begin{cPB}{black}{Naranja}
  \textbf{Núcleo:} proposiciones o enunciados que tienen por objeto influir en nuestro comportamiento, dirigir nuestra conducta en un sentido o el otro, conseguir que actuemos en una determinada manera que se considere deseable.  
\end{cPB}

\subsection{Clases}
\begin{itemize}
  \item \textbf{Normas de trato social:} Son prescripciones\footnote{Regla, directriz o mandato que indica cómo debe comportarse una persona en una situación específica.}, originadas al interior de un grupo social determinado, que tienden a la realización de ciertos fines como la urbanidad\footnote{Comportamiento acorde con los buenos modales.}, el decoro, la cortesía y otros semejantes, en las que la inobservancia\footnote{Falta de cumplimiento.} de los deberes impuestos se traduce en un tipo difuso de sanción\footnote{Desde la representación de la conducta infringida al infractor, haste el linchamiento (la sanción más grave y primitiva).}.
  \begin{itemize}
    \item Ejemplos de ellas: los saludos, regalos, invitaciones, visitas, etc.
    \item Hay autores como Giorgio del Vecchio que opina que la conducta del hombre sólo puede ser objeto de regulación moral o jurídica\footnote{Todo lo relacionado con el derecho, su ejercicio e interpretación.} (negando que las normas de trato social sean normas autónomas).
  \end{itemize}

  \item \textbf{Normas morales:} Conjunto de principios y normas que establecerían qué es lo que debemos hacer para actuar de un modo moralmente correcto y conseguir el bien a que aspiramos.
  \newline
  Se entiende por ética:
  \begin{enumerate}
    \item Sólo al actuar humano y no califica a éste de correcto o incorrecto, de bueno o malo.
    \item No solo se utiliza para referirse al comportamiento humano, sino que para calificar este comportamiento de bueno o malo.
    \item Para referirnos a uno de los 3 órdenes normativos (los usos sociales, el derecho y la moral). 
  \end{enumerate}

  \item \textbf{Normas jurídicas:} Son aquellas que regulan la conducta de los hombres que viven en sociedad, que provienen de actos de producción normativa ejecutados por la autoridad (legislador) a la que otras normas jurídicas del respectivo ordenamiento le otorgan competencia para tales actos, cuyo cumplimiento, además, está garantizado por la legitima posibilidad del uso de la fuerza socialmente organizada y que, por último, apuntan a la realización de aspiraciones de orden, paz y seguridad.
  \begin{itemize}
    \item A este tipo de normas, junto con la costumbre, la jurisprudencia\footnote{Conjunto de principios o normas generales que emanan de los fallos uniformes dictados por los Tribunales Superiores de Justicia para la aplicación e interpretación de las normas jurídicas.}, y los actos administrativos, se les denomina \textit{derecho}.
  \end{itemize}
  \textbf{Características:}
  \begin{itemize}
    \item \hlcolor{Violeta!50}{Exteriores:} regulan las conductas emitidas o manifestadas por los individuos y no aquellas que permanecen en el pensamiento.
    Existen excepciones, si una persona realiza una conducta prohibida (matar, lesionar, robar) la interioridad del sujeto es relevante (Si lo hizo intencional o por descuido).
    
    \item \hlcolor{Violeta!50}{Heterónomas:} son establecidas por una autoridad normativa que es ajena y está por sobre los individuos que deben obedecerlas.
    \begin{itemize}
      \item La característica heterónoma ocurre en los sistemas de democracias \newline representativas (se eligen a quienes dictaran las normas). 
      \item La característica heterónoma se pierde en los sistemas de democracias directas (es el pueblo que, reunido en asambleas, produce sus propias leyes), pues son los mismo sujetos quienes se someten a su propio querer, sus propias normas.
    \end{itemize}

    \item \hlcolor{Violeta!50}{Bilaterales:} Imponen deberes a un sujeto determinado (sujeto pasivo) y confiere facultades a otro (sujeto activo) para exigir el cumplimiento de tales deberes.
    
    \item \hlcolor{Violeta!50}{Coercibles:} Frente al incumplimiento de una norma jurídica, existe la posibilidad de exigir el cumplimiento por medio de la fuerza legítima (a través de órganos reguladores en su funcionamiento por el ordenamiento jurídico).
  \end{itemize}
\end{itemize}

\newpage
\section{El derecho}
Derecho es una voz polisémica\footnote{Tiene múltiples significados.}, tiene raíz latina que deriva de la voz \textbf{directum}, que significa lo que est\'a conforme a la regla, a la ley o la norma.
\newparagraph
\textbf{Definiciones:}
\begin{itemize}
  \item El Derecho es un sistema u orden normativo e institucional que regula la conducta externa de las personas, inspirando en los postulados de justicia y certeza jurídica, que regula la convivencia social y permite resolver los conflictos de relevancia jurídica, pudiendo imponerse coactivamente.
  \item \textit{Francesco Carnelutti-} define Derecho al conjunto de mandatos jurídicos (preceptos sancionados) que se constituyen para garantizar, dentro de un grupo (Estado), la paz amenazada por los conflictos de intereses entre sus miembros.
  \item Aplicado a un territorio especificó, el Derecho es un conjunto de principios y normas, esto es, el Ordenamiento Jurídico, que regula las relaciones entre los sujetos de una determinada nación, o entre Estados.
  \item El Derecho, o el Ordenamiento Jurídico de un Estado puede ser Escrito(Chile), o puede basarse en la costumbre, a la que llamamos Derecho Consuetudinario.
\end{itemize}
\noindent \textbf{Clasificación:}
\begin{itemize}
  \item \hlcolor{Celeste}{Derecho Público:} Conjunto de normas que rigen la actividad y la organización del Estado, como también las relaciones entre los particulares y el Estado.
  \begin{enumerate}[label=\alph*)-]
    \item \hlcolor{Celeste!50}{Derecho Constitucional}: aquel que establece los principios y reglas que regulan la forma del Estado, la forma de gobierno, los derechos constitucionales, las atribuciones y potestades de los poderes públicos.
    
    \item \hlcolor{Celeste!50}{Derecho Administrativo}: es aquel que regula la Administración Pública o Administración del Estado. Se vincula con el Derecho Constitucional, \newline específicamente con su parte orgánica: es el derecho común de la Administración Pública.
    
    \item \hlcolor{Celeste!50}{Derecho Penal}: es la rama del Derecho Público que regula la potestad punitiva\footnote{buscar} del Estado, integrando un conjunto de disposiciones que tipifican las conductas u omisiones calificadas como delitos, y reglan las condiciones para la aplicación de las penas o de medidas de seguridad o rehabilitación.
    
    \item \hlcolor{Celeste!50}{Derecho Financiero}: rama del Derecho Administrativo relativa al manejo de las finanzas públicas. Abarca tanto normas sobre tributos (Derecho Tributario), as\'i como normas financieras propiamente tales relativas al fato (Derecho Presupuestario).
    
    \item \hlcolor{Celeste!50}{Derecho Económico}: aquel que regla todas las \'areas de la economía que son objeto de regulaciones públicas.
    
    \item \hlcolor{Celeste!50}{Derecho Internacional Público}: es el conjunto de normas jurídicas, costumbres y principios jurídicos que regulan las relaciones entre los Estados.
    
    \item \hlcolor{Celeste!50}{Derecho Procesal}: conjunto de normas que señala los pasos a seguir ante los tribunales de justicia. Esto es, establece los procedimientos judiciales y la manera de proceder de todos quienes intervienen en ellos.
  \end{enumerate}

  \item \hlcolor{Violeta!80}{Derecho Privado:} conjunto de normas que regulan las relaciones de los particulares entre sí o las relaciones de estos con el Estado u otros organismos cuando actúan como simples personas privadas, pudiendo ser el Estado también.
  \textbf{Destacan las siguientes ramas:}
  \begin{enumerate}[label=\alph*)-]
    \item \hlcolor{Violeta!40}{Derecho Civil:} sintéticamente se le describe como el derecho privado común y general, puesto que constituye el núcleo del mismo.
    \begin{itemize}
      \item Es \textbf{común} porque es aplicable a todas las relaciones jurídicas privadas que no estén disciplinadas por otras ramas del derecho, además sus principios y normas generales son aplicables en casos de lagunas o vacíos legales.
      \item Es \textbf{general} porque es aplicable a todas las personas, salvo disposición especial.
      \item Se le conoce como \textit{``el conjunto de principios y preceptos jurídicos sobre la personalidad y las relaciones patrimoniales y de familia.''}
      \item Regula a los individuos desde el nacimiento hasta la muerte, tanto en su vida familiar, como en el ámbito económico y patrimonial.
    \end{itemize}

    \item \hlcolor{Violeta!40}{Derecho Comercial:} es la rama del derecho que rige las relaciones entre particulares, esto es, personas naturales y personas jurídicas, y que se encarga de otorgar un marco jurídico a los actos de comercio.
    
    \item \hlcolor{Violeta!40}{Derecho del Trabajo:} es el conjunto de normas y principios que tienen por objeto regular las relaciones entre empleadores y trabajadores entre ellos, y también con el Estado, fijando reglas que tienen por objeto brindar protección y tutela al trabajo dependiente.
  \end{enumerate}
\end{itemize}

\subsection{Fuentes del Derecho}
La expresión \textit{``fuentes del derecho''}\footnote{Origen o procedencia del derecho.}, tiene varios significados, sin embargo, solo veremos las 2 m\'as utilizadas:
\begin{enumerate}
  \item \hlcolor{Verde3!50}{Fuentes materiales del Derecho:} conjunto de factores de diversa \'indole, ya sea, políticos, económicos, sociales, morales, religiosos, científicos, técnicos, etc., que, presentes en una sociedad en un determinado momento, influyen de manera decisiva o importante en la producción y contenido de las normas jurídicas.
  \newline
  Ejemplos:
  \begin{itemize}
    \item Si en un determinado momento existe inflación, ser\'a necesaria la dictation de una norma que reajuste los sueldos.
    \item Cuando llego la pandemia muchos puestos de trabajo se vieron afectados, por lo que se hizo necesaria la dictation de leyes como la protección al empleo y de teletrabajo.
  \end{itemize}

  \item \hlcolor{Verde3!50}{Fuentes formales del Derecho:} son las formas en que este se manifiesta. 
\end{enumerate}

\subsection{Fuentes formales del Derecho}
\begin{enumerate}
  \item \hlcolor{Salmon!60}{La Ley:} \hyperlink{laLey}{Su contenido ser\'a revisado m\'as adelante, dentro del ordenamiento jur\'idico nacional.}
  
  \item \hlcolor{Salmon!60}{La Costumbre:} repetición constante y uniforme de una determinada forma de conducta por los miembros de una comunidad, unida a la convicción de que responde a un imperativo jurídico o una necesidad jurídica.
  \begin{itemize}
    \item En nuestro país, la Costumbre no constituye derecho salvo en los casos en que la Ley se remite a ella (art. 2 Código Civil).
    \item Es una fuente muy marcada en países con influencia británica.
    \item Diferencias entre la costumbre y la ley:
    \begin{enumerate}
      \item La costumbre surge espontáneamente; la ley, en cambio, lo hace de manera reflexiva.
      \item La costumbre es de formación lenta; la ley, comparativamente, es de formación rápida.
      \item La costumbre no tiene un autor conocido; la ley, por el contrario, tiene un autor de la iniciativa legislativa y el órgano del que emana.
      \item La costumbre tiene un carácter impreciso; la ley posee un mayor grado de precisión.
    \end{enumerate}
  \end{itemize}
  
  \item \hlcolor{Salmon!60}{La Jurisprudencia:} conjunto de principios o normas generales que emanan de los fallos uniformes dictados por los Tribunales Superiores de Justicia para la aplicación e interpretación de las normas jurídicas.
  
  \item \hlcolor{Salmon!60}{La Doctrina:} est\'a constituida por las opiniones, comentarios y en general por los trabajos de los autores relativos a materias de Derecho.
  \begin{itemize}
    \item Actualmente, se sostiene en forma mayoritaria que la doctrina no constituye fuente formal del derecho, sino que constituye fuente material del derecho, y esto, porque las opinión de los especialistas del derecho suele motivar la dictaci\'on de normas jurídicas e influir en su contenido.
  \end{itemize}
  
  \item \hlcolor{Salmon!60}{Los Actos Jur\'idicos de los particulares:} constituye fuente formal de derecho, pero limitada. (pag. 14.)
  % \begin{itemize}
  %   \item Los acontecimientos que ocurren en el mundo y que tienen origen pr\'oximo en la naturaleza o en la acción del hombre,
  %   \begin{itemize}
  %     \item Si producen consecuencias jurídicas se denominan hechos jurídicos.
  %     \item Si no producen consecuencias jurídicas, se denominan hechos simples o materiales.
  %   \end{itemize}    
  %   \item Son hechos de la naturaleza que producen efectos jurídicos, como el nacimiento, la muerte, los casos fortuitos, etc.
  % \end{itemize}
  
  \item \hlcolor{Salmon!60}{Los Actos Jur\'idicos de las personas jur\'idicas:} serie de actos jurídicos, los que tienen un efecto limitado, puesto que solo afectan a los miembros de estas personas o a quienes se encuentran ligados a ella.
  \begin{itemize}
    \item Ejemplo: La Universidad del Bio Bío es una persona jurídica, la cual tienen un estatuto (Norma suprema interna), y asimismo se dictan una serie de reglamentos y decretos, como el reglamento del personal o el reglamento estudiantil.
  \end{itemize}

  \item \hlcolor{Salmon!60}{Los Tratados Internacionales:} son acuerdos suscritos por dos o m\'as miembros de la comunidad internacional, generalmente Estados, con el objetivo de regular sus relaciones y establecer derechos y obligaciones reciprocas.
  \newline
  Pueden ser:
  \begin{itemize}
    \item \textbf{Bilaterales:} cuando son suscritos por 2 sujetos de derecho internacional público.
    \newline
    Ejemplo:
    \begin{itemize}
      \item Tratados limítrofes.
    \end{itemize}

    \item \textbf{Multilaterales:} cuando los sujetos de la relación jurídica son m\'as de dos partes.
    Pueden ser:
    \begin{enumerate}
      \item Generales: son universales.
      \item Restringidos:se limitan a un n\'umero determinado de sujetos.
      \item Abiertos: se puede llegar a ser parte del tratado a pesar de no haber participado en el proceso de formación del mismo.
      \item Cerrados: solo son parte los sujetos originarios.
    \end{enumerate}
    Ejemplos:
    \begin{itemize}
      \item Carta de las Naciones Unidas.
      \item Tratado de Roma 1957.
    \end{itemize}
  \end{itemize}
\end{enumerate}

\newpage
\section{Estado}
Es una comunidad social con una organización política común y un territorio y órganos de gobierno propios, que es soberana e independiente políticamente de otras comunidades.
\begin{itemize}
  \item Andre Hauriou (XIX \& XX): El Estado \textit{``es una agrupación humana fijada en un territorio determinado y en la que existe un orden social, político y jurídico, orientado hacia el bien común, establecido y mantenido por una autoridad dotada de poderes de coerción''}.
  \item Carr\'e de Malherí (1988):  El Estado \textit{``es una comunidad humana, fijada sobre un territorio propio, que posee una organización que resulta para ese grupo, en lo que respeta a las relaciones con sus miembros, una potencia suprema de acción, de mando y coerción''}.
\end{itemize}
Elementos que se consideran necesarios para la existencia de un Estado:
\begin{enumerate}
  \item \hlcolor{Verde!50}{Elemento humano o Población:} la población de un determinado Estado puede estar conformada por nacionales y extranjeros. Además, a las personas se les puede reconocer como ciudadanos, con el objeto de participar en la vida política del país.
  
  \item \hlcolor{Verde!50}{Territorio:} territorio nacional es un concepto geográfico, referido a una porción de la superficie del planeta que pertenece y es administrada por un determinado Estado, es decir, \textbf{donde ejerce su soberanía.}
  \begin{cPB}{black}{Naranja}
    En Chile, est\'a conformado por:
    \begin{itemize}
      \item Espacio terrestre.
      \item Espacio marítimo.
      \item Espacio aéreo.
      \item Territorio ficto o jurídico (Embajadas creo recordar).
    \end{itemize}
  \end{cPB}
  
  \item \hlcolor{Verde!50}{Poder:} un pueblo que habita un territorio determinado requiere de organización para actuar en conjunto. La necesidad de una autoridad para evitar la anarquía. Por lo que surge como titular de este poder el Estado, y no como un individuo determinado.
  Se caracteriza por:
  \begin{itemize}
    \item Originario: su realidad y cualidades son inherentes e inseparables de su existencia.
    \item Autónomo: no existe otro poder de mayor jerarquía.
    \item Independiente del exterior: sus decisiones no dependen de fuera del Estado.
    \item Coactivo: posee el monopolio de la fuerza organizada al interior de la sociedad.
    \item Centralizado: emana de un centro de decisión política al cual la Nación est\'a subordinada.
    \item Delimitado territorialmente: rige en el territorio del Estado y a los habitantes de este.
  \end{itemize}
  El poder político es legal cuando se somete a la Constitución y las leyes. Una cualidad del poder del Estado es la soberanía, en el sentido que dicho poder no admite a ningún otro ni sobre \'el, ni en concurrencia con \'el.
\end{enumerate}

\subsection{Soberanía}
Es el ejercicio del poder político sobre el territorio del Estado.
\newline
Teorías respecto de en quien reside la Soberanía:
\begin{itemize}
  \item Jean Bodin, ella reside en la autoridad que detenta el poder, quien est\'a separado o dividido de la población.
  \item Abata Sieyes, la nación conforma un ente totalmente distinto de los miembros que la componen. Este ente necesita actuar a través de representantes (Principio Representativo). 
  \item Jean Jaques Roussea, sostiene que la soberanía no pertenece a la nación sino al pueblo. En virtud del pacto social los individuos se desprenden de su porción, en favor de la voluntad general.
  \item Jellinek, Gerber \& Daban, la soberanía radica o es detentada por el Estado considerado como un sujeto de derecho público, como una persona jurídica que es independiente del gobernante de turno, del pueblo y de la nación.
  \newline
  Por lo que el Estado se constituye como una corporación fijada sobre un territorio y que se organiza a través del gobierno y que est\'a dotada de poder mando. Los gobernantes no son los representantes del estado, sino que son los órganos del mismo.
\end{itemize}
\begin{cPB}{black}{Violeta}
  En lo que respecta a nuestro país, (art 5, Constitución Política de la República): \textit{``La soberanía reside esencialmente en la Nación. Su ejercicio se realiza por el pueblo a través del plebiscito y de elecciones periódicas y, también, por las autoridades que esta Constitución establece. Ningún sector del pueblo ni individuo alguno puede atribuirse su ejercicio. \newline El ejercicio de la soberanía reconoce como limitación al respecto a los derechos esenciales que emanan de la naturaleza humana...''}.
\end{cPB}

\subsection{Fin del Estado}
El objetivo del Estado es el bien común. El bien común no se refiere al bien de todos -como si todos fueran una unidad real-, sino el conjunto de condiciones apropiadas para que todos -grupos intermedios y personas individuales- alcancen su bien particular.
\begin{itemize}
  \item Según la corriente aristotélica-tomista, el fin objetivo del bien común est\'a dado por la búsqueda del orden, la justicia, el bienestar y la paz externa.
  \item Modernamente, John Rawls, \textit{``el bien común de alcanzar la justicia política para todos los ciudadanos y de preservar la cultura libre que esa justicia hace posible''}.
  \newpage
  \item De acuerdo a la Constitución Política chilena, la finalidad del Estado es estar al servicio de la persona humana promoviendo el bien común. Además, es deber del Estado resguardar la seguridad nacional, dar protección a la población, la familia y su fortalecimiento, promover la integración armónica de todos los sectores de la Nación y asegurar el derecho de las personas a participar con igualdad de oportunidades en la vida nacional (art 1).
\end{itemize}

\subsection{Ciudadanía}
Es la condición que reconoce a una persona una serie de derechos políticos y sociales que le permiten intervenir en la política de un país determinado. No debe confundirse con la Nacionalidad, que es la calidad de que esta dotada una persona por pertenecer a un Estado o nación determinados.
Un ciudadano tiene una serie de derechos y obligaciones que le permiten participar en la vida política.
\begin{itemize}
  \item La Constitucional Política establece que \textit{``Son ciudadanos los chilenos que hayan cumplido dieciocho años de edad y que no hayan sido condenados a una pena aflictiva.''} (art 13, inciso 1).
\end{itemize}

\subsection{Estado de Derecho}
Cuando hablamos del Estado de Derecho, nos estamos refiriendo a la idea que el Estado debe estar sometido al derecho, con el objetivo de evitar el abuso de poder provenientes del absolutismo.
\begin{itemize}
  \item Ha sido definido como \textit{``la situación jurídica de un estado que permite el cumplimiento de los fines del mismo basados en el irrestricto respeto a la persona humana, a la libre generación del poder, en el orden jurídico, de imperio de la Ley aplicable tanto a gobernantes como a gobernados''}.
  \item Podemos conceptualizar al Estado de Derecho como aquella forma de organización del poder político sometida a un conjunto no solo de leyes o normas, sino que también de valores y principios que conforman el ordenamiento jurídico, al que deben sujetarse las personas y el Estado.
\end{itemize}
Características:
\begin{enumerate}
  \item \textbf{Imperio de la Ley:} El principio del imperio de la Ley supone el reconocimiento de la jerarquía normativa, la cual asigna a la Constitución el grado de m\'as alto rango. De lo anterior podemos se\~nalar que el derecho no es s\'olo un conjunto de normas, sino un sistema, dentro del cual la norma inferior no puede contradecir a la de jerarquía superior y las normas jurídicas obligan a todos, incluso al Estado y los órganos que lo integran, por eso se afirma que el estado de derecho es el gobierno de la Ley u no de las personas.
  
  \item \textbf{División de Poderes:} Para Montesquieu la libertad descansa sobre la divisi\'on de poderes, el legislativo, ejecutivo y judicial, separados entre s\'i. Esta constituye la mejor garantía para la libertad de los particulares, ya que los poderes rivalizan, se equilibran, cada poder es un celoso guardián de su respectivo ámbito de competencia.
  \begin{enumerate}
    \item Poder Legislativo: A este poder le corresponde la función legislativa, en virtud del cual, declara y determina las normas que han de regir las relaciones p\'ublicas y privadas de los miembros de la comunidad para alcanzar la armonía necesaria de la vida social, y abre el cause jurídico que conduzca a todos al logro y perfeccionamiento de los fines humanos. \textbf{Es el que hace las leyes, las modifica y derogan las que existen.}
    \begin{itemize}
      \item De acuerdo a la mayoría de las constituciones promulgadas a partir de fines del sigo XVIII, \textbf{la función legislativa se encuentra radicada en un órgano colegiado que se estima representativo de la voluntad general o nacional, por cuanto sus miembros son designados por elección popular}.
      \item En Chile, el poder legislativo se encuentra radicado en el Congreso Nacional, órgano bicameral compuesto por la Cámara de Diputados (155) y el Senado(50).
    \end{itemize}
    
    \item Poder Ejecutivo: es aquel que tiene por objeto el Gobierno y la Administración del Estado.
    \begin{itemize}
      \item En Chile, las funciones de Jefe de Gobierno y Jefe de Estado est\'an integradas en una sola autoridad, que es el Presidente de la República.
    \end{itemize}
    
    \item Poder Judicial: a este le corresponde la función jurisdiccional, administrar justicia, es decir, les corresponde aplicar el derecho al caso concreto, y hacerlo cumplir coactivamente si fuese necesario.
    \begin{itemize}
      \item En Chile, esta función es realizada por los Tribunales de Justicia.
    \end{itemize}
  \end{enumerate}
  \begin{itemize}
    \item Para Montesquieu, \hlcolor{Amarillo!50}{si un mismo órgano estatal ejercía el poder legislativo y el poder ejecutivo, \textbf{ no podía existir la libertad}}, porque este órgano impondría leyes tiránicas para tir\'anicamente ejecutarlas.
    \item También se\~nadaba que \hlcolor{Amarillo!50}{la uni\'on del poder judicial con el legislativo era algo \textbf{condenable}}
    \item Por \'ultimo, sostenía \hlcolor{Amarillo!50}{si el poder judicial estuviese en las mismas manos que el poder ejecutivo, los jueces reunirían el poder de juzgar y ejecutar, no pudiendo ser por ello neutrales.}
  \end{itemize}

  \item \textbf{Legalidad de la Administración:} esto significa que los órganos de la administración deben someter su actuar a la ley, bajo sanción de nulidad del acto y de generar las reponsabilidades que corresponda.
  
  \item \textbf{Derechos Fundamentales:} constituyen un l\'imite al estado, es decir, en su actuar, ni el estado ni sus órganos pueden vulnerar los derechos fundamentales. Se considera parte de estos derechos fundamentales la libertad, la propiedad, y la seguridad del individuo (art 19, Constitución)
\end{enumerate}

\newpage
\section{Constitución Política de la República}
Llamada también Carta Fundamental, es la norma jurídica de mayor jerarquía dentro del ordenamiento jurídico nacional, junto con los tratados internacionales ratificados por Chile y que se encuentren vigentes.
\begin{itemize}
  \item Define el régimen de los derechos y libertades de los ciudadanos.
  \item Delimita los poderes e instituciones de la organización política.
\end{itemize}

\subsection{La Ley} \hypertarget{laLey}{}
La fuente formal por antonomasia\footnote{Se utiliza para indicar que algo o alguien es el más representativo o conocido en su categoría.} del derecho es la ley.
\newline
Sobre la palabra ley existen 3 sentidos, a saber:
\begin{enumerate}
  \item Ley en sentido ampl\'isimo: Se refiere a toda norma jurídica de observancia general, en cuya producción a intervenido uno o m\'as órganos del Estado, en este sentido la palabra ley pasa a ser sinónimo de legislación.
  \item Ley en sentido amplio: Es toda norma observancia general en cuya producción intervienen, conjuntamente, el poder ejecutivo con el legislativo.
  \item Ley en sentido restringido: La palabra ley se identifica con aquella norma emanada del poder legislativo y que cumple con el procedimiento establecido en la Constitución para nacer a la vida del derecho.
\end{enumerate}

\subsection{Definición Código Civil}
El artículo 1 del Código Civil dispone que \textit{``La ley es una declaración de la voluntad soberana que, manifestada en la forma prescrita por la Constitución, manda, prohíbe o permite''}.
A partir de esta definición se desprende que existen 3 tipos de leyes:
\begin{enumerate}
  \item Ley imperativa: es aquella que manda a hacer algo, o a cumplir con una serie de requisitos para que un acto o contrato tenga eficacia jurídica. (Voto obligatorio)
  \item Ley prohibitiva: es aquella que manda a no hacer algo, que impide una determinada conducta bajo todo respecto o consideración. (Prohibido fumar)
  \item Ley permisiva: es aquella que permite la realización de un acto determinado, o que reconocen a un sujeto una determinada facultad. (Ley de Transparencia)
\end{enumerate}

\newpage
\subsection{Formación de la Ley}
Son las etapas por las que debe pasar un proyecto de ley para convertirse finalmente en una ley de la república.
\newline
\textbf{Fases del proceso de formación de la ley:}
\begin{enumerate}
  \item Iniciativa: consiste en el acto por medio del cual un proyecto de ley se somete a la consideración del poder legislativo.
  
  \item Discusión: es el estudio y análisis del proyecto de ley que llevan a cabo los parlamentarios sobre el mismo, incluyendo el pertinente debate a que da lugar.
  
  \item Aprobación: luego de agotada la discusión cada Cámara presta su conformidad al proyecto. Se lleva a cabo por medio de las respectivas votaciones, cuyos quorum\footnote{Número mínimo de miembros que deben estar presentes en una asamblea o reunión para que las decisiones que se tomen sean válidas.} varían dependiendo de la clase de ley de que se trate.
  
  \item Sanción: es el acto por el cual el Presidente de la República da su aprobación al proyecto previamente aprobado por el congreso.
  \begin{itemize}
    \item Aprobación expresa: cuando el Presidente promulga el proyecto de ley sin m\'as trámites.
    
    \item Aprobación tácita: cuando transcurren 30 días desde la recepción del proyecto de ley por parte del Presidente de la República sin que preste su aprobación ni lo devuelva al congreso.
    \begin{itemize}
      \item El Presidente tiene el veto. Si el Presidente desaprueba el proyecto, deberá devolverlo a la Cámara de origen con sus observaciones convenientes, dentro del término de 30 días.
    \end{itemize}
  \end{itemize}  
  
  \item Promulgación: es el acto por el cual el Presidente de la República da constancia de la existencia de la ley, fija su texto y ordena cumplirla, a través de un decreto promulgatorio.
  
  \item Publicación: consiste en el acto mediante el cual se comunica o informa el contenido de la ley, lo cual se hace mediante la inserción del texto de la ley en el Diario Oficial.
\end{enumerate}

\subsection{Decreto con Fuerza de Ley}
Son normas emanadas del Presidente de la República sobre materia propia de ley, que \'este dicta en virtud de una delegación de facultades legislativas por parte del Congreso Nacional.

\subsection{Decreto Ley}
Son normas emanadas del poder ejecutivo y tratan sobre materias propias de ley, pero que, a diferencia de lo que ocurre con los Decretos con Fuerza de Ley, se dictan sin que haya una delegación de facultades por parte del Congreso.

\subsection{Decretos y Potestad Reglamentaria}
\begin{itemize}
  \item Potestad Reglamentaria: es la facultad que la Constitución y las leyes otorgan al Presidente de la República y a otras autoridades administrativas, para producir normas jurídicas sobre materias de interés público, pero cuya regulación no este reservada a la ley, y tiene por objeto facilitar la aplicación de otras leyes y al mejor y m\'as eficaz cumplimiento de las funciones del gobierno y de la administración.
  \item Esta potestad se ejerce a través de los Decretos:
  \begin{enumerate}
    \item Reglamentarios: son aquellos dictados por el Presidente de la República y que tienen por objeto contribuir a la adecuada ejecución de las leyes.
    \item Simples decretos: son aquellos los cuales dicta la autoridad ejecutiva para llevar a cabo actividades de gobierno y administración, y que solo afectan a una cantidad especifica de personas, es decir, a diferencia de la ley, no est\'an establecidas para la generalidad de las persona.
    \item Instrucciones: son comunicaciones que emiten los jefes superiores de la \newline administración y que est\'an dirigidas a sus subordinados, y tienen por finalidad indicar los criterios y acciones que se deben emplear para la mejor aplicación de la ley o un reglamento, as\'i como las medidas a adoptar para el mejor funcionamiento de un servicio público.
    \item Ordenanzas: son norma emanadas de la autoridad alcaldicia, y que es general y obligatoria para la comunidad dentro del territorio de la comuna.
  \end{enumerate}
\end{itemize}
\end{document}
