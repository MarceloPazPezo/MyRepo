\documentclass{templateApunte}
\usepackage{tcolorbox}

\definecolor{Violeta}{RGB}{124,0,254}
\definecolor{Verde2}{RGB}{130,254,0} % Complementario
\definecolor{Naranja2}{RGB}{254,124,0} % Triadico
\definecolor{Verde3}{RGB}{0,254,124} % Triadico

\definecolor{Amarillo}{RGB}{249,228,0}
\definecolor{Azul2}{RGB}{0,21,249} % Complementario
\definecolor{Celeste2}{RGB}{0,249,228} % Triadico
\definecolor{Rosa}{RGB}{228,0,249} % Triadico

\definecolor{Naranja}{RGB}{255,175,0}
\definecolor{Azul3}{RGB}{0,80,255} % Complementario
\definecolor{Cian}{RGB}{0,255,175} % Triadico
\definecolor{Violeta2}{RGB}{175,0,255} % Triadico

\definecolor{Rojo}{RGB}{245,0,79}
\definecolor{Verde4}{RGB}{0,245,166} % Complementario
\definecolor{Verde}{RGB}{79,245,0} % Triadico
\definecolor{Azul}{RGB}{0,79,245} % Triadico

\definecolor{Celeste}{RGB}{0,191,255}
\definecolor{Salmon}{RGB}{255,0,157}

\newcommand{\newparagraph}{\par\vspace{\baselineskip}\noindent}
\newcommand{\hlcolor}[2]{{\sethlcolor{#1}\hl{#2}}}

\tcbuselibrary{skins}
\usetikzlibrary{shadings}
\newcounter{counter_comentario}
\newcounter{counter_observacion}
\tcbset{
  base/.style={
    empty,
    frame engine=path,
    colframe=white,
    sharp corners,
    %title={Comentario \thetcbcounter},
    attach boxed title to top left={yshift*=-\tcboxedtitleheight},
    boxed title style={size=minimal, top=4pt, left=4pt},
    coltitle=black,
    fonttitle=\large\it,
  }
}
\newtcolorbox{cP}[5]{%
  base,
  title={#3 \csname the#2\endcsname}, % Título personalizado #2 = nombre contador ; #3 = Título
  drop fuzzy shadow, % Sombra del cuadro de texto
  coltitle=#1,
  borderline west={3pt}{-3pt}{#4}, % Borde Izquierdo #4 = color del borde
  attach boxed title to top left={xshift=-3mm, yshift*=-\tcboxedtitleheight/2},
  boxed title style={right=3pt, bottom=3pt, overlay={
    \draw[draw=#5, fill=#5, line join=round]
      (frame.south west) -- (frame.north west) -- (frame.north east) --
      (frame.south east) -- ++(-2pt, 0) -- ++(-2pt, -4pt) --
      ++(-2pt, 4pt) -- cycle; % #5 = color del fondo
  }}, % Cuadro de titulo
  overlay unbroken={
    \scoped \shade[left color=#5!30!black, right color=#5]
    ([yshift=-0.2pt]title.south west) -- ([xshift=-3pt, yshift=-0.2pt]title.south-|frame.west) -- ++(0, -4pt) -- cycle;
  }, % Sombra de titulo #5 = color del titulo
  before upper={\stepcounter{#2}}
}
\newtcolorbox{cPB}[2]{%
  base,
  coltitle=#1,
  % drop fuzzy shadow, % Sombra del cuadro de texto
  borderline west={3pt}{-3pt}{#2}, % Borde Izquierdo #4 = color del borde
}
\begin{document}
% Seteo de contadores a 1
\setcounter{counter_comentario}{1}
\setcounter{counter_observacion}{1}

\imagenlogoU{img/LogoElNube.png}
\linklogoU{https://github.com/MarceloPazPezo}
\linkQRDoc{https://github.com/MarceloPazPezo/MyRepo/tree/main/Icinf/Semestre\%208/Ingenieria-de-Software/Certamen-1/Certamen_1.pdf}
\titulo{Certamen 1}
\asignatura{Ingeniería de Software}
\autor{
Marcelo Paz
}
\vDoc{1.0.1}
\tipoDoc{Apunte}

% Metadatos del PDF
\title{[\asignatura]-\titulo}
\author{
    \autor
}
\portada
\margenes % Crear márgenes

\section{Recopilación de información}
\begin{enumerate}
  \item \hyperlink{entrevista}{Entrevistas.} 
  \item \hyperlink{grupos_focales}{Grupos focales.}
  \item \hyperlink{talleres_facilitados}{Talleres facilitados.}
  \item \hyperlink{tecnicas_grupales_de_creatividad}{Técnicas grupales de creatividad.}
  \item \hyperlink{tecnicas_grupales_de_toma_de_decisiones}{Técnicas grupales de toma de decisiones.}
  \item \hyperlink{cuestionarios_y_encuestas}{Cuestionarios y encuestas.}
  \item \hyperlink{observacion}{Observación.}
  \item \hyperlink{prototipos}{Prototipos.}
  \item \hyperlink{estudios_comparativos}{Estudios comparativos.}
  \item \hyperlink{diagramas_de_contexto}{Diagramas de contexto.}
  \item \hyperlink{analisis_de_documentos}{Análisis de documentos.}
\end{enumerate}

\subsection{Entrevista}\hypertarget{entrevista}{}
Es una \hlcolor{Amarillo}{manera formal o informal de obtener información} de los interesados, a través de un \hlcolor{Amarillo}{diálogo directo} con ellos. Se lleva a cabo habitualmente \hlcolor{Amarillo}{realizando preguntas (preparadas o espontáneas) registrando las respuestas.}
Se recomienda entrevistar a participantes con experiencia en el proyecto, a patrocinadores y otros ejecutivos, así como a expertos en la materia.
\begin{itemize}
  \item \textbf{Ventajas:}
  \begin{itemize}
    \item Permite obtener información detallada/confidencial.
    \item Preguntas espontáneas permiten obtener información adicional no prevista.
  \end{itemize}
  \item \textbf{Desventajas:}
  \begin{itemize}
    \item Necesario alguien habilidoso y buen orador.
    \item Puede ser subjetivo y sesgado.
    \item Difícil de analizar.
  \end{itemize}
\end{itemize}

\newpage
\subsection{Grupos focales}\hypertarget{grupos_focales}{}
Reúnen a interesados y expertos en la materia, previamente seleccionados, a fin de conocer sus expectativas y actitudes con respecto a un producto, servicio o resultado propuesto. Un moderador capacitado guía al grupo a través de una discusión interactiva diseñada para ser más coloquial que una entrevista individual.
\begin{itemize}
  \item \textbf{Ventajas:}
  \begin{itemize}
    \item Permite obtener una mayor cantidad de información en un tiempo más corto.
  \end{itemize}
  \item \textbf{Desventajas:}
  \begin{itemize}
    \item Igual que la entrevista, se necesita a alguien habilidoso y buen orador, puede ser subjetivo y sesgado.
    \item Además de que al ser un grupo, dificulta la obtención completa de información.
  \end{itemize}
\end{itemize}

\subsection{Talleres facilitados}\hypertarget{talleres_facilitados}{}
Son \hlcolor{Amarillo}{sesiones focalizadas} que reúnen a los interesados clave para \hlcolor{Amarillo}{definir requisitos del producto.}

\subsection{Técnicas grupales de creatividad}\hypertarget{tecnicas_grupales_de_creatividad}{}
Son \hlcolor{Amarillo}{actividades en grupo que permiten identificar los requisitos del proyecto y del producto.}
\begin{itemize}
  \item \textbf{Tormenta de ideas:}
  \item \textbf{Técnicas de grupo nominal:}
  \item \textbf{Mapa conceptual/mental:}
  \item \textbf{Diagrama de afinidad:}
  \item \textbf{Análisis de decisiones con múltiples criterios:}
\end{itemize}

\subsection{Técnicas grupales de toma de decisiones}\hypertarget{tecnicas_grupales_de_toma_de_decisiones}{}
Es un proceso de evaluación que maneja múltiples alternativas, con un resultado esperado en forma de acciones futuras. Estas técnicas se pueden utilizar para generar, clasificar y asignar prioridades a los requisitos del producto.
\begin{itemize}
  \item \textbf{Unanimidad:}
  \item \textbf{Mayoría:}
  \item \textbf{Pluralidad:}
  \item \textbf{Dictador:}
\end{itemize}

\newpage
\subsection{Cuestionarios y encuestas}\hypertarget{cuestionarios_y_encuestas}{}
Conjunto de preguntas escritas, diseñadas para recoger información rápidamente de un gran número de encuestados.
\begin{itemize}
  \item \textbf{Ventajas:}
  \begin{itemize}
    \item Puede ser anónimo.
    \item Fácil de analizar.
    \item Puede ser enviado a un gran número de personas.
  \end{itemize}
  \item \textbf{Desventajas:}
  \begin{itemize}
    \item Necesario alguien habilidoso en la redacción de preguntas.
  \end{itemize}
\end{itemize}

\subsection{Observación}\hypertarget{observacion}{}
Proporcionan una manera directa de ver a las personas en su ambiente, y el modo en que realizan sus trabajos o tareas y ejecutan los procesos.
\begin{itemize}
  \item \textbf{Ventajas:}
  \begin{itemize}
    \item Útil para entender procesos detallados.
  \end{itemize}
  \item \textbf{Desventajas:}
  \begin{itemize}
    \item Necesario alguien habilidoso en la observación.
  \end{itemize}
\end{itemize}

\subsection{Prototipos}\hypertarget{prototipos}{}
Método para obtener una realimentación rápida en relación con los requisitos, mientras proporciona un modelo operativo del producto esperado antes de construirlo.
\begin{itemize}
  \item \textbf{Ventajas:}
  \begin{itemize}
    \item Rápida realimentación de requisitos.
    \item Permite a los interesados ver y sentir el producto.
  \end{itemize}
  \item \textbf{Desventajas:}
  \begin{itemize}
    \item Costoso.
    \item Es necesario tener al cliente/interesado presente.
  \end{itemize}
\end{itemize}

\subsection{Estudios comparativos}\hypertarget{estudios_comparativos}{}
Implica cotejar las prácticas reales o planificadas, tales como procesos y operaciones, con las de aquellas organizaciones comparables a fin de identificar las mejores prácticas, generar ideas de mejora y proporcionar una base para medir el desempeño.

\subsection{Diagramas de contexto}\hypertarget{diagramas_de_contexto}{}
Ejemplo de un modelo de alcance. El diagrama representa visualmente las entradas de negocio, actores que proporcionana entradas y las salidas del sistema de negocio y los actores que reciben las salidas.

\subsection{Análisis de documentos}\hypertarget{analisis_de_documentos}{}
Se utiliza para obtener requisitos mediante el examen de la documentación existente y la idenficación de la información relevante para los requisitos.
Algunos ejemplos de documentos son:
\begin{itemize}
  \item Planes de negocio.
  \item Literatura de mercadeo.
  \item Acuerdos.
  \item Solicitudes de propuesta.
  \item Flujo de procesos actuales.
  \item Modelo lógico de datos.
  \item Repositorio de reglas de negocio.
  \item Documentación de software de la aplicación.
  \item Registros de problemas/indicentes.
  \item Políticas.
  \item Procedimientos.
  \item Documentación normativa (leyes, códigos u ordenanzas, etc.).
\end{itemize}
\end{document}