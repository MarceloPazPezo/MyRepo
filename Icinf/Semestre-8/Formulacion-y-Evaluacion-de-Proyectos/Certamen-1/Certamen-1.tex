\documentclass{templateNote}
\usepackage{tcolorbox}
\usepackage{tabularx}

\definecolor{Violeta}{RGB}{124,0,254}
\definecolor{Verde2}{RGB}{130,254,0} % Complementario
\definecolor{Naranja2}{RGB}{254,124,0} % Triadico
\definecolor{Verde3}{RGB}{0,254,124} % Triadico

\definecolor{Amarillo}{RGB}{249,228,0}
\definecolor{Azul2}{RGB}{0,21,249} % Complementario
\definecolor{Celeste2}{RGB}{0,249,228} % Triadico
\definecolor{Rosa}{RGB}{228,0,249} % Triadico

\definecolor{Naranja}{RGB}{255,175,0}
\definecolor{Azul3}{RGB}{0,80,255} % Complementario
\definecolor{Cian}{RGB}{0,255,175} % Triadico
\definecolor{Violeta2}{RGB}{175,0,255} % Triadico

\definecolor{Rojo}{RGB}{245,0,79}
\definecolor{Verde4}{RGB}{0,245,166} % Complementario
\definecolor{Verde}{RGB}{79,245,0} % Triadico
\definecolor{Azul}{RGB}{0,79,245} % Triadico

\definecolor{Celeste}{RGB}{0,191,255}
\definecolor{Salmon}{RGB}{255,0,157}

\newcommand{\newparagraph}{\par\vspace{\baselineskip}\noindent}
\newcommand{\hlcolor}[2]{{\sethlcolor{#1}\hl{#2}}}
\newcommand{\comillas}[1]{``#1''}
\tcbuselibrary{skins}
\usetikzlibrary{shadings}
\tcbset{
  base/.style={
    empty,
    frame engine=path,
    colframe=white,
    sharp corners,
    %title={Comentario \thetcbcounter},
    attach boxed title to top left={yshift*=-\tcboxedtitleheight},
    boxed title style={size=minimal, top=4pt, left=4pt},
    coltitle=black,
    fonttitle=\large\it,
  }
}
\newtcolorbox{CuadroPersonalizado}[4]{%
  base,
  title={#2}, % Título personalizado #2 = nombre contador ; #3 = Título
  drop fuzzy shadow, % Sombra del cuadro de texto
  coltitle=#1,
  borderline west={3pt}{-3pt}{#3}, % Borde Izquierdo #4 = color del borde
  attach boxed title to top left={xshift=-3mm, yshift*=-\tcboxedtitleheight/2},
  boxed title style={right=3pt, bottom=3pt, overlay={
    \draw[draw=#4, fill=#4, line join=round]
      (frame.south west) -- (frame.north west) -- (frame.north east) --
      (frame.south east) -- ++(-2pt, 0) -- ++(-2pt, -4pt) --
      ++(-2pt, 4pt) -- cycle; % #5 = color del fondo
  }}, % Cuadro de titulo
  overlay unbroken={
    \scoped \shade[left color=#4!30!black, right color=#4]
    ([yshift=-0.2pt]title.south west) -- ([xshift=-3pt, yshift=-0.2pt]title.south-|frame.west) -- ++(0, -4pt) -- cycle;
  }, % Sombra de titulo #5 = color del titulo
}
\begin{document}

\imagenlogoU{img/LogoElNube.png}
\linklogoU{https://github.com/MarceloPazPezo}
\linkQRDoc{https://github.com/MarceloPazPezo/MyRepo/tree/main/Icinf}
\titulo{Certamen 1}
\asignatura{Formulación y Evaluación de Proyectos}
\autor{
Marcelo Paz
}
\vDoc{1.0.0}
\tipoDoc{Apunte}

% Metadatos del PDF
\title{[\asignatura]-\titulo}
\author{
    \autor
}
\portada
\margenes % Crear márgenes

\section{¿Qué es un proyecto?}
\textbf{Definición:} \hlcolor{Amarillo!50}{Es una concepción formal y fundamentada de una idea que pretende materializarse.} Es una suerte de \textit{''borrador de la realidad futura''} en la que se incorporan las previsiones acerca de todos los elementos del entorno, sobre los cuales influirá y actuará la idea que pretendemos materializar.
\newline
Se habla de:\newparagraph

\begin{minipage}[t]{0.45\textwidth}
  \begin{itemize}
    \item Proyecto de Ley.
    \item Proyecto de País.
    \item Proyecto Deportivo.
    \item Proyecto inmobiliario.
  \end{itemize}
\end{minipage}
\begin{minipage}[t]{0.45\textwidth}
  \begin{itemize}
    \item Proyecto de ampliación.
    \item Proyecto de empresa.
    \item Proyecto de modernización.
  \end{itemize}
\end{minipage}

\section{Tipos de proyecto}
\begin{itemize}
  \item \textbf{Proyecto de Negocio o Proyecto de empresa (Business Plan).}
  \item \textbf{Proyecto de Modernización.}
  \item \textbf{Proyecto de Amplificación.}
\end{itemize}

\section{Etapas de un proyecto}
\begin{itemize}
  \item \textbf{Preparación y Evaluación de Proyectos.}
  \begin{itemize}
    \item Idea.
    \item Perfil.
    \item Prefactibilidad.
    \item Factibilidad.
  \end{itemize}

  \item \textbf{Administración y Control de Proyectos.}
  \begin{itemize}
    \item Diseño.
    \item Ejecución.
    \item Operación.
  \end{itemize}
\end{itemize}

\newpage
\section{Estudios del Proyecto}
\begin{itemize}
  \item \hyperlink{EM}{\textbf{Estudio de Mercado}.}
  \item \hyperlink{ET}{\textbf{Estudio Técnico}.}
  \item \hyperlink{EO}{\textbf{Estudio Organizacional}.}
  \item \hyperlink{EL}{\textbf{Estudio Legal}.}
  \item \hyperlink{EEF}{\textbf{Estudio Económico Financiero}.}
\end{itemize}

\subsection{Estudio de mercado}\hypertarget{EM}{}
\textbf{DEF.} \hlcolor{Amarillo!50}{Es la recopilación, procesamiento y análisis de información respecto del diseño, producción, comercialización y consumo de bienes y servicios.}
\begin{itemize}
  \item \textbf{Mostrara:}
  \begin{itemize}
    \item Quiénes son los destinatarios del proyecto.
    \item Las necesidades que el proyecto abordará y que justifican su implementación, describiendo el diseño de los productos y servicios generados y cómo satisfacen las necesidades detectadas.
  \end{itemize}
  \item  \textbf{Cuantificará:}
  \begin{itemize}
    \item Los beneficios que el proyecto generará a los destinatarios del proyecto.
  \end{itemize}
  \item \textbf{Describe:}
  \begin{itemize}
    \item Las condiciones del entorno en que el proyecto se implementa, condiciones económicas generales, la competencia, los proveedores de insumos, y distribuidores.
  \end{itemize}
\end{itemize}

\begin{CuadroPersonalizado}{black}{Comentario del profesor}{Verde2!90!black}{Verde2!90!black}
  \begin{itemize}
    \item El cliente es un beneficiario de un bien o servicio.
    \item El estudio del mercado le da el contexto y la información para el nacimiento de un proyecto.
    \item El mercado \textbf{NO} es algo abstracto.
    \item Es importante estudiar una amplia gama de variables, como la competencia, la demanda, la oferta, los precios, los canales de distribución, entre otros.
    \item Un ejemplo de ventaja competitiva es con \textbf{LIDER} que tiene productos con un bajo precio y del \textbf{JUMBO} que tiene productos de calidad/variedad.
  \end{itemize}
\end{CuadroPersonalizado}

\begin{CuadroPersonalizado}{black}{Extra: 'Referencias a uno de los libros'}{Naranja!90!black}{Naranja!90!black}
  \begin{itemize}
    \item Análisis del Entorno del Mercado.
    \item Análisis del Cliente
    \begin{itemize}
      \item Análisis del Cliente y sus necesidades.
      \item Determinación del mercado objetivo.
      \item Definición de Estrategia de Posicionamiento.
      \item Determinación de la ventaja competitiva.
    \end{itemize}
    \item Análisis del Producto.
    \begin{itemize}
      \item Estudio de atributos de los productos.
    \end{itemize}
    \item Análisis del Precio.
    \item Estimación de la Demanda.
    \begin{itemize}
      \item Medición de mercados potenciales. Principales segmentos. Grado de penetración.
      \item Pronóstivo de Demanda Industrial.
      \item Estimación de la participación de mercado de la empresa.
      \item Análisis de las ventas de la empresa.
    \end{itemize}
  \end{itemize}
\end{CuadroPersonalizado}

\begin{CuadroPersonalizado}{black}{Ejercicio}{Celeste!90!black}{Celeste!90!black}
  \textbf{Caso:} Antes se realizaba el control de inventario utilizando lapiz y papel.
  \newline
  \textbf{Análisis:}
  \begin{itemize}
    \item El hacer el control manual requeria muchas HH.
    \item Esto genera un costo muy alto e innecesario pues hace perder ganancias (sobre todo en el caso de las pymes).
  \end{itemize}
  \textbf{Solución:} Se implementa un sistema de control de inventario digital/automatico, el cual reduce ese costo innecesario.
\end{CuadroPersonalizado}

\newpage
\subsubsection{Análisis del producto}
\textbf{DEF.} \hlcolor{Amarillo!50}{Es todo aquello que por su adquisición, uso o consumo puede satisfacer un deseo o una necesidad.} Un producto puede ser un objeto material, una persona, una idea, un lugar, una experiencia. \newline
Todo producto es un conjunto de atributos o características que tienen una utilidad o función base, y varias utilidades secundarias de distinto tipo que complementan la utilidad básica (estética, imagen, prestigio, conveniencia).

\begin{CuadroPersonalizado}{black}{Comentario del profesor}{Verde2!90!black}{Verde2!90!black}
  \begin{itemize}
    \item Cuando el caso de uso esta mal formulado, el producto falla o no cumple con las expectativas.
  \end{itemize}
\end{CuadroPersonalizado}
\textbf{Niveles:}
\begin{enumerate}
  \item \textbf{Nivel Básico:} Es la función principal del producto.
  \item \textbf{Nivel Esperado:} Son las características que el cliente espera del producto.
  \item \textbf{Nivel Aumentado:} Son las características que el cliente no espera pero que le aportan valor. 'Garantía'.
  \begin{itemize}
    \item Garantias.
    \item Transporte
    \item Instalacion
  \end{itemize}
\end{enumerate}
\begin{CuadroPersonalizado}{black}{Ejemplo}{Celeste!90!black}{Celeste!90!black}
  \textbf{Caso:} Una de las tendencias demográficas más marcadas hoy en día es el crecimiento de la población de la llamada \textit{\comillas{tercera edad}}. 

  \textbf{Se pide:} Utilizando el modelo de producto multiatributo, diseñar un producto/ servicio de centro de cuidado y atención al adulto mayor que se adecue a necesidades del segmento del mercado descrito.
  \newline
  \textbf{Solución:}
  \begin{itemize}
    \item \textbf{Producto Basico:} Lugar para residencia de adultos mayores.
    \item \textbf{Producto Real:}
    \begin{itemize}
      \item Características:
      \begin{itemize}
        \item Residencia permanente full-time; o part time.
        \item Habitaciones individuales; baño individual; T.V.; Wi-Fi; etc.
        \item Espacios abiertos; jardines; instalación deportiva; etc.
        \item Enfermería; kinesiologia; atenciones especiales; etc.
        \item Clases de gimnasia; juegos de mesas; etc.
      \end{itemize}
      \item Diseño:
      \begin{itemize}
        \item Limpieza; orden; Confort; Luminozidad; Seguridad.
      \end{itemize}
      \item Calidad:
      \begin{itemize}
        \item 90\% Encuesta de satisfacción.
        \item 5\% Retiro de clientes por año.
        \item $ < $ 3 incidencias por mes.
      \end{itemize}
      \item Nombre: \textit{'Residencia Santa Clara'}.
      \item Empaque:
      \begin{itemize}
        \item Contrato.
        \item Folleto informativo.
      \end{itemize}
    \end{itemize}
  \end{itemize}
\end{CuadroPersonalizado}
\subsubsection{Estimación de la demanda}
\textbf{DEF.} Es el volumen total que compraría un grupo definido de consumidores en una zona geográfica definida, en un lapso definido, bajo determinadas condiciones de entorno y sujeto a cierto nivel de esfuerzo promocional de la industria.
\begin{equation*}
  q_t^d \qquad t = 1,2,...,n
\end{equation*}
Para analizar la demanda tiene que estar bien definida la \textbf{unidad de transación} del bien o servicio, teniendo la precaución de que éste coincida con la generación de los ingresos, por unidad de tiempo.
\begin{center}
  \begin{tabular}{lc}
    \hline
    Pan & kgs / mes \\
    Servicios organización eventos & eventos / año \\
    Telefonía & abonos / mes \\
    Museo & visitas / años \\
    \hline
  \end{tabular}
\end{center}
\begin{CuadroPersonalizado}{black}{Comentario del profesor}{Verde2!90!black}{Verde2!90!black}
  \begin{itemize}
    \item A nosotros nos va a interesar la medida anual para la realizacion del proyecto (Evaluación).
    \item En la gran mayoria de productos es necesario aplicar filtros para los calculos.
  \end{itemize}
\end{CuadroPersonalizado}

\textbf{Métodos rápidos de estimación de demanda}
\newparagraph
\begin{itemize}
  \item \textbf{Método de múltiplos del promedio.}
  \begin{equation*}
    q_t^d = N \times \overline{q_t}
  \end{equation*}
  Donde:
  \begin{itemize}
    \item $q_t^d$: Cantidad física demandada por el mercado en un año $t$.
    \item $N$: Cantidad de compradores (c) en el mercado.
    \item $\overline{q_t}$: Cantidad promedio (tasa de uso) del comprador promedio en un periodo $t$.
  \end{itemize}
  \begin{CuadroPersonalizado}{black}{Ejemplo}{Celeste!90!black}{Celeste!90!black}
    Estimar la demanda por carne de vacuno en la ciudad de Chillán.
    \begin{itemize}
      \item $N$ = cantidad de compradores potenciales = 227.634 c.
      \item $\overline{q_t}$ = compras promedio de carne por habitante en un año = 17 kg/c $\cdot$ año
      \item $P$ = precio medio de carne (kg) = 7.248 $\$/kg$
    \end{itemize}
    Demanda anual física estimada:
    \begin{equation*}
      q_t^d = N \times \overline{q_t} = 227.634 \times 17 = 3.869.278 \text{ kg/año}
    \end{equation*}
    Demanda anual en dinero estimada:
    \begin{equation*}
      q_t^v = q_t^d \times P = 3.869.278 \times 7.248 = 28.009.453.160 \text{ \$/año}
    \end{equation*}
  \end{CuadroPersonalizado}
  
  Podemos fácilmente transformar la demanda física a términos monetarios multiplicando la demanda física por el precio del producto ($P$).
  \begin{equation*}
    q_t^v = q_d^t \times P = N \times \overline{q_t} \times P
  \end{equation*}

  \newpage
  \item \textbf{Método de tasas en cadena}
  \newparagraph
  Se utiliza normalmente para calcular la cantidad de compradores potenciales de un producto / servicio a partir de un universo conocido, aplicando sucesivos coeficientes reductores.
  \begin{equation*}
    N = U \times f_1 \times f_2 \times ...
  \end{equation*}
  Donde:
  \begin{itemize}
    \item $N$: Cantidad de compradores potenciales.
    \item $U$: Universo conocido al que pertenecen los compradores
    \item $f_1, f_2, ...$: Coeficientes que representan porcentajes del subconjunto sucesivo de población que posee determinnadas características de interés.
  \end{itemize}
  \begin{CuadroPersonalizado}{black}{Ejemplo}{Celeste!90!black}{Celeste!90!black}
    Se desea estimar el número potencial de compradores de un servicio sanitario para adultos mayores que padecen diabetes.
    \begin{itemize}
      \item Población área metropolitana Santiago (U) = 7.982.808 c
      \item Porcentaje de mayores de 60 años ($f_1$) = 9,85\%
      \item Incidente de la diabetes en mayores de 60 años ($f_2$) = 12\%
    \end{itemize}
    La cantidad potencial de compradores será:
    \begin{equation*}
       N = U \times f_1 \times f_2 = 7.982.808 \times 0,0985 \times 0,12 = 94.357 \text{ c}
    \end{equation*}
  \end{CuadroPersonalizado}

  \item \textbf{Estimación de las ventas del proyecto (empresa)}
  \newparagraph
  La demanda por el producto o demanda industrial debe complementarse con una estimación de las ventas del proyecto. Éstas serán el resultado de la participación de mercado que alcance. Una manera rápida de visualizar la participación de mercado es descomponerla en dos coeficientes:
  \begin{itemize}
    \item Una tasa de acceso ($g_1$), dado por el porcentaje del mercado al que se logrará comunicar efectivamente la oferta del proyecto, de acuerdo al plan de marketing.
    \item Una tasa de aceptación ($g_2$), que comprenderá el porcentaje de los compradores accedidos que realizarán compras.
  \end{itemize}
  Si suponemos que los compradores que aceptan la oferta tienen la misma tasa de uso que todos, entonces podmos estimar las ventas del proyecto como:
  \begin{equation*}
    q_{t,i}^d = q_t^d \times q_{1,i} \times q_{2,i}
  \end{equation*}
  Si multiplicamos por el precio unitario del producto ($P$) obtendremos la estimación de las ventas del proyecto (empresa) en dinero.
  \begin{equation*}
    q_{t,i}^v = q_{t,i}^d \times P
  \end{equation*}
  \begin{CuadroPersonalizado}{black}{Ejemplo}{Celeste!90!black}{Celeste!90!black}
    En el caso de las ventas de carne en la ciudad de Chillán. Suponiendo una tasa de acceso del 40\% y una tasa de aceptación del 8\%, las ventas físicas del proyecto sería:
    \begin{equation*}
      q_{t,i}^d = 3.869.778 \text{kg/año} \times 0,4 \times 0,08 = 123.833 \text{kg/año}
    \end{equation*}
    Lo anterior significará ventas anuales en dinero por:
    \begin{equation*}
      q_{t,i}^v = 123.833 \text{kg/año} \times 5.248 \text{\$/kg} = 649.875.584 \text{\$/año}
    \end{equation*}
  \end{CuadroPersonalizado}
\end{itemize}

\subsection{Pronóstico de la demanda futura}
Una vez obtenida la demanda actual, corresponde, a objeto de proyectar nuestro plan hacia el futuro, estimar los valores de la demanda para los próximos periodos.
\newline
El prónostico de la demanda consiste en estimar valores para la demanda de los períodos sucesivos, anticipando el comportamiento de los consumidores, sujeto a condiciones de entorno y esfuerzo comercial dados.
\newline
\textbf{Según el tipo de recolección podemos identificar los siguientes métodos de pronóstico:}
\begin{table}[H]
  \centering
  \begin{tabular}{|l|l|}
    \hline
    \textbf{MÉTODO DE RECOLECCIÓN} & \textbf{MÉTODO DE PRONÓSTICO} \\
    \hline
    &\\
    Primario. Interrogativo & - Encuesta de intención de compra \\
    & - Panel de vendedores \\
    & - Panel de expertos \\
    &\\
    \hline
    &\\
    Primario. Observación & - Mercado de Prueba \\
    &\\
    \hline
    &\\
    Secundario & - Análisis de regresión \\
    & - Análisis de series de tiempo \\
    & - Indicadores guía \\
    \hline
  \end{tabular}
\end{table}

\newpage
\section{Clase 29-08}
% \begin{CuadroPersonalizado}{black}{Modelos}{Verde2!90!black}{Verde2!90!black}
%   \begin{align*}
%     y_i &= a \cdot e^{b\cdot x_i + \varepsilon_i} \\
%     ln(y_i) &= ln(a) + b \cdot x_i + \varepsilon_i \\
%     \mathbb{E}[ln(y_i)|x_i] &= \widehat{ln(y_i)} = \widehat{ln(a)} + \widehat{b} \cdot x_i \\ 
%     % y_i &= a + b x_i + \varepsilon_i
%   \end{align*}
% \end{CuadroPersonalizado}
\begin{CuadroPersonalizado}{black}{Ejercicio en clase}{Verde2!90!black}{Verde2!90!black}
  La demanda anual de camionetas 4x4 en la región del Bío-Bío ha tenido el siguiente comportamiento en los últimos 7 años.
  \begin{table}[H]
    \centering
    \begin{tabular}{|c|c|}
      \hline
      \textbf{Periodo} & \textbf{Ventas camionetas} \\
      \hline
      1 & 713 \\
      2 & 715 \\
      3 & 777 \\
      4 & 863 \\
      5 & 997 \\
      6 & 1210 \\
      7 & 1591 \\
      \hline
    \end{tabular}
  \end{table}

  Con esta información, se pide:
  \begin{enumerate}
    \item Utilizando el análisis de regresión, determine el modelo (lineal o exponencial) que mejor representa la revolución de la demanda anual de camionetas en la región del Bío-Bío. [Evalúe el ajuste de los datos a cada modelo].
    \begin{itemize}
      \item $\widehat{a} = 431.71$
      \begin{align*}
        \widehat{a} &= \overline{y} - \widehat{b} \cdot \overline{x} \\
        &= 980.857142 - 137.285714 \cdot 4 \\
        &= 431.71
      \end{align*}
      \item $\widehat{b} = 137.28$
      \begin{align*}
        \widehat{b} &= \frac{\widehat{\rho_{xy}} \cdot \widehat{\sigma_x} \cdot \widehat{\sigma_y}}{\widehat{\sigma_x^2}} \\
        &=  \frac{\widehat{\rho_{xy}} \cdot \widehat{\sigma_y}}{\widehat{\sigma_x}} \\
        &= \frac{r \cdot Sy}{Sx} \\
        &= \frac{0.920426 \cdot 322.210298}{2.160246} \\
        &= 137.28
      \end{align*}
      % \item $\sum x_i = 28$
      % \item $\overline{x} = 4$
      % \item $\sigma_x = $
      % \item  
    \end{itemize}
    Donde:
    \begin{itemize}
      \item $\overline{x}$ = Media de años.
    \end{itemize}
  \end{enumerate}
\end{CuadroPersonalizado}

\newpage
\subsection{Estudio Técnico}\hypertarget{ET}{}
\textbf{DEF.}
\begin{itemize}
  \item \textbf{Mostrará:}
  \begin{itemize}
    \item Que el proyecto considera la alternativa tecnolpogica óptima de acuerdo a las posibilidades disponibles.
  \end{itemize}
  \item \textbf{Deberá:}
  \begin{itemize}
    \item Desplegar un completo \textit{'lay out'} los procesos productivos, operacionales y administrativos que se van a implementar.
  \end{itemize}
  \item \textbf{Proveerá:}
  \begin{itemize}
    \item La información necesaria para cuantificar el monto de las inversiones y los costos de operación (ejecución) del proyecto.
  \end{itemize}
\end{itemize}

\newpage
\subsection{Estudio Organizacional}\hypertarget{EO}{}
\textbf{DEF.}
\begin{itemize}
  \item \textbf{Se comprende:}
  \begin{itemize}
    \item La conformación del equipo de personas que desempeñarán las operaciones del proyecto y en definitiva, quienes tendrán la responsabilidad de materializarlo.
  \end{itemize}
  \item Los requirimientos de personal llevan a estimar los costos laborales, un importante ítem de los costos de operación.
  \item \textbf{Permite:}
  \begin{itemize}
    \item Conocer las posibiliades de contar con el personal idóneo para el éxito del proyecto. Dentro de este campo deben considerarse también lo relacionado con los procesos de gestión y la disponibilidad de los sistemas administrativos necesarios para operarlos.
  \end{itemize}
  \item Mención especial debe hacerse al equipo directivo que encabezará la realización del proyecto. Esto es particularmente crítico en los proyectos de nuevos negocios o de formación de nuevas empresas.
\end{itemize}

\newpage
\subsection{Estudio Legal}\hypertarget{EL}{}
\textbf{DEF.}
\begin{itemize}
  \item Este aspecto es crucial en determinados emprendimientos, y consiste en la delimitación de las normas jurídicas que afectarán el desarrollo de las operaciones.
  \item Los aspectos legales más comunes son:
  \begin{itemize}
    \item Normas de constitución y funcionamiento de sociedades y corporaciones.
    \item Legislación de comercio.
    \item Normas de protección al consumidor.
    \item Legislación tributaria, arancelaria y de derechos municipales.
    \item Normas sanitarias, de seguridad, ambientales, de urbanismo.
    \item Normas laborales.
    \item Normas de propiedad industrial y marcas comerciales.
  \end{itemize}
\end{itemize}

\newpage
\subsection{Estudio Económico Financiero}\hypertarget{EEF}{}
\textbf{DEF.}
\begin{itemize}
  \item En este an\'alisis predominan las consideraciones econ\'omicas de costos y beneficios del proyecto en t\'erminos monetarios.
  \item Debe resumirse sistemáticamente toda la información financiera asociada al proyecto la cual es derivada de los estudios mencionados anteriormente.
  \item \textbf{Se identifican:}
  \begin{itemize}
    \item Se identifican las inversiones, los ingresos (beneficios) y los costos operacionales y los valores residuales de las inversiones.
  \end{itemize}
  \item \textbf{Se determina:}
  \begin{itemize}
    \item La forma de financiamiento de las inversiones.
    \item Los flujos de caja que generará la realización del proyecto.
  \end{itemize}
  \item \textbf{Se calculan:}
  \begin{itemize}
    \item Diversos indicadores de evaluación de inversiones como el período de recuperación, el valor actual neto y la tasa interna de retorno.
  \end{itemize}
  \item \textbf{Se analiza:}
  \begin{itemize}
    \item El sensibilidad de los indicadores (principalmente el Valor Actual Neto) a la variación de determinados parámetros.
    \item El nivel de riesgo de la inversión a través de la volatilidad de los flujos de caja.
  \end{itemize}
\end{itemize}

\newpage

\section{Ejercicio}
\noindent
La operación de sellado y escalado requieren de 1 operario para cada máquina. Teniendo en cuenta lo anterior, SE PIDE lo siguiente:
\begin{enumerate}
  \item Elabore un cuadro de requerimientos de recursos para cada una de las etapas de producción señaladas.
\begin{center}
  \begin{tabularx}{\textwidth}{|X|X|X|}
    \hline
    Columna 1 & Columna 2 & Columna 3 \\
    \hline
    Contenido 1 & 
    \begin{itemize}
      \item Elemento 1
      \item Elemento 2
      \item Elemento 3
    \end{itemize} & 
    Contenido 3 \\
    \hline
  \end{tabularx}
  % \begin{tabularx}{\textwidth}{XXXXXX}
  %   \textbf{Operación} & \textbf{Insumos} & \textbf{Suministros} & \textbf{Equipo} & \textbf{Personal} & \textbf{Espacio} \\
  %   \hline
  %   Limpieza & Machas natural & \begin{itemize}
  %     \item Agua potable.
  %     \item EPHP.
  %     \item Electricidad.
  %     \item Mantención herramientas.
  %     \item Elementos de Limpieza.
  %   \end{itemize} & 1 & 1 & 1 m^2 \\
  %   % \end{array} & \begin{array}{c}
  %   %   Estación de trabajo \\
  %   %   Set. herramientas \\
  %   %   Aire acondicionado \\
  %   %   Bandeja y recipientes
  %   % \end{array} & 1 & 1 m^2 \\
  % \end{tabularx}
\end{center}
\end{enumerate}
\end{document}